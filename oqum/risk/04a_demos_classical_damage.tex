The same hazard input as described in the Classical PSHA-based Risk demo was used for this demo. Thus, the workflow to produce the set of hazard curves described in \ref{sec:demos_classical_risk} is also valid herein. Then, to run the Classical PSHA-based Damage demo, users should navigate to the folder containing the demo input models and configuration files and employ the following command:

\begin{Verbatim}[frame=single, commandchars=\\\{\}, samepage=true]
user@ubuntu:~\$ oq-engine --rh job_hazard.ini
\end{Verbatim}

which will produce the following sample hazard output:

\begin{Verbatim}[frame=single, commandchars=\\\{\}, samepage=true]
Calculation 5 results:
  id | output_type | name
 420 | Hazard Curve | Hazard Curve rlz-102-PGA
 419 | Hazard Curve (multiple imts) | hc-multi-imt-rlz-102
\end{Verbatim}

In this demo, the damage distribution for each asset in the exposure model is produced. The following command launches the risk calculations:

\begin{Verbatim}[frame=single, commandchars=\\\{\}, samepage=true]
user@ubuntu:~\$ oq-engine --rr job_risk.ini --hazard-output-id 419
\end{Verbatim}

and the following sample outputs are obtained:

\begin{Verbatim}[frame=single, commandchars=\\\{\}, samepage=true]
Calculation 6 results:
 id | output_type | name
 421 | Damage Per Asset | Damage distribution for hazard=419
\end{Verbatim}