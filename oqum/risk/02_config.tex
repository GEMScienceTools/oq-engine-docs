This Chapter summarises the structure of the information necessary to define
the different input data to be used with the OpenQuake-engine risk
calculators. Input data for scenario-based and probabilistic seismic damage
and risk analysis using OpenQuake-engine are organised into:

\begin{itemize}

  \item An exposure model file in the NRML format, as described in 
    Section~\ref{sec:exposure}.

  \item A file describing the \gls{vulnerability model}
    (Section~\ref{sec:vulnerability}) for loss calculations, or a 
  	file describing the \gls{fragility model} (Section~\ref{sec:fragility})
    for damage calculations. Optionally, a file describing the
    \gls{consequence model} (Section~\ref{sec:consequence}) can also be
  	provided in order to calculate losses from the estimated damage
  	distributions.

  \item A general calculation configuration file.

  \item Hazard inputs. These include hazard curves for the classical
    probabilistic damage and risk calculators, ground motion fields for the
    scenario damage and risk calculators, or stochastic event sets for the
    probabilistic event based calculators. As of OpenQuake-engine v1.6, in
    general, there are five different ways in which hazard calculation
    parameters or results can be provided to the OpenQuake-engine in order to
    run the subsequent risk calculations:

    \begin{itemize}

      \item Use a single configuration file for running the hazard and risk
      calculations sequentially

      \item Use separate configuration files for running the hazard and risk
      calculations sequentially

      \item Use a configuration file for the risk calculation along with all
      hazard outputs from a previously completed, compatible
      OpenQuake-engine hazard calculation

      \item Use a configuration file for the risk calculation along with a
      specific hazard output from a previously completed, compatible
      OpenQuake-engine hazard calculation

      \item Use a configuration file for the risk calculation along with
      hazard input files in the OpenQuake NRML format

    \end{itemize}

\end{itemize}

The file formats for exposure, fragility, consequence, and vulnerability
models have been described earlier in Chapter~\ref{chap:riskintro}. The
configuration file is the primary file that provides the OpenQuake-engine
information regarding both the definition of the input models (e.g. exposure,
site parameters, fragility, consequence, or vulnerability models) as well as
the parameters governing the risk calculation.

Information regarding the configuration file for running hazard calculations
using OpenQuake-engine can be found in
Section~\ref{sec:hazard_configuration_file}. Some initial parameters of the
configuration file common to all of the risk calculators are presented below.
The remaining parameters that are specific to each risk calculator are
discussed in subsequent sections.

\inputminted[firstline=1,firstnumber=1,fontsize=\footnotesize,frame=single,linenos,bgcolor=lightgray]{ini}{oqum/risk/Verbatim/config_example.ini}\\

\begin{itemize}

  \item \Verb+description+: a parameter that can be used to include some
  information about the type of calculations that are going to be performed.

  \item \Verb+calculation_mode+: this parameter specifies the type of
  calculation to be run. Valid options for the \Verb+calculation_mode+ for
  the risk calculators are: \Verb+scenario_damage+, \Verb+scenario_risk+,
  \Verb+classical_damage+, \Verb+classical_risk+, \Verb+event_based_risk+,
  and \Verb+classical_bcr+.

  \item \Verb+exposure_file+: this parameter is used to specify the path to
  the \gls{exposure model} file.

\end{itemize}

Depending on the type of risk calculation, other parameters besides the
aforementioned ones may need to be provided. We illustrate in the following
sections different examples of the configuration file for the different risk
calculators.


\section{Scenario Damage Calculator}
\label{sec:config_scenario_damage}
For this calculator, the parameter \Verb+calculation_mode+ should be set to
\Verb+scenario_damage+.

\paragraph{Example 1}

This example illustrates a scenario damage calculation which uses a single
configuration file to first compute the ground motion fields for the given
rupture model and then calculate damage distribution statistics based on the
ground motion fields. A minimal job configuration file required for
running a scenario damage calculation is shown below:

\inputminted[firstline=1,firstnumber=1,fontsize=\footnotesize,frame=single,linenos,bgcolor=lightgray]{ini}{oqum/risk/Verbatim/config_scenario_damage_combined.ini}\\

The general parameters \Verb+description+ and \Verb+calculation_mode+, and
\Verb+exposure_file+ have already been described earlier. The other parameters
seen in the above example configuration file are described below:

\begin{itemize}

  \item \Verb+rupture_model_file+: a parameter used to define the path
	to the earthquake \gls{rupture model} file describing the scenario event.

  \item \Verb+rupture_mesh_spacing+: a parameter used to specify the mesh size
  	(in km) used by OpenQuake-engine to discretize the rupture.
  	Note that smaller the mesh spacing, greater are
  	(1) the precision in the calculation and
  	(2) the computational demand.

  \item \Verb+structural_fragility_file+: a parameter used to define the path
	to the structural \gls{fragility model} file.

\end{itemize}

\section{Scenario Risk Calculator}
\label{sec:config_scenario_risk}
In order to run this calculator, the parameter \Verb+calculation_mode+ needs
to be set to \Verb+scenario_risk+. Most of the job configuration parameters 
required for running a scenario risk calculation are the same as those 
described in the previous section for the scenario damage calculator.
The remaining parameters specific to the scenario risk calculator 
are illustrated below.

\begin{Verbatim}[frame=single, commandchars=\\\{\}, samepage=true]
...
structural_vulnerability_file = structural_vulnerability_model.xml
nonstructural_vulnerability_file = nonstructural_vulnerability_model.xml
contents_vulnerability_file = contents_vulnerability_model.xml
business_interruption_vulnerability_file = downtime_vulnerability_model.xml
occupants_vulnerability_file = occupants_vulnerability_model.xml

asset_correlation = 0.7
master_seed = 3
insured_losses = true
\end{Verbatim}

\begin{itemize}

  \item \Verb+structural_vulnerability_file+: this parameter is used to
    specify the path to the structural \gls{vulnerability model} file

  \item \Verb+nonstructural_vulnerability_file+: this parameter is used to
    specify the path to the nonstructural\gls{vulnerability model} file

  \item \Verb+contents_vulnerability_file +: this parameter is used to
    specify the path to the contents \gls{vulnerability model} file

  \item \Verb+business_interruption_vulnerability_file +: this parameter is
    used to specify the path to the business interruption
    \gls{vulnerability model} file

  \item \Verb+occupants_vulnerability_file+: this parameter is used to
    specify the path to the occupants \gls{vulnerability model} file

  \item \texttt{asset\_correlation} if the uncertainty in the loss ratios
    has been defined within the \gls{vulnerability model}, users can specify
    a coefficient of correlation that will be used in the Monte Carlo sampling
    process of the loss ratios, between the assets that share the same
    \gls{taxonomy}. If the \texttt{asset\_correlation} is set to one,
    the loss ratio residuals will be perfectly correlated. On the other hand,
    if this parameter is set to zero, the loss ratios will be sampled
    independently. Any value between zero and one will lead to increasing
    levels of correlation. If this parameter is not defined, the
    OpenQuake-engine will assume zero correlation in the vulnerability.

  \item \Verb+master_seed+: this parameter is used to control the random
    number generator in the loss ratio sampling process. If the same
    \Verb+master_seed+ is defined at each calculation run, the same random loss
    ratios will be generated, thus allowing reproducibility of the results.

  \item \Verb+insured_losses+: this parameter specifies whether insured losses
    should be calculated (\Verb+true+) or not (\Verb+false+).

\end{itemize}

\section{Classical Probabilistic Seismic Damage Calculator}
\label{sec:config_classical_damage}
In order to run this calculator, the parameter \Verb+calculation_mode+ needs to be set to \Verb+classical_damage+. Similar to the Scenario Damage calculator, there is only one parameter specific to this calculator, which is the \gls{fragility model} file path, as presented below.
\begin{Verbatim}[frame=single, commandchars=\\\{\}, samepage=true]
...
fragility_file = fragility_model.xml
\end{Verbatim}

\section{Classical Probabilistic Seismic Risk Calculator}
\label{sec:config_classical_risk}
In order to run this calculator, the parameter \Verb+calculation_mode+ needs
to be set to \Verb+classical_risk+.

Most of the job configuration parameters required for running a classical
probabilistic risk calculation are the same as those described in the previous
section for the classical probabilistic damage calculator. The remaining
parameters specific to the classical probabilistic risk calculator are
illustrated through the examples below.

\paragraph{Example 1}

This example illustrates a classical probabilistic risk calculation which uses
a single configuration file to first compute the hazard curves for the given
source model and ground motion model and then calculate loss exceedance curves
and maps based on the hazard curves. A minimal job configuration file required
for running a classical probabilistic risk calculation is shown below:

\inputminted[firstline=1,firstnumber=1,fontsize=\footnotesize,frame=single,linenos,bgcolor=lightgray,label=job.ini]{ini}{oqum/risk/verbatim/config_classical_risk_combined.ini}\\

Apart from the calculation mode, the only difference with the example job
configuration file shown in Example~1 of
Section~\ref{sec:config_classical_damage} is the use of a vulnerability model
instead of a fragility model.

As with the Scenario Risk calculator, it is possible to specify one or more
vulnerability model files in the same job configuration file, using the
parameters:

\begin{itemize}

  \item \Verb+structural_vulnerability_file+,

  \item \Verb+nonstructural_vulnerability_file+,

  \item \Verb+contents_vulnerability_file+,

  \item \Verb+business_interruption_vulnerability_file+, and/or

  \item \Verb+occupants_vulnerability_file+

\end{itemize}

It is important that the
\Verb+lossCategory+ parameter in the metadata section for each provided
vulnerability model file (``structural'', ``nonstructural'', ``contents'',
``business\_interruption'', or ``occupants'') should match the loss type
defined in the configuration file by the relevant keyword above.

In this case, the hazard curves will be computed at each of the locations of
the assets in the exposure model, for each of the intensity measure types
found in the provided set of vulnerability models. The above calculation can
be run using the command line:

\begin{Verbatim}[frame=single, commandchars=\\\{\}, samepage=true]
user@ubuntu:~\$ oq-engine --run job.ini
\end{Verbatim}

After the calculation is completed, a message similar to the following will be
displayed:

\begin{Verbatim}[frame=single, commandchars=\\\{\}, samepage=true]
Calculation 2749 completed in 24 seconds. Results:
  id | output_type | name
5373 | Loss Curve | loss curves. type=structural, hazard=5371
5374 | Loss Curve | loss curves. type=nonstructural, hazard=5371
5375 | Loss Curve | insured loss curves. type=structural hazard=5371
\end{Verbatim}


\section{Event-Based Probabilistic Seismic Risk Calculator}
\label{sec:config_event_based_risk}
The parameter \Verb+calculation_mode+ needs to be set to
\Verb+event_based_risk+ in order to use this calculator. Similarly to that
described for the Scenario Risk Calculator, a Monte Carlo sampling process is
also employed within this module to take into account the loss ratio
uncertainty. Hence, the parameters \Verb+asset_correlation+ and
\Verb+master_seed+ need to be defined as previously described. This calculator
is also capable of estimating insured losses and therefore, the
\Verb+insured_losses+ attribute can be specified as well. The parameter
``risk\_investigation\_time'' specifies the time period for which the event
loss tables and loss exceedance curves will be calculated. If this parameter
is not provided in the risk job configuration file, the time period used is
the same as that specifed in the hazard calculation.

The remaining parameters are presented below.

\begin{Verbatim}[frame=single, commandchars=\\\{\}, samepage=true]
...
structural_vulnerability_file = struct_vul_model.xml
nonstructural_vulnerability_file = nonstruct_vul_model.xml
contents_vulnerability_file = cont_vul_model.xml
business_interruption_vulnerability_file = bus_int_vul_model.xml
occupants_vulnerability_file = occ_vul_model.xml

asset_correlation = 0.7
master_seed = 3
insured_losses = true

specific_assets = asset_11 asset_132 asset_303 asset_611

loss_curve_resolution = 20
conditional_loss_poes = 0.01, 0.05, 0.10
\end{Verbatim}

\begin{itemize}

  \item \Verb+loss_curve_resolution+: since this calculator uses an event-based
    approach, a large number of levels of loss (and associated probabilities of
    exceedance) is computed (one per event) for each asset. The oq-risklib will
    use this large set of results to extrapolate a loss curve, whose number of
    points are controlled by this parameter. By default, the OpenQuake-engine
    assumes the \Verb+loss_curve_resolution+ equal to 20.

  \item \Verb+risk_investigation_time+: this parameter specifies the time
    period for which the event loss tables and loss exceedance curves will be
    calculated. If this parameter is not provided in the risk job configuration
    file, the time period used is the same as that specifed in the hazard
    calculation.

  \item \Verb+conditional_loss_poes+: this parameter is used to define the
    probabilities of exceedance at which loss maps are to be produced.

  \item \Verb+specific_assets+: this parameter specifies the set of assets for
    which the asset event loss table will be generated.

\end{itemize}

\section{Retrofit Benefit-Cost Ratio Calculator}
\label{sec:config_benefit_cost}
As previously explained, this calculator uses loss exceedance curves which are
calculated using the Classical PSHA-based Risk calculator. The
\Verb+calculation_mode+ should be set to \Verb+classical_bcr+ and the
calculator-specific part of the configuration file should be defined as
presented below.

\begin{Verbatim}[frame=single, commandchars=\\\{\}, samepage=true]
...
structural_vulnerability_file = structural_vulnerability_model.xml
vulnerability_retrofitted_file = retrofitted_vulnerability_model.xml

lrem_steps_per_interval = 2

interest_rate = 0.05
asset_life_expectancy = 50
\end{Verbatim}

\begin{itemize}
\item  \Verb+vulnerability_retrofitted_file+: this parameter is used to specify the path to the \gls{vulnerability model} file containing the \glspl{vulnerability function} for the retrofitted assets;
\item  \Verb+interest_rate+: this parameter represents the interest rate and it serves the purposes of taking into account the variation of building value throughout time;
\item  \Verb+asset_life_expectancy+: this variable defines the life expectancy, or design life, of the assets.
\end{itemize}

\section{Running the Risk Calculators}
\label{sec:running_risk}
Using the command line interface, risk calculations can be launched and the
resulting outputs can be extracted. This section describes all the currently
implemented commands and presents examples for each of the calculators. One of
the first tasks that needs to be performed is the definition of the seismic
hazard input.

As mentioned in Chapter~\ref{chap:riskinputs}, the risk calculations can use
the results produced by the hazard component of the OpenQuake-engine.
Moreover, for the two scenario-based calculators, users also have the option
of loading a set of ground motion fields that might have been produced using
the OpenQuake-engine, or other software.

In order to load ground motion fields based on a single earthquake event, it
is important to ensure that the ground motion values have been saved
according to the NRML schema as presented in
Section~\ref{subsec:output_event_based_psha}. Then, the following line can be
included in the risk job configuration file:

\begin{Verbatim}[frame=single, commandchars=\\\{\}, samepage=true]
gmfs_file = gmfs_filename.xml
\end{Verbatim}

Whether a user chooses to load pre-computed ground motion fields, or calculate
this input using the hazard component of the OpenQuake-engine, a unique
\verb+id+ is associated to the set of ground motion fields, as depicted below.

\begin{Verbatim}[frame=single, commandchars=\\\{\}, samepage=true]
Calculation 3 results:
id | output_type | name
12 | gmf_scenario | gmf_scenario
\end{Verbatim}

This is the parameter that will be used when launching the risk calculations
to indicate which hazard input should be employed. In the case of the
scenario-based calculators, there is only a single hazard input (one or a set
of ground motion fields). For the remaining calculators, where probabilistic
seismic hazard is used, it is possible to have multiple hazard inputs due to
the employment of logic trees, as described in
Section~\ref{sec:hazard_logic_trees}. In the following illustration, a set of
hazard results produced using the Classical PSHA calculator is presented.

\begin{Verbatim}[frame=single, commandchars=\\\{\}, samepage=true]
Calculation 4 results:
id | output_type | name
32 | hazard_curve | hc-rlz-32-PGA
33 | hazard_curve | hc-rlz-33-PGA
34 | hazard_curve | hc-rlz-34-PGA
35 | hazard_curve | hc-rlz-35-PGA
36 | hazard_curve | mean curve for PGA
37 | hazard_curve | quantile curve (poe>= 0.15) for imt PGA
38 | hazard_curve | quantile curve (poe>= 0.85) for imt PGA
\end{Verbatim}

In this case, since the logic tree had four branches, fours sets of hazard
curves were produced, each one with its own \verb+id+. In addition, mean and
quantile hazard curves were also produced. A user may choose to run risk
calculations using results from one of the branches or mean/quantile curves.
To do so, the id of the respective hazard result should be employed when
launching the risk calculations, as depicted below.

\begin{Verbatim}[frame=single, commandchars=\\\{\}, samepage=true]
user@ubuntu:~\$ oq-engine --run-risk job.ini --hazard-output-id
<hazard_output_id>
\end{Verbatim}

or simply:

\begin{Verbatim}[frame=single, commandchars=\\\{\}, samepage=true]
user@ubuntu:~\$ oq-engine --rr job.ini --ho <hazard_output_id>
\end{Verbatim}

On the other hand, a user might also want to run the risk calculations
considering all the hazard results from a certain calculation run. In this
case, rather than providing the \verb+hazard-output-id+, users need to provide
the id of the hazard calculation as follows.

\begin{Verbatim}[frame=single, commandchars=\\\{\}, samepage=true]
user@ubuntu:~\$ oq-engine --run-risk job.ini --hazard-calculation-id
<hazard_calculation_id>
\end{Verbatim}

or simply:

\begin{Verbatim}[frame=single, commandchars=\\\{\}, samepage=true]
user@ubuntu:~\$ oq-engine --rr job.ini --co <hazard_calculation_id>
\end{Verbatim}

\cleardoublepage