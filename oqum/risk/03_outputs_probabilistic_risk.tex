\subsection{Loss exceedance curves}

Loss exceedance curves can be calculated using the Classical PSHA-based Risk
or Probabilistic Event-based Risk calculators, and they provide the
probability of exceeding a set of levels of loss, within a given time span (or
investigation interval). Similarly to what has been described for the
probabilistic loss maps, also here it is necessary to define the parameters
\Verb+investigationTime+, \Verb+sourceModelTreePath+, \Verb+gsimTreePath+ and
\Verb+unit+. Then, the set of loss exceedance curves are stored as presented
in the following example.

\begin{Verbatim}[frame=single, commandchars=\\\{\}, samepage=false]
\textcolor{gray}{<?xml version="1.0" encoding="UTF-8"?>}
<nrml xmlns:gml="http://www.opengis.net/gml"
<\textcolor{red}{lossCurves} investigationTime="50" sourceModelTreePath="b1"
    gsimTreePath="b1" unit="EUR">
     <\textcolor{green}{lossCurve} assetRef="a1">
          <gml:Point>
            <gml:pos>83.31 29.46</gml:pos>
          </gml:Point>
          <\textcolor{blue}{poE}> 0.970 0.297 0.137 0.019 0.005 0.001</\textcolor{blue}{poE}>
          <\textcolor{blue}{losses}> 235 477 989 4102 7444 15631</\textcolor{blue}{losses}>
          <\textcolor{blue}{lossRatios}> 0.02 0.03 0.06 0.26 0.48 1.0</\textcolor{blue}{lossRatios}>
     <\textcolor{green}{/lossCurve}>
     ...
     <\textcolor{green}{lossCurve} assetRef="a999">
          <gml:Point>
            <gml:pos>83.33 28.71</gml:pos>
          </gml:Point>
          <\textcolor{blue}{poE}> 0.99 0.714 0.112 0.020 0.004 0.001</\textcolor{blue}{poE}>
          <\textcolor{blue}{losses}>58 402 819 3664 8001 13540</\textcolor{blue}{losses}>
          <\textcolor{blue}{lossRatios}> 0.02 0.04 0.07 0.32 0.59 1.0</\textcolor{blue}{lossRatios}>
     <\textcolor{green}{/lossCurve}>
<\textcolor{red}{/lossCurves}>
</nrml>
\end{Verbatim}

Each \Verb+lossCurve+ is associated with a location (defined within the
\Verb+gml:Point+ attribute) and a reference to the \gls{asset}
(\Verb+assetRef+) whose loss is being represented. Then, three lists of values
are stored: the probabilities of exceedance (\Verb+poE+), levels of absolute
loss (\Verb+losses+) and percentages of loss (\Verb+lossRatios+).


\subsection{Probabilistic loss maps}

A loss map contains the spatial distribution of the losses throughout the
region of interest. This result can be produced by the Scenario Risk
calculator (representing the losses from a single event), or from the
Probabilistic Event-based Risk or Classical PSHA-based Risk calculators
(representing the expected losses from probabilistic seismic hazard). In the
former case, the loss map is comprised of a mean loss and respective standard
deviation for each \gls{asset}, whilst for the latter, a single value is
provided, representing the expected loss for a given return period (or
probability of exceedance for a certain time span, or investigation interval).
In the following example, a loss map due to a single earthquake is presented.

\begin{Verbatim}[frame=single, commandchars=\\\{\}, samepage=false]
\textcolor{gray}{<?xml version="1.0" encoding="UTF-8"?>}
<nrml xmlns:gml="http://www.opengis.net/gml"
      xmlns="http://openquake.org/xmlns/nrml/0.5">
<\textcolor{red}{lossMap} lossCategory="buildings" unit="EUR">
     <\textcolor{green}{node}>
          <gml:Point>
            <gml:pos>83.31 29.46</gml:pos>
          </gml:Point>
          \textcolor{blue}{loss} assetRef="a1" mean="53.3" stdDev="109.25"/>
          \textcolor{blue}{loss} assetRef="a2" mean="386.0" stdDev="695.7"/>
          \textcolor{blue}{loss} assetRef="a3" mean="303.1" stdDev="447.4"/>
          \textcolor{blue}{loss} assetRef="a4" mean="298.9" stdDev="453.7"/>
     <\textcolor{green}{/node}>
    ...
     <\textcolor{green}{node}>
          <gml:Point>
            <gml:pos>83.33 28.71</gml:pos>
          </gml:Point>
          \textcolor{blue}{loss} assetRef="a997" mean="277.3" stdDev="100.8"/>
          \textcolor{blue}{loss} assetRef="a998" mean="219.6" stdDev="123.5"/>
          \textcolor{blue}{loss} assetRef="a999" mean="576.3" stdDev="210.9"/>
     <\textcolor{green}{/node}>
<\textcolor{red}{/lossMap}>
</nrml>
\end{Verbatim}

\begin{itemize}
\item  \Verb+lossCategory+: the type of losses that are being stored. This parameter is taken from the \gls{vulnerability model} that was used in the loss calculations (e.g. fatalities, economic loss);
\item  \Verb+unit+: this attribute is used to define the units in which the losses are being measured (e.g. EUR);
\item  \Verb+node+: each loss map is comprised by various nodes, each node possibly containing a number of \glspl{asset}. The location of the node is defined by a latitude and longitude in decimal degrees within the field \Verb+gml:Point+. The mean loss (\Verb+mean+) and associated standard deviation (\Verb+stdDev+) for each \gls{asset} (identified by the parameter \Verb+assetRef+) is stored in the \Verb+loss+ field.
\end{itemize}

For the probabilistic loss maps (expected losses for a given return period), a
set of additional parameters need to be considered as depicted in the
following example.

\begin{Verbatim}[frame=single, commandchars=\\\{\}, samepage=false]
\textcolor{gray}{<?xml version="1.0" encoding="UTF-8"?>}
<nrml xmlns:gml="http://www.opengis.net/gml"
      xmlns="http://openquake.org/xmlns/nrml/0.5">
<\textcolor{red}{lossMap} investigationTime="50" poE="0.1" sourceModelTreePath="b1"
        gsimTreePath="b1" lossCategory="buildings" unit="EUR">
     <\textcolor{green}{node}>
          <gml:Point>
            <gml:pos>83.31 29.46</gml:pos>
          </gml:Point>
          \textcolor{blue}{loss} assetRef="a1" value="696.1"/>
          \textcolor{blue}{loss} assetRef="a2" value="4201.4"/>
          \textcolor{blue}{loss} assetRef="a3" value="2666.0"/>
          \textcolor{blue}{loss} assetRef="a4" value="1291.8"/>
     <\textcolor{green}{/node}>
    ...
     <\textcolor{green}{node}>
          <gml:Point>
            <gml:pos>83.33 28.71</gml:pos>
          </gml:Point>
          \textcolor{blue}{loss} assetRef="a997" value="4077.3"/>
          \textcolor{blue}{loss} assetRef="a998" value="2466.4"/>
          \textcolor{blue}{loss} assetRef="a999" value="4434.5"/>
     <\textcolor{green}{/node}>
<\textcolor{red}{/lossMap}>
</nrml>
\end{Verbatim}

\begin{itemize}
\item  \Verb+investigationTime+: time span used to compute the probability of exceedance;
\item  \Verb+poE+: parameter specifying the probability of exceedance (e.g. 0.1);
\item  \Verb+sourceModelTreePath+: this is a parameter indicating the path used to create the seismic source model;
\item  \Verb+gsimTreePath+: this parameter designates the ground motion model;
\item  \Verb+node+: this attribute follows an identical structure as seen in the previous example, but only a single loss (\Verb+value+) is provided per \gls{asset}.
\end{itemize}


\subsection{Stochastic event loss tables}

Unlike what was described for the other outputs, the event loss tables are
exported using a comma separated value (.csv) file format. In this structure,
each row contains the rupture id, magnitude and aggregated loss (sum of the
losses from the collection of assets within the region of interest), for each
event within the stochastic event sets. The following example depicts an
example of this output.

\begin{Verbatim}[frame=single, commandchars=\\\{\}, samepage=false]
Rupture,Magnitude,Aggregate Loss
1,8.25,79197
2,8.25,74478
3,7.75,46458
4,7.75,45153
5,7.75,42569
6,8.25,40649
7,7.75,38868
8,7.75,37707
9,7.75,37141
...
\end{Verbatim}

The asset event loss tables provide calculated losses for each of the assets
in the exposure model, for each event within the stochastic event sets. In
these tables, each row contains the rupture id, the asset id, magnitude and
asset loss. Note that only assets that sustain nonzero losses in a rupture are
listed in the asset event loss table. An example of this output type is
provided below:

\begin{Verbatim}[frame=single, commandchars=\\\{\}, samepage=false]
Rupture, Asset, Magnitude, Loss
col=00|ses=0239|src=1727-356|rup=001-01, asset_10, 4.7, 13785407.3791
col=00|ses=0239|src=1727-356|rup=001-01, asset_11, 4.7, 17124507.6341
col=00|ses=0239|src=1727-356|rup=001-01, asset_12, 4.7, 16654913.6026
.
.
.
col=01|ses=6905|src=1120-2|rup=011-01, asset_53, 6.7, 18506576.0384
col=01|ses=6905|src=1120-2|rup=011-01, asset_58, 6.7, 12095498.7066
\end{Verbatim}