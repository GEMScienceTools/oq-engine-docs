In this section, the NRML schema for the \gls{vulnerability model} is described in detail. In order to do so, a graphical representation of a \gls{vulnerability model} (mean loss ratio for a set of intensity measure levels) is illustrated in Figure~\ref{fig:vulModel}, and the equivalent NRML file is then presented. Note that although the uncertainty for each loss ratio is not represented in the aforementioned figure, it has been considered in the input NRML file, by means of a coefficient of variation per loss ratio and a probabilistic distribution, which can currently be set to lognormal or beta. This model is comprised of two discrete \glspl{vulnerability function} and uses spectral acceleration for a given period of vibration as the intensity measure type.

\begin{figure}[ht]
\centering
\includegraphics[width=10cm,height=6cm]{figures/risk/vulnerabilityModel.pdf}
\caption{Graphical representation of a vulnerability model.}
\label{fig:vulModel}
\end{figure}

Each component of the associated NRML file is presented herein:

\begin{Verbatim}[frame=single, commandchars=\\\{\}, samepage=true]
<?xml version="1.0" encoding="UTF-8"?>
<nrml xmlns:gml="http://www.opengis.net/gml"
      xmlns="http://openquake.org/xmlns/nrml/0.4">
<\textcolor{red}{vulnerabilityModel}>
    <\textcolor{green}{discreteVulnerabilitySet} vulnerabilitySetID="OpenQuake2013"	
    assetCategory="buildings"    lossCategory="economic loss">
        ...
\end{Verbatim}

At the top of the NRML schema, the following metadata are being stored:
\begin{itemize}
\item  \Verb+vulnerabilitySetID+: A unique key used to identify the \gls{vulnerability model} instance within the OpenQuake-engine;
\item  \Verb+assetCategory+: An attribute that describes the asset typology (e.g.: population, buildings, contents);
\item  \Verb+lossCategory+: An attribute that describes the type of loss being modelled for the assetCategory (e.g. fatalities, structural replacement cost, contents replacement cost).
\end{itemize}

\begin{Verbatim}[frame=single, commandchars=\\\{\}, samepage=true]
    ...
        <\textcolor{blue}{IML}  IMT = "SA(0.3)"> 0.061 0.129 0.188 0.273 0.398 0.579 
        0.843 1.227 1.856 2.485 <\textcolor{blue}{/IML}>
        ...
\end{Verbatim}

Within this component, an attribute specifying the intensity measure type (e.g.: Sa, PGA, MMI) is defined, followed by the list of intensity measure levels. This set of values is common to all of the \glspl{vulnerability function} in the model.

\begin{Verbatim}[frame=single, commandchars=\\\{\}, samepage=true]
        ...
        <\textcolor{blue}{discreteVulnerability}  vulnerabilityFunctionID="typeA" 
        probabilisticDistribution="LN">
            <\textcolor{magenta}{lossRatio}> 0.002 0.007 0.014 0.028 0.058 0.118
            0.223 0.370 0.446 0.523 <\textcolor{magenta}{/lossRatio}>
            <\textcolor{magenta}{coefficientsVariation}> 0.012 0.058 0.079 0.159 0.265
            0.244 0.211 0.152 0.088 0.082 <\textcolor{magenta}{/coefficientsVariation}>
        <\textcolor{blue}{/discreteVulnerability}>
        <\textcolor{blue}{discreteVulnerability}  vulnerabilityFunctionID="typeB"
        probabilisticDistribution="LN">
            <\textcolor{magenta}{lossRatio}> 0.006 0.025 0.052 0.108 0.215 0.391
            0.613 0.820 0.894 0.967 <\textcolor{magenta}{/lossRatio}>
            <\textcolor{magenta}{coefficientsVariation}> 0.010 0.054 0.082 0.167 0.285
            0.278 0.261 0.132 0.084 0.021 <\textcolor{magenta}{/coefficientsVariation}>
        <\textcolor{blue}{/discreteVulnerability}>
    <\textcolor{green}{/discreteVulnerabilitySet}
<\textcolor{red}{/vulnerabilityModel}>
</nrml>
\end{Verbatim}

Finally, for each discrete \gls{vulnerability function} the following parameters are required:
\begin{itemize}
\item  \Verb+ vulnerabilityFunctionID +: A unique key that is used to relate each \gls{vulnerability function} with the \glspl{asset} in the \gls{exposure model};
\item  \Verb+ probabilisticDistribution +: An attribute that establishes the type of probabilistic distribution used to model the uncertainty in loss ratio. At the moment, the OpenQuake-engine supports lognormal (\Verb+LN+) and beta (\Verb+BT+) distributions;
\item  \Verb+ lossRatio +: A set of mean loss ratios (one for each intensity measure level defined previously). These values can represent different losses such as fatality rates (ratio between the number of fatalities and total population exposed) or so-called damage ratio (ratio between the repair cost and the replacement cost of a given structure).
\item  \Verb+ coefficientsVariation +: A set of coefficients of variation (one per loss ratio) that describes the uncertainty in the loss ratio. If users do not want to consider the uncertainty, this set of parameters can be set to zero, and the OpenQuake-engine assumes each loss ratio as a deterministic value.
\end{itemize}

In the previously described \gls{vulnerability model} all of the \glspl{vulnerability function} were defined in terms of a single intensity measure type (Sa for 0.3 seconds). However, the current version of the engine also allows the employment of a \gls{vulnerability model} that is comprised of \glspl{vulnerability function} that each use distinct intensity measure types. In the following example, the schema of a \gls{vulnerability model} in which three intensity measure types were used (PGA, PGV and Sa for 0.3 seconds) is presented.

\begin{Verbatim}[frame=single, commandchars=\\\{\}, samepage=false]
<?xml version="1.0" encoding="UTF-8"?>
<nrml xmlns:gml="http://www.opengis.net/gml"
      xmlns="http://openquake.org/xmlns/nrml/0.4">
<\textcolor{red}{vulnerabilityModel}>
    <\textcolor{green}{discreteVulnerabilitySet} vulnerabilitySetID="Nepal13_PGA"
    assetCategory="buildings"    lossCategory="economic loss">
        <\textcolor{blue}{IML}  IMT = "PGA"> 0.1 0.2 0.4 0.7 1.0 1.3 <\textcolor{blue}{/IML}>
       <\textcolor{blue}{discreteVulnerability}  vulnerabilityFunctionID="RC1"
        probabilisticDistribution="LN">
            <\textcolor{magenta}{lossRatio}> 0.02 0.1 0.3 0.6 0.8 0.9 <\textcolor{magenta}{/lossRatio}>
            <\textcolor{magenta}{coefficientsVariation}> 0.7 0.5 0.3 0.2 0.1 0.05
            <\textcolor{magenta}{/coefficientsVariation}>
        <\textcolor{blue}{/discreteVulnerability}>
    <\textcolor{green}{/discreteVulnerabilitySet}
    <\textcolor{green}{discreteVulnerabilitySet} vulnerabilitySetID="Nepal13_PGV"
    assetCategory="buildings"    lossCategory="economic loss">
        <\textcolor{blue}{IML}  IMT = "PGV"> 5 20 40 60 80 100 <\textcolor{blue}{/IML}>
       <\textcolor{blue}{discreteVulnerability}  vulnerabilityFunctionID="RC2"
        probabilisticDistribution="LN">
            <\textcolor{magenta}{lossRatio}> 0.05 0.2 0.3 0.4 0.5 0.6 <\textcolor{magenta}{/lossRatio}>
            <\textcolor{magenta}{coefficientsVariation}> 0.6 0.3 0.2 0.1 0.05 0.05
            <\textcolor{magenta}{/coefficientsVariation}>
        <\textcolor{blue}{/discreteVulnerability}>
    <\textcolor{green}{/discreteVulnerabilitySet}
    <\textcolor{green}{discreteVulnerabilitySet} vulnerabilitySetID="Nepal13_SA"
    assetCategory="buildings"    lossCategory="economic loss">
        <\textcolor{blue}{IML}  IMT = "SA(0.3)"> 0.1 0.3 0.6 0.9 1.2 1.5 <\textcolor{blue}{/IML}>
       <\textcolor{blue}{discreteVulnerability}  vulnerabilityFunctionID="RC3" 
        probabilisticDistribution="LN">
            <\textcolor{magenta}{lossRatio}> 0.01 0.06 0.12 0.17 0.24 0.33 <\textcolor{magenta}{/lossRatio}>
            <\textcolor{magenta}{coefficientsVariation}> 1.5 1.1 1.0 0.9 0.8 0.5
            <\textcolor{magenta}{/coefficientsVariation}>
        <\textcolor{blue}{/discreteVulnerability}>
    <\textcolor{green}{/discreteVulnerabilitySet}
<\textcolor{red}{/vulnerabilityModel}>
</nrml>
\end{Verbatim}

Several methodologies to derive vulnerability functions are currently being evaluated by \gls{acr:gem} and will be a part of a set of modelling tools. Scripts to convert \glspl{vulnerability function} stored in Excel or ASCII files into NRML have already being developed, and can be found at the GEM Science tools repository at GitHub (\textcolor{blue}{\Verb+http://github.com/GEMScienceTools+}).