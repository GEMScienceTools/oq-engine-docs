The Scenario Risk Calculator produces the following set of output files for
all loss types (amongst ``structural'', ``nonstructural'', ``contents'',
``occupants'', or ``business\_interruption'') for which a vulnerability model
file was provided in the configuration file:

\begin{enumerate}

  \item \Verb+agg_loss+: this file contains the aggregated scenario
    loss statistics for the entire portfolio of \glspl{asset} defined
    in the \gls{exposuremodel}. The mean (\Verb+mean+) and standard
    deviation (\Verb+stddev+) of the total loss for the portfolio of
    \glspl{asset} are listed in this file.

  \item \Verb+loss_map+: this file contains mean (\Verb+mean+) and
    associated standard deviation (\Verb+stddev+) of the scenario loss for all
    \glspl{asset} at each of the unique locations in the \gls{exposuremodel}.

\end{enumerate}

If the calculation involves multiple \glspl{acr:gmpe}, separate output files
are generated for each of the above outputs, for each of the different
\glspl{acr:gmpe} used in the calculation.

These different output files for Scenario Risk calculations are described in
more detail in the following subsections.


\subsection{Scenario loss statistics}
\label{subsec:scenario_loss_statistics}

% \subsubsection{Asset loss statistics}
% \label{subsubsec:scenario_asset_loss_statistics}

% \subsubsection{Taxonomy loss statistics}
% \label{subsubsec:scenario_taxonomy_loss_statistics}

\subsubsection{Total loss statistics}
\label{subsubsec:scenario_total_loss_statistics}

This output is always produced for a Scenario Risk calculation and comprises a
mean total loss and associated standard deviation for the selected earthquake
rupture. These results are stored in a comma separate value (.csv) file as
illustrated in the example shown in Table~\ref{output:scenario_loss_total}.

\begin{table}[htbp]
\centering
\begin{tabular}{llrr}

\hline
\rowcolor{lightgray}
\bf{LossType} & \bf{Unit} & \bf{Mean} & \bf{Standard Deviation} \\
\hline
structural & USD & 8717775315.66 & 2047771108.36 \\
\hline

\end{tabular}
\caption{Example of a scenario total loss output file}
\label{output:scenario_loss_total}
\end{table}

The important attributes in a scenario total loss statistics output file are
described below:


\begin{itemize}

  \item \Verb+LossType+: the type of losses that are being stored. This
    parameter is taken from the \gls{vulnerabilitymodel} that was used in the
    loss calculations (e.g. fatalities, economic loss).

  \item \Verb+Unit+: this attribute defines the units in which the losses are
    being measured (e.g. USD or EUR). These units are the same as those defined
    in the \gls{exposuremodel} used for the calculation.

  \item \Verb+Mean+: the mean total loss across the portfolio of assets for the
    selected earthquake rupture.

  \item \Verb+Standard Deviation+: the standard deviation of the total loss 
    across the portfolio of assets for the selected earthquake rupture.

\end{itemize}


\subsection{Scenario loss maps}
\label{subsec:scenario_loss_map}

A scenario loss map contains the spatial distribution of the losses throughout
the region of interest. The scenario loss map comprises a mean loss and
respective standard deviation for each \gls{asset} for the selected earthquake
rupture, as shown in the example file in Listing~\ref{lst:output_scenario_loss_map}.

\begin{listing}[htbp]
  \inputminted[firstline=1,firstnumber=1,fontsize=\footnotesize,frame=single,bgcolor=lightgray]{xml}{oqum/risk/verbatim/output_scenario_loss_map.xml}
  \caption{Example scenario loss map}
  \label{lst:output_scenario_loss_map}
\end{listing}

The important attributes in a scenario loss map are described below:

\begin{itemize}

  \item \Verb+LossType+: the type of losses that are being stored. This
    parameter is taken from the \gls{vulnerabilitymodel} that was used in the
    loss calculations (e.g. fatalities, economic loss).

  \item \Verb+unit+: this attribute defines the units in which the losses are
    being measured (e.g. EUR).

  \item \Verb+node+: each loss map comprises various nodes, each node possibly
    containing a number of \glspl{asset}. The location of the node is defined
    by a latitude and longitude in decimal degrees within the field
    \Verb+gml:Point+. The mean loss (\Verb+mean+) and associated standard
    deviation (\Verb+stdDev+) for each \gls{asset} (identified by the parameter
    \Verb+assetRef+) is stored in the \Verb+loss+ field.

\end{itemize}