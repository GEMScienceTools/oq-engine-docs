For this calculator, the parameter \Verb+calculation_mode+ should be set to
\Verb+scenario_damage+.

\paragraph{Example 1}

This example illustrates a scenario damage calculation which uses a single
configuration file to first compute the ground motion fields for the given
rupture model and then calculate damage distribution statistics based on the
ground motion fields. A minimal job configuration file required for running a
scenario damage calculation is shown in
Listing~\ref{lst:config_scenario_damage_combined}.

\begin{listing}[htbp]
  \inputminted[firstline=1,firstnumber=1,fontsize=\footnotesize,frame=single,linenos,bgcolor=lightgray,label=job.ini]{ini}{oqum/risk/verbatim/config_scenario_damage_combined.ini}
  \caption{Example combined configuration file for running a scenario damage calculation (\href{https://raw.githubusercontent.com/GEMScienceTools/oq-engine-docs/master/oqum/risk/verbatim/config_scenario_damage_combined.xml}{Download example})}
  \label{lst:config_scenario_damage_combined}
\end{listing}

The general parameters \Verb+description+ and \Verb+calculation_mode+, and
\Verb+exposure_file+ have already been described earlier. The other parameters
seen in the above example configuration file are described below:

\begin{itemize}

  \item \Verb+rupture_model_file+: a parameter used to define the path
	to the earthquake \gls{rupturemodel} file describing the scenario event.

  \item \Verb+rupture_mesh_spacing+: a parameter used to specify the mesh size
  	(in km) used by the \glsdesc{acr:oqe} to discretize the rupture.
  	Note that the smaller the mesh spacing, the greater will be
  	(1) the precision in the calculation and
  	(2) the computational demand.

  \item \Verb+structural_fragility_file+: a parameter used to define the path
	to the structural \gls{fragilitymodel} file.

\end{itemize}

In this case, the ground motion fields will be computed at each of the
locations of the assets in the exposure model. Ground motion fields will be
generated for each of the intensity measure types found in the provided set of
fragility models. The above calculation can be run using the command line:

\begin{minted}[fontsize=\footnotesize,frame=single,bgcolor=lightgray]{shell-session}
user@ubuntu:~\$ oq engine --run job.ini
\end{minted}

After the calculation is completed, a message similar to the following will be
displayed:

\begin{minted}[fontsize=\footnotesize,frame=single,bgcolor=lightgray]{shell-session}
Calculation 2680 completed in 13 seconds. Results:
  id | output_type | name
5069 | datastore   | dmg_by_asset_and_collapse_map
5070 | datastore   | dmg_by_taxon
5071 | datastore   | dmg_total
\end{minted}

Note that one or more of the following parameters can be used in the same job
configuration file to provide the corresponding fragility model files:

\begin{itemize}

  \item \Verb+structural_fragility_file+: a parameter used to define the path
    to a structural \gls{fragilitymodel} file

  \item \Verb+nonstructural_fragility_file+: a parameter used to define the path
    to a nonstructural \gls{fragilitymodel} file

  \item \Verb+contents_fragility_file+: a parameter used to define the path
    to a contents \gls{fragilitymodel} file

  \item \Verb+business_interruption_fragility_file+: a parameter used to define
    the path to a business interruption \gls{fragilitymodel} file

\end{itemize}

It is important that the \Verb+lossCategory+ parameter in the metadata section
for each provided fragility model file (``structural'', ``nonstructural'',
``contents'', or ``business\_interruption'') should match the loss type
defined in the configuration file by the relevant keyword above.


\paragraph{Example 2}

This example illustrates a scenario damage calculation which uses separate
configuration files for the hazard and risk parts of a scenario damage
assessment. The first configuration file shown in
Listing~\ref{lst:config_scenario_damage_hazard} contains input models and
parameters required for the computation of the ground motion fields due to a
given rupture. The second configuration file shown in
Listing~\ref{lst:config_scenario_damage} contains input models and parameters
required for the calculation of the damage distribution for a portfolio of
assets due to the ground motion fields.

\begin{listing}[htbp]
  \inputminted[firstline=1,firstnumber=1,fontsize=\footnotesize,frame=single,linenos,bgcolor=lightgray,label=job\_hazard.ini]{ini}{oqum/risk/verbatim/config_scenario_hazard.ini}
  \caption{Example hazard configuration file for a scenario damage calculation (\href{https://raw.githubusercontent.com/GEMScienceTools/oq-engine-docs/master/oqum/risk/verbatim/config_scenario_hazard.xml}{Download example})}
  \label{lst:config_scenario_damage_hazard}
\end{listing}

\begin{listing}[htbp]
  \inputminted[firstline=1,firstnumber=1,fontsize=\footnotesize,frame=single,linenos,bgcolor=lightgray,label=job\_damage.ini]{ini}{oqum/risk/verbatim/config_scenario_damage.ini}
  \caption{Example risk configuration file for a scenario damage calculation (\href{https://raw.githubusercontent.com/GEMScienceTools/oq-engine-docs/master/oqum/risk/verbatim/config_scenario_damage.xml}{Download example})}
  \label{lst:config_scenario_damage}
\end{listing}


In this example, the set of intensity measure types for which the ground
motion fields should be generated is specified explicitly in the configuration
file using the parameter \Verb+intensity_measure_types+. If the hazard
calculation outputs are intended to be used as inputs for a subsequent
scenario damage or risk calculation, the set of intensity measure types
specified here must include all intensity measure types that are used in the
fragility or vulnerability models for the subsequent damage or risk
calculation.

In the hazard configuration file illustrated above
(Listing~\ref{lst:config_scenario_damage_hazard}), the list of sites at which
the ground motion values will be computed is provided in a CSV file, specified
using the \Verb+sites_csv+ parameter. The sites used for the hazard
calculation need not be the same as the locations of the assets in the
exposure model used for the following risk calculation. In such cases, it is
recommended to set a reasonable search radius (in km) using the
\Verb+asset_hazard_distance+ parameter for the \glsdesc{acr:oqe} to look for
available hazard values, as shown in the job\_damage.ini example file above.

The only new parameters introduced in risk configuration file for this example
(Listing~\ref{lst:config_scenario_damage}) are the \Verb+region_constraint+
and \Verb+asset_hazard_distance+ parameters, which are described below; all
other parameters have already been described in earlier examples.

\begin{itemize}

  \item \Verb+region_constraint+: this is an optional parameter, applicable
    only to risk calculations, which defines the polygon that will be used for
    filtering the assets from the exposure model. Assets outside of this region
    will not be considered in the risk calculations. This region is defined
    using pairs of coordinates that indicate the vertices of the polygon, which
    should be listed in the Well-known text (WKT) format:

    region\_constraint = lon\_1 lat\_1, lon\_2 lat\_2, ..., lon\_n lat\_n

    For each point, the longitude is listed first, followed by the latitude,
    both in decimal degrees. The list of points defining the polygon can be
    provided either in a clockwise or counter-clockwise direction.

    If the \Verb+region_constraint+ is not provided, all assets in the exposure
    model are considered for the risk calculation.

    This parameter is useful in cases where the exposure model covers a region
    larger than the one that is of interest in the current calculation.

  \item \Verb+asset_hazard_distance+: this parameter indicates the maximum
    allowable distance between an \gls{asset} and the closest hazard input.
    Hazard inputs can include hazard curves or ground motion intensity values.
    If no hazard input site is found within the radius defined by the
    \Verb+asset_hazard_distance+, the asset is skipped and a message is
    provided mentioning the id of the asset that is affected by this issue.

    If multiple hazard input sites are found within the radius defined by the
    this parameter, the hazard input site with the shortest distance from the
    asset location is associated with the asset. It is possible that the
    associated hazard input site might be located outside the polygon defined
    by the \Verb+region_constraint+.

\end{itemize}


Now, the above calculations described by the two configuration files
``job\_hazard.ini'' and ``job\_damage.ini'' can be run sequentially using the
command line as follows:

\begin{minted}[fontsize=\footnotesize,frame=single,bgcolor=lightgray]{shell-session}
user@ubuntu:~\$ oq engine --run job_hazard.ini,job_damage.ini
\end{minted}

The hazard and risk calculations can also be run separately. In that case, the
calculation id for the hazard calculation or the output id for the specific
ground motion fields output generated by the hazard calculation should be
provided to the \glsdesc{acr:oqe} while running the risk calculation using the
options \Verb+--hazard-calculation-id+ (or \Verb+--hc+) and 
\Verb+--hazard-output-id+ (or \Verb+--ho+) respectively. This is shown below:

\begin{minted}[fontsize=\footnotesize,frame=single,bgcolor=lightgray]{shell-session}
user@ubuntu:~\$ oq engine --run job_hazard.ini
\end{minted}

After the hazard calculation is completed, a message similar to the one below
will be displayed in the terminal:

\begin{minted}[fontsize=\footnotesize,frame=single,bgcolor=lightgray]{shell-session}
Calculation 2681 completed in 4 seconds. Results:
  id | output_type | name
5072 | datastore   | gmfs
\end{minted}

In the example above, the calculation~id of the hazard calculation is 2681.
There is only one output from this calculation, i.e., the \glspl{acr:gmf}. The
output~id for the \glspl{acr:gmf} generated by the above calculation is 5072.

The risk calculation for computing the damage distribution statistics for the
portfolio of \glspl{asset} can now be run using either:

\begin{minted}[fontsize=\footnotesize,frame=single,bgcolor=lightgray]{shell-session}
user@ubuntu:~\$ oq engine --run job_damage.ini --hc 2681
\end{minted}

or

\begin{minted}[fontsize=\footnotesize,frame=single,bgcolor=lightgray]{shell-session}
user@ubuntu:~\$ oq engine --run job_damage.ini --ho 5072
\end{minted}

After the calculation is completed, a message similar to the one listed above
in Example~1 will be displayed.

In order to retrieve the calculation~id of a previously run hazard calculation,
the option \Verb+--list-hazard-calculations+ (or \Verb+--lhc+) can be used to
display a list of all previously run hazard calculations:

\begin{minted}[fontsize=\footnotesize,frame=single,bgcolor=lightgray]{shell-session}
job_id |     status |         last_update |         description
  2609 | successful | 2015-12-01 14:14:14 | Mid Nepal earthquake
  ...
  2681 | successful | 2015-12-12 10:00:00 | Scenario hazard example
\end{minted}

The option \Verb+--list-outputs+ (or \Verb+--lo+) can be used to display a
list of all outputs generated during a particular calculation. For instance,

\begin{minted}[fontsize=\footnotesize,frame=single,bgcolor=lightgray]{shell-session}
user@ubuntu:~\$ oq engine --lo 2681
\end{minted}

will produce the following display:

\begin{minted}[fontsize=\footnotesize,frame=single,bgcolor=lightgray]{shell-session}
  id | output_type | name
5072 | datastore   | gmfs
\end{minted}


\paragraph{Example 3}

The example shown in Listing~\ref{lst:config_scenario_damage_gmf} illustrates
a scenario damage calculation which uses a file listing a precomputed set of
\glspl{acr:gmf}. These \glspl{acr:gmf} can be computed using the
\glsdesc{acr:oqe} or some other software. The \glspl{acr:gmf} must be provided
in the \gls{acr:nrml} schema as presented in
Section~\ref{subsec:output_event_based_psha}. The damage distribution is
computed based on the provided \glspl{acr:gmf}.

\begin{listing}[htbp]
  \inputminted[firstline=1,firstnumber=1,fontsize=\footnotesize,frame=single,linenos,bgcolor=lightgray,label=job.ini]{ini}{oqum/risk/verbatim/config_scenario_damage_gmf.ini}
  \caption{Example configuration file for a scenario damage calculation using a precomputed set of ground motion fields (\href{https://raw.githubusercontent.com/GEMScienceTools/oq-engine-docs/master/oqum/risk/verbatim/config_scenario_damage_gmf.xml}{Download example})}
  \label{lst:config_scenario_damage_gmf}
\end{listing}

\begin{itemize}

  \item \Verb+gmfs_file+: a parameter used to define the path
	  to the \glspl{acr:gmf} file in the \gls{acr:nrml} schema. This file must
    define \glspl{acr:gmf} for all of the intensity measure types used in the
    \gls{fragilitymodel}.

\end{itemize}

The above calculation can be run using the command line:

\begin{minted}[fontsize=\footnotesize,frame=single,bgcolor=lightgray]{shell-session}
user@ubuntu:~\$ oq engine --run job.ini
\end{minted}


\paragraph{Example 4}

This example illustrates a the hazard job configuration file for a scenario
damage calculation which uses two \glspl{acr:gmpe} instead of only one.
Currently, the set of \glspl{acr:gmpe} to be used for a scenario calculation
can be specified using a logic tree file, as demonstrated in
\ref{subsec:gmlt}. As of \glsdesc{acr:oqe18}, the weights in the logic tree
are ignored, and a set of \glspl{acr:gmf} will be generated for each
\gls{acr:gmpe} in the logic tree file. Correspondingly, damage distribution
statistics will be generated for each set of \gls{acr:gmf}.

The file shown in Listing~\ref{lst:input_scenario_gmlt} lists the two
\glspl{acr:gmpe} to be used for the hazard calculation:

\begin{listing}[htbp]
  \inputminted[firstline=1,firstnumber=1,fontsize=\footnotesize,frame=single,linenos,bgcolor=lightgray,label=gsim\_logic\_tree.xml]{xml}{oqum/risk/verbatim/input_scenario_gmlt.xml}
  \caption{Example ground motion logic tree for a scenario calculation (\href{https://raw.githubusercontent.com/GEMScienceTools/oq-engine-docs/master/oqum/risk/verbatim/input_scenario_gmlt.xml}{Download example})}
  \label{lst:input_scenario_gmlt}
\end{listing}

The only change that needs to be made in the hazard job configuration file is
to replace the \Verb+gsim+ parameter with \Verb+gsim_logic_tree_file+, as
demonstrated in Listing~\ref{lst:config_scenario_hazard_gmlt}.

\begin{listing}[htbp]
  \inputminted[firstline=1,firstnumber=1,fontsize=\footnotesize,frame=single,linenos,bgcolor=lightgray,label=job\_hazard.ini]{ini}{oqum/risk/verbatim/config_scenario_hazard_gmlt.ini}
  \caption{Example configuration file for a scenario damage calculation using a logic-tree file (\href{https://raw.githubusercontent.com/GEMScienceTools/oq-engine-docs/master/oqum/risk/verbatim/config_scenario_hazard_gmlt.xml}{Download example})}
  \label{lst:config_scenario_hazard_gmlt}
\end{listing}


\paragraph{Example 5}

This example illustrates a scenario damage calculation which specifies
fragility models for calculating damage to structural and nonstructural
components of structures, and also specifies \gls{consequencemodel} files for
calculation of the corresponding losses.

A minimal job configuration file required for running a scenario damage
calculation followed by a consequences analysis is shown in
Listing~\ref{lst:config_scenario_damage_consequences}.

\begin{listing}[htbp]
  \inputminted[firstline=1,firstnumber=1,fontsize=\footnotesize,frame=single,linenos,bgcolor=lightgray,label=job.ini]{ini}{oqum/risk/verbatim/config_scenario_damage_consequences.ini}
  \caption{Example configuration file for a scenario damage calculation followed by a consequences analysis (\href{https://raw.githubusercontent.com/GEMScienceTools/oq-engine-docs/master/oqum/risk/verbatim/config_scenario_damage_consequences.xml}{Download example})}
  \label{lst:config_scenario_damage_consequences}
\end{listing}

Note that one or more of the following parameters can be used in the same job
configuration file to provide the corresponding \gls{consequencemodel} files:

\begin{itemize}

  \item \Verb+structural_consequence_file+: a parameter used to define the path
    to a structural \gls{consequencemodel} file

  \item \Verb+nonstructural_consequence_file+: a parameter used to define the path
    to a nonstructural \gls{consequencemodel} file

  \item \Verb+contents_consequence_file+: a parameter used to define the path
    to a contents \gls{consequencemodel} file

  \item \Verb+business_interruption_consequence_file+: a parameter used to define
    the path to a business interruption \gls{consequencemodel} file

\end{itemize}

It is important that the \Verb+lossCategory+ parameter in the metadata section
for each provided \gls{consequencemodel} file (``structural'', ``nonstructural'',
``contents'', or ``business\_interruption'') should match the loss type
defined in the configuration file by the relevant keyword above.

The above calculation can be run using the command line:

\begin{minted}[fontsize=\footnotesize,frame=single,bgcolor=lightgray]{shell-session}
user@ubuntu:~\$ oq engine --run job.ini
\end{minted}

After the calculation is completed, a message similar to the following will be
displayed:

\begin{minted}[fontsize=\footnotesize,frame=single,bgcolor=lightgray]{shell-session}
Calculation 1579 completed in 37 seconds. Results:
  id | output_type | name
8990 | datastore | csq_by_asset
8991 | datastore | csq_by_taxon
8992 | datastore | csq_total
8993 | datastore | dmg_by_asset_and_collapse_map
8994 | datastore | dmg_by_taxon
8995 | datastore | dmg_total
\end{minted}
