For this calculator, the parameter \Verb+calculation_mode+ should be set to
\Verb+scenario_damage+.

\paragraph{Example 1}

This example illustrates a scenario damage calculation which uses a single
configuration file to first compute the ground motion fields for the given
rupture model and then calculate damage distribution statistics based on the
ground motion fields. A minimal job configuration file required for
running a scenario damage calculation is shown below:

\inputminted[firstline=1,firstnumber=1,fontsize=\footnotesize,frame=single,linenos,bgcolor=lightgray]{ini}{oqum/risk/Verbatim/config_scenario_damage_combined.ini}\\

The general parameters \Verb+description+ and \Verb+calculation_mode+, and
\Verb+exposure_file+ have already been described earlier. The other parameters
seen in the above example configuration file are described below:

\begin{itemize}

  \item \Verb+rupture_model_file+: a parameter used to define the path
	to the earthquake \gls{rupture model} file describing the scenario event.

  \item \Verb+rupture_mesh_spacing+: a parameter used to specify the mesh size
  	(in km) used by OpenQuake-engine to discretize the rupture.
  	Note that smaller the mesh spacing, greater are
  	(1) the precision in the calculation and
  	(2) the computational demand.

  \item \Verb+structural_fragility_file+: a parameter used to define the path
	to the structural \gls{fragility model} file.

\end{itemize}