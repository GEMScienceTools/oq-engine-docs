This demo uses the same probabilistic seismic hazard assessment (PSHA) model
described in the previous example. However, instead of hazard curves, sets of
ground motion fields are required. To trigger the hazard calculations the
following command needs to be used:

\begin{Verbatim}[frame=single, commandchars=\\\{\}, samepage=true]
user@ubuntu:~\$ oq-engine --run job_hazard.ini
\end{Verbatim}

and the following results are expected:

\begin{Verbatim}[frame=single, commandchars=\\\{\}, samepage=true]
Calculation 15 results:
  id | output_type | name
 414 | Ground Motion Field | GMF rlz-101
 412 | Stochastic Event Set | SES Collection 0
 413 | Stochastic Event Set | SES Collection 1
\end{Verbatim}

Again, since there is only one branch in the logic tree, only one set of
ground motion fields will be used in the risk calculations, which can be
launched through the following command:

\begin{Verbatim}[frame=single, commandchars=\\\{\}, samepage=true]
user@ubuntu:~\$ oq-engine --run job_risk.ini --hc 15
\end{Verbatim}

leading to the following outputs:

\begin{Verbatim}[frame=single, commandchars=\\\{\}, samepage=true]
Calculation 16 results:
  id | output_type | name
 415 | Event Loss Table | Event Loss Table type=structural, hazard=414
 416 | Event Loss Asset | Event Loss Asset type=structural, hazard=414
 417 | Loss Curve | loss curves. type=structural, hazard=414
 418 | Aggregate Loss Curve | aggregate loss curves. loss_type=structural 
 hazard=414||gmf||GMF rlz-101
\end{Verbatim}