The Scenario Damage Calculator produces the following set of output files for
all loss types (amongst ``structural'', ``nonstructural'', ``contents'', or
``business\_interruption'') for which a fragility model file was provided in
the configuration file:

\begin{enumerate}

  \item \Verb+dmg_dist_per_asset+: this file contains the damage distribution
    statistics for each of the individual \glspl{asset} defined in the
    \gls{exposure model} that fall within the \Verb+region_constraint+ and have
    a computed \gls{acr:gmf} value available within the defined
    \Verb+asset_hazard_distance+. For each \gls{asset}, the mean number of
    buildings (\Verb+mean+) and associated standard deviation (\Verb+stddev+)
    of the number of buildings in each damage state are listed in this file.

  \item \Verb+dmg_dist_per_taxonomy+: this file contains the aggregated damage
    distribution statistics for each of the \glspl{taxonomy} defined in the
    \gls{exposure model}. For each \gls{taxonomy}, the mean number of
    buildings (\Verb+mean+) and associated standard deviation (\Verb+stddev+)
    of the number of buildings in each damage state are listed in this file.

  \item \Verb+dmg_dist_total+: this file contains the aggregated damage
    distribution statistics for the entire portfolio of \glspl{asset} defined
    in the \gls{exposure model}. The mean (\Verb+mean+) and associated standard
    deviation (\Verb+stddev+) of the total number of buildings in each
    damage state are listed in this file.

  \item \Verb+collapse_map+: this file contains mean (\Verb+mean+) and
    associated standard deviation (\Verb+stddev+) of the number of buildings
    in the ultimate limit state for all \glspl{asset} at each of the unique
    locations in the \gls{exposure model}.

\end{enumerate}

In addition to the above output files which are produced for all Scenario
Damage calculations, the following set of output files for all loss types
(amongst ``structural'', ``nonstructural'', ``contents'', or
``business\_interruption'') for which a consequence model file was also
provided in the configuration file:

\begin{enumerate}
\setcounter{enumi}{4}

  \item \Verb+csq_by_asset+: this file contains the scenario consequence
    statistics for each of the individual \glspl{asset} defined in the
    \gls{exposure model} that fall within the \Verb+region_constraint+ and have
    a computed \gls{acr:gmf} value available within the defined
    \Verb+asset_hazard_distance+. For each \gls{asset}, the mean consequences
    (\Verb+mean+) and associated standard deviation (\Verb+stddev+) are listed
    in this file.

  \item \Verb+csq_by_taxon+: this file contains the aggregated scenario
    consequence statistics for each of the \glspl{taxonomy} defined in the
    \gls{exposure model}. For each \gls{taxonomy}, the mean consequences
    (\Verb+mean+) and associated standard deviation (\Verb+stddev+) are listed
    in this file.

  \item \Verb+csq_total+: this file contains the aggregated scenario
    consequence statistics for the entire portfolio of \glspl{asset} defined
    in the \gls{exposure model}. The mean consequences (\Verb+mean+) and 
    associated standard deviation (\Verb+stddev+) are listed in this file.

\end{enumerate}

These different output files are described in more detail in the following
subsections.


\subsection{Scenario damage statistics}
\label{subsec:scenario_damage_statistics}

\subsubsection{Asset damage statistics}
\label{subsubsec:scenario_asset_damage_statistics}

This output contains the damage distribution statistics for each of the
individual \glspl{asset} defined in the \gls{exposure model} that fall within
the \Verb+region_constraint+ and have a computed \gls{acr:gmf} value available
within the defined \Verb+asset_hazard_distance+. An example output file for
structural damage is shown in the file snippet below:

\inputminted[firstline=1,firstnumber=1,fontsize=\footnotesize,frame=single,linenos,bgcolor=lightgray]{xml}{oqum/risk/verbatim/output_scenario_damage_asset.xml}\\

The key fields in the above output file are the following:

\begin{itemize}

  \item \Verb+damageStates+: this field serves the purposes of storing the set
    of damage states, as defined in the \gls{fragility model} employed in the
    calculations

  \item \Verb+DDNode+: this attribute is used to store the damage distribution
    of a number of \glspl{asset}, at a given location (defined within the
    attribute \Verb+gml:Point+). For each \gls{asset}, the mean number of
    buildings (\Verb+mean+) and associated standard deviation (\Verb+stddev+)
    of the number of buildings in each damage state are listed.

\end{itemize}


\subsubsection{Taxonomy damage statistics}
\label{subsubsec:scenario_taxonomy_damage_statistics}

The Scenario Damage calculator can also estimate the expected total number of
buildings of a certain \gls{taxonomy} in each damage state. This distribution
of damage per building \gls{taxonomy} is depicted in the following example
output file snippet:

\inputminted[firstline=1,firstnumber=1,fontsize=\footnotesize,frame=single,linenos,bgcolor=lightgray]{xml}{oqum/risk/verbatim/output_scenario_damage_taxonomy.xml}\\

In the damage distribution per \gls{taxonomy}, each \Verb+DDNode+ contains the
statistics of the number of buildings in each damage state, belonging to a
given building class as specified by the \Verb+taxonomy+ attribute.


\subsubsection{Total damage statistics}
\label{subsubsec:scenario_total_damage_statistics}

Finally, a total damage distribution can also be calculated, which contains
the mean and standard deviation of the total number of buildings in each
damage state, as illustrated below.

\begin{Verbatim}[frame=single, commandchars=\\\{\}, samepage=false]
\textcolor{gray}{<?xml version="1.0" encoding="UTF-8"?>}
<nrml xmlns:gml="http://www.opengis.net/gml"
      xmlns="http://openquake.org/xmlns/nrml/0.5">
<\textcolor{red}{totalDmgDist}>
     <\textcolor{green}{damageStates}>
         no_damage
         slight
         moderate
         complete
     <\textcolor{green}{/damageStates}>
     <\textcolor{green}{damage} ds="no_damage" mean="456450.2" stddev="26376.62"/>
     <\textcolor{green}{damage} ds="slight" mean="88102.3" stddev="3283.9"/>
     <\textcolor{green}{damage} ds="moderate" mean="103564.6" stddev="3487.1"/>
     <\textcolor{green}{damage} ds="complete" mean="275891.1" stddev="26676.8"/>
<\textcolor{red}{/totalDmgDist}>
</nrml>
\end{Verbatim}


\subsection{Scenario collapse maps}

Collapse maps are part of the Scenario Damage calculator outputs. These
results provide the spatial distribution of the number of the collapsed
buildings throughout the area of interest. An example of the schema is
presented below.

\begin{Verbatim}[frame=single, commandchars=\\\{\}, samepage=false]
\textcolor{gray}{<?xml version="1.0" encoding="UTF-8"?>}
<nrml xmlns:gml="http://www.opengis.net/gml"
      xmlns="http://openquake.org/xmlns/nrml/0.5">
<\textcolor{red}{collapseMap}>
     <\textcolor{green}{CMNode}>
          <gml:Point>
            <gml:pos>83.31 29.46</gml:pos>
          </gml:Point>
          <\textcolor{blue}{cf} assetRef="a1" mean="227.1" stdDev="95.8"/>
          <\textcolor{blue}{cf} assetRef="a2" mean="703.2" stdDev="240.2"/>
          <\textcolor{blue}{cf} assetRef="a3" mean="199.5" stdDev="63.3"/>
          <\textcolor{blue}{cf} assetRef="a4" mean="357.8" stdDev="136.1"/>
     <\textcolor{green}{/CMNode}>
    ...
     <\textcolor{green}{CMNode}>
          <gml:Point>
            <gml:pos>83.33 28.71</gml:pos>
          </gml:Point>
          <\textcolor{blue}{cf} assetRef="a997" mean="239.4" stdDev="102.0"/>
          <\textcolor{blue}{cf} assetRef="a998" mean="733.0" stdDev="253.2"/>
          <\textcolor{blue}{cf} assetRef="a999" mean="207.4" stdDev="66.5"/>
     <\textcolor{green}{/CMNode}>
<\textcolor{red}{/collapseMap}>
</nrml>
\end{Verbatim}

This schema follows the same structure of the loss maps presented previously.
Thus, the results for a number of \glspl{asset} at a given location are stored
within the field \Verb+CMNode+. This field is associated with a location
(defined within the \Verb+gml:Point+ attribute) and it contains the mean
number of collapses (\Verb+mean+) and respective standard deviation
(\Verb+stdDev+) for each \gls{asset} (identified by the parameter
\Verb+assetRef+).