\subsection{Scenario damage statistics}

The damage distribution is part of the outputs from the Scenario Damage
calculator, and can be provided in three ways: per \gls{asset}, per taxonomy
or the total damage distribution. In the following illustration, an example of
the NRML schema for the damage distribution per \gls{asset} is presented:

\begin{Verbatim}[frame=single, commandchars=\\\{\}, samepage=false]
\textcolor{gray}{<?xml version="1.0" encoding="UTF-8"?>}
<nrml xmlns:gml="http://www.opengis.net/gml"
      xmlns="http://openquake.org/xmlns/nrml/0.5">
<\textcolor{red}{dmgDistPerAsset}>
     <\textcolor{green}{damageStates}>
         no_damage
         slight
         moderate
         complete
     <\textcolor{green}{/damageStates}>
     <\textcolor{green}{DDNode}>
          <gml:Point>
            <gml:pos>83.31 29.46</gml:pos>
          </gml:Point>
          <\textcolor{blue}{asset} assetRef="a1">
            <damage ds="no_damage" mean="486.6" stddev="130.1"/>
            <damage ds="slight" mean="118.8" stddev="9.9"/>
            <damage ds="moderate" mean="130.3" stddev="20.3"/>
            <damage ds="complete" mean="186.5" stddev="90.8"/>
          <\textcolor{blue}{/asset}>
          <\textcolor{blue}{asset} assetRef="2">
            <damage ds="no_damage" mean="877.08" stddev="257.9"/>
            <damage ds="slight" mean="171.3" stddev="13.2"/>
            <damage ds="moderate" mean="161.5" stddev="014.5"/>
            <damage ds="complete" mean="563.8" stddev="223.6"/>
          <\textcolor{blue}{/asset}>
     <\textcolor{green}{/DDNode}>
     ...
     <\textcolor{green}{DDNode}>
          <gml:Point>
            <gml:pos>83.91 28.19</gml:pos>
          </gml:Point>
          <\textcolor{blue}{asset} assetRef="999">
            <damage ds="no_damage" mean="21.5" stddev="16.6"/>
            <damage ds="slight" mean="15.5" stddev="8.7"/>
            <damage ds="moderate" mean="39.1" stddev="17.3"/>
            <damage ds="complete" mean="493.5" stddev="53.1"/>
          <\textcolor{blue}{/asset}>
     <\textcolor{green}{/DDNode}>
<\textcolor{red}{/dmgDistPerAsset}>
</nrml>
\end{Verbatim}

\begin{itemize}
\item  \Verb+damageStates+: this field serves the purposes of storing the set of damage states, as defined in the \gls{fragility model} employed in the calculations;
\item  \Verb+DDNode+: this attribute is used to store the damage distribution of a number of \glspl{asset}, at a given location (defined within the attribute \Verb+gml:Point+). For each \gls{asset}, the mean number of buildings (\Verb+mean+) and associated standard deviation (\Verb+stddev+) in each damage state is defined.
\end{itemize}

The Scenario Damage calculator can also estimate the total number of buildings
with a certain \gls{taxonomy}, in each damage state. This  distribution of
damage per building \gls{taxonomy} is depicted in the following example.

\begin{Verbatim}[frame=single, commandchars=\\\{\}, samepage=false]
\textcolor{gray}{<?xml version="1.0" encoding="UTF-8"?>}
<nrml xmlns:gml="http://www.opengis.net/gml"
      xmlns="http://openquake.org/xmlns/nrml/0.5">
<\textcolor{red}{dmgDistPerAsset}>
     <\textcolor{green}{damageStates}>
         no_damage
         slight
         moderate
         complete
     <\textcolor{green}{/damageStates}>
     <\textcolor{green}{DDNode}>
      <taxonomy>W</taxonomy>
      <damage ds="no_damage" mean="456450.2" stddev="26376.62"/>
      <damage ds="slight" mean="88102.3" stddev="3283.9"/>
      <damage ds="moderate" mean="103564.6" stddev="3487.1"/>
      <damage ds="complete" mean="275891.1" stddev="26676.8"/>
     <\textcolor{green}{/DDNode}>
     ...
     <\textcolor{green}{DDNode}>
      <taxonomy>RC</taxonomy>
      <damage ds="no_damage" mean="4484.2" stddev="460.9"/>
      <damage ds="slight" mean="932.4" stddev="106.7"/>
      <damage ds="moderate" mean="1691.7" stddev="177.9"/>
      <damage ds="complete" mean="7659.5" stddev="799.3"/>
     <\textcolor{green}{/DDNode}>
<\textcolor{red}{/dmgDistPerAsset}>
</nrml>
\end{Verbatim}

In the damage distribution per \gls{taxonomy}, each \Verb+DDNode+ contains the
statistics of the number of buildings in each damage state, belonging to a
given building class as specified in the \Verb+taxonomy+ attribute. Finally, a
total damage distribution can also be calculated, which contains the mean and
standard deviation of the total number of buildings in each damage state, as
illustrated below.

\begin{Verbatim}[frame=single, commandchars=\\\{\}, samepage=false]
\textcolor{gray}{<?xml version="1.0" encoding="UTF-8"?>}
<nrml xmlns:gml="http://www.opengis.net/gml"
      xmlns="http://openquake.org/xmlns/nrml/0.5">
<\textcolor{red}{totalDmgDist}>
     <\textcolor{green}{damageStates}>
         no_damage
         slight
         moderate
         complete
     <\textcolor{green}{/damageStates}>
     <\textcolor{green}{damage} ds="no_damage" mean="456450.2" stddev="26376.62"/>
     <\textcolor{green}{damage} ds="slight" mean="88102.3" stddev="3283.9"/>
     <\textcolor{green}{damage} ds="moderate" mean="103564.6" stddev="3487.1"/>
     <\textcolor{green}{damage} ds="complete" mean="275891.1" stddev="26676.8"/>
<\textcolor{red}{/totalDmgDist}>
</nrml>
\end{Verbatim}


\subsection{Scenario collapse maps}

Collapse maps are part of the Scenario Damage calculator outputs. These
results provide the spatial distribution of the number of the collapsed
buildings throughout the area of interest. An example of the schema is
presented below.

\begin{Verbatim}[frame=single, commandchars=\\\{\}, samepage=false]
\textcolor{gray}{<?xml version="1.0" encoding="UTF-8"?>}
<nrml xmlns:gml="http://www.opengis.net/gml"
      xmlns="http://openquake.org/xmlns/nrml/0.5">
<\textcolor{red}{collapseMap}>
     <\textcolor{green}{CMNode}>
          <gml:Point>
            <gml:pos>83.31 29.46</gml:pos>
          </gml:Point>
          <\textcolor{blue}{cf} assetRef="a1" mean="227.1" stdDev="95.8"/>
          <\textcolor{blue}{cf} assetRef="a2" mean="703.2" stdDev="240.2"/>
          <\textcolor{blue}{cf} assetRef="a3" mean="199.5" stdDev="63.3"/>
          <\textcolor{blue}{cf} assetRef="a4" mean="357.8" stdDev="136.1"/>
     <\textcolor{green}{/CMNode}>
    ...
     <\textcolor{green}{CMNode}>
          <gml:Point>
            <gml:pos>83.33 28.71</gml:pos>
          </gml:Point>
          <\textcolor{blue}{cf} assetRef="a997" mean="239.4" stdDev="102.0"/>
          <\textcolor{blue}{cf} assetRef="a998" mean="733.0" stdDev="253.2"/>
          <\textcolor{blue}{cf} assetRef="a999" mean="207.4" stdDev="66.5"/>
     <\textcolor{green}{/CMNode}>
<\textcolor{red}{/collapseMap}>
</nrml>
\end{Verbatim}

This schema follows the same structure of the loss maps presented previously.
Thus, the results for a number of \glspl{asset} at a given location are stored
within the field \Verb+CMNode+. This field is associated with a location
(defined within the \Verb+gml:Point+ attribute) and it contains the mean
number of collapses (\Verb+mean+) and respective standard deviation
(\Verb+stdDev+) for each \gls{asset} (identified by the parameter
\Verb+assetRef+).