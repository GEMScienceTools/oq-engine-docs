The parameter \Verb+calculation_mode+ needs to be set to \Verb+event_based_risk+ in order to use this calculator. Similarly to that described for the Scenario Risk Calculator, a Monte Carlo sampling process is also employed within this module to take into account the loss ratio uncertainty. Hence, the parameters \Verb+asset_correlation+ and \Verb+master_seed+ need to be defined as previously described. This calculator is also capable of estimating insured losses and therefore, the \Verb+insured_losses+ attribute should be specified as well. The parameter ``risk\_investigation\_time'' specifies the time period for which the event loss tables and loss exceedance curves will be calculated. If this parameter is not provided in the risk job configuration file, the time period used is the same as that specifed in the hazard calculation. The Probabilistic Event-based Risk Calculator can disaggregate the losses based on magnitude/distance and location (longitude/latitude) of the events. In order to do so, it is necessary to define the bin width of each of these parameters, as illustrated in the following example.

The remaining parameters are presented below.

\begin{Verbatim}[frame=single, commandchars=\\\{\}, samepage=true]
...
structural_vulnerability_file = struct_vul_model.xml
nonstructural_vulnerability_file = nonstruct_vul_model.xml
contents_vulnerability_file = cont_vul_model.xml
business_interruption_vulnerability_file = bus_int_vul_model.xml
occupants_vulnerability_file = occ_vul_model.xml

asset_correlation = 0.7
master_seed = 3
insured_losses = true

sites_disagg = 85.07917, 27.4625
mag_bin_width = 0.5
distance_bin_width = 20
coordinate_bin_width = 0.5

specific_assets = asset_11 asset_132 asset_303 asset_611

loss_curve_resolution = 20
conditional_loss_poes = 0.01, 0.05, 0.10
\end{Verbatim}

\begin{itemize}
\item \Verb+loss_curve_resolution+: since this calculator uses an event\--based ap\-proach, a large number of levels of loss (and associated probabilities of exceedance) is computed (one per event) for each asset. The oq-risklib will use this large set of results to extrapolate a loss curve, whose number of points are controlled by this parameter. By default, the OpenQuake-engine assumes the \Verb+loss_curve_resolution+ equal to 20;
\item  \Verb+risk_investigation_time+: this parameter specifies the time period for which the event loss tables and loss exceedance curves will be calculated. If this parameter is not provided in the risk job configuration file, the time period used is the same as that specifed in the hazard calculation;
\item  \Verb+conditional_loss_poes+: this parameter is used to define the probabilities of exceedance at which loss maps are to be produced;
\item  \Verb+sites_disagg+: list of locations (pairs of longitude and latitude) where the loss disaggregation should be carried out. Notice that in order to perform the loss disaggregation, assets needs to exist at those locations;
\item  \Verb+mag_bin_width+: this parameter specifies the with of the magnitude bins (in Mw);
\item  \Verb+distance_bin_width+: this parameter specifies the with of the distance bins (in km);
\item  \Verb+coordinate_bin_width+: this parameter specifies the with of the coordinates bins (in decimal degrees);
\item  \Verb+specific_assets+: this parameter specifies the set of assets for which the asset event loss table will be generated;
\end{itemize}

The definition of the parameters for the loss disaggregation follow the same rules established for the seismic hazard disaggregation described in section~\ref{subsec:config_hazard_disaggregation}.

\paragraph{Note regarding the new event-based risk calculators}

Starting from this release some of the scientific calculators are saving their inputs and outputs in a single HDF5 file, called the datastore. The HDF5 format is a well known standard in the scientific community, can be read/written by a variety of programming languages and with different tools and it is a state-of-the-art technology when it comes to managing large numeric datasets. The change to the HDF5 technology provides huge performance benefits compared to the earlier approach used by the engine, which involved storing arrays in PostgreSQL.

In OpenQuake 1.5 the event based calculators based on Postgres (both hazard and risk) are officially deprecated. They are still present and work as before, but they are being replaced with new versions of the calculators based on the HDF5 technology. The actual removal of the old calculators is scheduled for OpenQuake 1.6. The change will have no impact on regular users, who will simply notice a definite improvement in erformance. Nonetheless, the change will affect power users who are performing queries on the OpenQuake database, since there will be nothing left in the database once we remove the old calculators.

In order to make the transition easier, OpenQuake 1.5 already includes the new versions of the event based calculators based on HDF5, so it is possible to use them right now. The new calculators can be run in OpenQuake 1.5 with the command:

\begin{Verbatim}[frame=single, commandchars=\\\{\}, fontsize=\small]
user@ubuntu:~$ oq-engine --lite --run job_haz.ini,job_risk.ini
\end{Verbatim}

If you do not pass the --lite flag the old calculators will be run by default. In future releases of the engine, the remaining calculators based on Postgres will be progressively replaced by the new calculators based on HDF5. At the end of this process, which will be spread over several upcoming releases, the --lite flag will be removed. All of the old calculators relying on the database will be replaced internally by the newer ``lite'' versions based on HDF5 and the old calculators will not be available anymore. The OpenQuake database will only contain accessory information (essentially a table with the users and references to the outputs of each user) but nothing relevant for the scientific computation.