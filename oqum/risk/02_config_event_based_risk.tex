The parameter \Verb+calculation_mode+ needs to be set to
\Verb+event_based_risk+ in order to use this calculator. Similarly to that
described for the Scenario Risk Calculator, a Monte Carlo sampling process is
also employed within this module to take into account the loss ratio
uncertainty. Hence, the parameters \Verb+asset_correlation+ and
\Verb+master_seed+ need to be defined as previously described. This calculator
is also capable of estimating insured losses and therefore, the
\Verb+insured_losses+ attribute can be specified as well. The parameter
``risk\_investigation\_time'' specifies the time period for which the event
loss tables and loss exceedance curves will be calculated. If this parameter
is not provided in the risk job configuration file, the time period used is
the same as that specifed in the hazard calculation.

The remaining parameters are presented below.

\begin{Verbatim}[frame=single, commandchars=\\\{\}, samepage=true]
...
structural_vulnerability_file = struct_vul_model.xml
nonstructural_vulnerability_file = nonstruct_vul_model.xml
contents_vulnerability_file = cont_vul_model.xml
business_interruption_vulnerability_file = bus_int_vul_model.xml
occupants_vulnerability_file = occ_vul_model.xml

asset_correlation = 0.7
master_seed = 3
insured_losses = true

loss_curve_resolution = 20
conditional_loss_poes = 0.01, 0.05, 0.10
\end{Verbatim}

\begin{itemize}

  \item \Verb+loss_curve_resolution+: since this calculator uses an event-based
    approach, a large number of levels of loss (and associated probabilities of
    exceedance) is computed (one per event) for each asset. The oq-risklib will
    use this large set of results to extrapolate a loss curve, whose number of
    points are controlled by this parameter. By default, the OpenQuake-engine
    assumes the \Verb+loss_curve_resolution+ equal to 20.

  \item \Verb+risk_investigation_time+: this parameter specifies the time
    period for which the event loss tables and loss exceedance curves will be
    calculated. If this parameter is not provided in the risk job configuration
    file, the time period used is the same as that specifed in the hazard
    calculation.

  \item \Verb+conditional_loss_poes+: this parameter is used to define the
    probabilities of exceedance at which loss maps are to be produced.

\end{itemize}