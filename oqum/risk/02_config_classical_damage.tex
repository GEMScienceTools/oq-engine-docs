In order to run this calculator, the parameter \Verb+calculation_mode+ needs
to be set to \Verb+classical_damage+.

\paragraph{Example 1}

This example illustrates a classical probabilistic damage calculation which
uses a single configuration file to first compute the hazard curves for the
given source model and ground motion model and then calculate damage
distribution statistics based on the hazard curves. A minimal job
configuration file required for running a classical probabilistic damage
calculation is shown below:

\inputminted[firstline=1,firstnumber=1,fontsize=\footnotesize,frame=single,linenos,bgcolor=lightgray,label=job.ini]{ini}{oqum/risk/verbatim/config_classical_damage_combined.ini}\\

The general parameters \Verb+description+ and \Verb+calculation_mode+, and
\Verb+exposure_file+ have already been described earlier in
Section~\ref{sec:config_scenario_damage}. The parameters related to the
hazard curves computation have been described earlier in
Section~\ref{subsec:config_classical_psha}.

In this case, the hazard curves will be computed at each of the locations of
the assets in the exposure model, for each of the intensity measure types
found in the provided set of fragility models. The above calculation can be
run using the command line:

\begin{Verbatim}[frame=single, commandchars=\\\{\}, samepage=true]
user@ubuntu:~\$ oq-engine --run job.ini
\end{Verbatim}

After the calculation is completed, a message similar to the following will be
displayed:

\begin{Verbatim}[frame=single, commandchars=\\\{\}, samepage=true]
Calculation 2741 completed in 12 seconds. Results:
  id | output_type | name
5359 | Damage Per Asset | Damage distribution for hazard=5357
\end{Verbatim}


\paragraph{Example 2}

This example illustrates a classical probabilistic damage calculation which
uses separate configuration files for the hazard and risk parts of a classical
probabilistic damage assessment. The first configuration file contains input
models and parameters required for the computation of the hazard curves. The
second configuration file contains input models and parameters required for
the calculation of the probabilistic damage distribution for a portfolio of
assets based on the hazard curves and fragility models.

\inputminted[firstline=1,firstnumber=1,fontsize=\footnotesize,frame=single,linenos,bgcolor=lightgray,label=job\_hazard.ini]{ini}{oqum/risk/verbatim/config_classical_hazard.ini}\\

\inputminted[firstline=1,firstnumber=1,fontsize=\footnotesize,frame=single,linenos,bgcolor=lightgray,label=job\_damage.ini]{ini}{oqum/risk/verbatim/config_classical_damage.ini}\\

Now, the above calculations described by the two configuration files
``job\_hazard.ini'' and ``job\_damage.ini'' can be run sequentially or
separately, as illustrated in Example~2 in
Section~\ref{sec:config_scenario_damage}. The new parameters introduced in the
above example configuration file are described below:

\begin{itemize}

  \item \Verb+risk_investigation_time+: an optional parameter that can be used
    in probabilistic damage or risk calculations where the period of interest
    for the risk calculation is different from the period of interest for the 
    hazard calculation. If this parameter is not explicitly set, the 
    OpenQuake-engine will assume that the risk calculation is over the same 
    time period as the preceding hazard calculation.

  \item \Verb+steps_per_interval+: an optional parameter that can be used to
    specify whether discrete fragility functions in the fragility models should
    be discretized further, and if so, how many intermediate steps to use for
    the discretization. Setting 

    steps\_per\_interval = n

    will result in the OpenQuake-engine discretizing the discrete fragility
    models using (n - 1) interpolation steps between each pair of 
    {intensity level, poe} points.

    The default value of this parameter is one, implying no interpolation.

\end{itemize}
