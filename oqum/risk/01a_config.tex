The configuration file (or job.ini file) represents the location where the
paths to the input files, the parameters controlling the risk calculations and
the type of outputs are defined. Some initial parameters common to all the
risk calculators are presented below. The remaining parameters that are
specific to each risk calculator are discussed in subsequent sections. For
additional information about how each parameter is being used within the
methodologies implemented in the oq-engine, users advised to consult the
OpenQuake-engine Book (Risk).

\begin{Verbatim}[frame=single, commandchars=\\\{\}, samepage=true]
[general]
description = Scenario Risk Nepal
calculation_mode = scenario_risk

exposure_file = exposure_model.xml
region_constraint = 78.0 31.5,89.5 31.5,89.5 25.5,78 25.5
asset_hazard_distance = 10
...
\end{Verbatim}

\begin{itemize}
\item  \Verb+description+: a parameter that can be used to include some information about the type of calculations that are going to be performed;
\item  \Verb+calculation_mode+: this parameter sets the type of calculations. The key word for each risk calculator is described in the following sections;
\item  \Verb+exposure_file+: this parameter is used to specify the path to the \gls{exposure model} file;
\item  \Verb+region_constraint+: this field is used to define the polygon enclosing the region of interest. Assets outside of this region will not be considered in the risk calculations. This region is defined using pairs of coordinates (longitude and latitude in decimal degrees) that indicate the vertices of the polygon;
\item  \Verb+asset_hazard_distance+: this parameter indicates the maximum allowable distance between an \gls{asset} and the closest hazard input. If no hazard input is found within this distance, the \gls{asset} is skipped and a message is provided mentioning the id of the asset that is affected by this issue. If this parameter is not provided, the OpenQuake-Engine assumes the maximum allowable distance as 5 km.
\end{itemize}

Depending on the type of calculations, other parameters besides the
aforementioned ones need to be provided, as will be described in the following
sections.

\subsection{Scenario Damage Calculator}
\label{subsec:config_scenario_damage}
For this calculator, the parameter \Verb+calculation_mode+ needs to be defined as \Verb+scenario_damage+. There is only one parameter specific to this calculator, which is the \gls{fragility model} file path, as presented below.

\begin{Verbatim}[frame=single, commandchars=\\\{\}, samepage=true]
...
fragility_file = fragility_model.xml
\end{Verbatim}

\begin{itemize}
\item  \Verb+fragility_file+: a parameter used to define the path to the \gls{fragility model} file.
\end{itemize}

\subsection{Scenario Risk Calculator}
\label{subsec:config_scenario_risk}
In order to run this calculator, the parameter \Verb+calculation_mode+ needs to be set to \Verb+scenario_risk+. The remaining parameters are illustrated bellow.

\begin{Verbatim}[frame=single, commandchars=\\\{\}, samepage=true]
...
structural_vulnerability_file = struct_vul_model.xml
nonstructural_vulnerability_file = nonstruct_vul_model.xml
contents_vulnerability_file = cont_vul_model.xml
business_interruption_vulnerability_file = bus_int_vul_model.xml
occupants_vulnerability_file = occ_vul_model.xml

asset_correlation = 0.7
master_seed = 3
insured_losses = true
\end{Verbatim}

\begin{itemize}
\item  \Verb+structural_vulnerability_file+: this parameter is used to specify the path to the structural \gls{vulnerability model} file;
\item  \Verb+nonstructural_vulnerability_file+: this parameter is used to specify the path to the non-structural\gls{vulnerability model} file;
\item  \Verb+contents_vulnerability_file +: this parameter is used to specify the path to the contents \gls{vulnerability model} file;
\item  \Verb+business_interruption_vulnerability_file +: this parameter is used to specify the path to the business interruption \gls{vulnerability model} file;
\item  \Verb+vulnerability_file+: this parameter is used to specify the path to the occupants \gls{vulnerability model} file;
\item \texttt{asset\_cor\-re\-la\-tion} if the uncertainty in the loss ratios has been defined within the \gls{vulnerability model}, users can specify a coefficient of correlation that will be used in the Monte Carlo sampling process of the loss ratios, between the assets that share the same \gls{taxonomy}. If the \texttt{asset\_cor\-re\-la\-tion} is set to one, the loss ratio residuals will be perfectly correlated. On the other hand, if this parameter is set to zero, the loss ratios will be sampled independently. Any value between zero and one will lead to increasing levels of correlation. If this parameter is not defined, the OpenQuake-engine assumes no correlation in the vulnerability;
\item  \Verb+master_seed+: this parameter is used to control the random generator in the loss ratio sampling process. This way, if the same \Verb+master_seed+ is defined at each calculation run, the same random loss ratios will be generated, thus allowing replicability of the results;
\item  \Verb+insured_losses+: this parameter is used to define if insured losses should be calculated (\Verb+true+) or not (\Verb+false+).
\end{itemize}

\subsection{Classical Probabilistic Seismic Damage Calculator}
\label{subsec:config_classical_damage}
\input{oqum/risk/01a_config_classical_damage}

\subsection{Classical Probabilistic Seismic Risk Calculator}
\label{subsec:config_classical_risk}
\input{oqum/risk/01a_config_classical_risk}

\subsection{Event-Based Probabilistic Seismic Risk Calculator}
\label{subsec:config_event_based_risk}
\input{oqum/risk/01a_config_event_based_risk}

\subsection{Retrofit Benefit-Cost Ratio Calculator}
\label{subsec:config_benefit_cost}
\input{oqum/risk/01a_config_benefit_cost}

\cleardoublepage