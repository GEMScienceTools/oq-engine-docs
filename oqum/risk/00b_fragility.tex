This section describes the schema currently used to store \glspl{fragility
model}, which are required for the Scenario Damage Calculator and the
Classical Probabilistic Seismic Damage Calculator. A \gls{fragility model}
defines a set of \glspl{fragility function}, describing the probability of
exceeding a set of limit, or damage, states. These \gls{fragility function}
can be defined in two ways: discrete or continuous.

For discrete fragility functions, sets of probabilities of exceedance (one set
per limit state) are defined for a list of intensity measure levels, as
illustrated in Figure~\ref{fig:fragility-discrete}.

\begin{figure}[ht]
\centering
\includegraphics[width=12cm]{figures/risk/fragility-discrete.pdf}
\caption{Graphical representation of a discrete fragility model}
\label{fig:fragility-discrete}
\end{figure}

The \glspl{fragility function} can also be defined as continuous functions,
through the use of cumulative lognormal distribution functions. In
Figure~\ref{fig:fragility-continuous}, a continuous \gls{fragility model} is
presented.

\begin{figure}[ht]
\centering
\includegraphics[width=12cm]{figures/risk/fragility-continuous.pdf}
\caption{Graphical representation of a continuous fragility model}
\label{fig:fragility-continuous}
\end{figure}

An example \gls{fragility model} comprising one discrete \gls{fragility
function} and one continuous \gls{fragility function} is shown below:

\inputminted[firstline=1,firstnumber=1,fontsize=\footnotesize,frame=single,linenos,bgcolor=lightgray]{xml}{oqum/risk/Verbatim/input_fragility.xml}\\


The initial portion of the schema contains general information that describes 
some general aspects of the \gls{fragility model}.

\begin{itemize}

    \item \Verb+id+: a unique key used to identify the \gls{fragility model}

    \item \Verb+assetCategory+: an optional string used to specify the type of
    \glspl{asset} for which \glspl{fragility function} will be defined in this
    file (e.g: buildings, lifelines)

    \item \Verb+lossCategory+: valid strings for this attribute are 
    ``structural'', ``nonstructural'', ``contents'', and 
    ``business\_interruption''

    \item \Verb+description+: a brief string with further information about the
    \gls{exposure model}

\end{itemize}

\inputminted[firstline=4,firstnumber=4,lastline=9,fontsize=\footnotesize,frame=single,linenos,bgcolor=lightgray]{xml}{oqum/risk/Verbatim/input_fragility.xml}\\

The information in the metadata section is common to all of the functions in
the \gls{fragility model} and needs to be included at the beginning of every
\gls{fragility model} file. The parameters are described below:

\begin{itemize}

    \item \Verb+description+: a brief string with further information about the
    \gls{fragility model}, for example, which building typologies are covered or 
    the source of the functions in the \gls{fragility model}

    \item \Verb+limitStates+: this field is used to define the number and 
    nomenclature of each limit state. Four limit states are employed in the 
    example above, but it is possible to use any number of discrete states,
    as long as a fragility curve is always defined for each limit state. The 
    limit states must be provided as a set of strings separated by whitespaces 
    between each limit state. Please ensure that there is no whitespace within 
    the name of any individual limit state.

\end{itemize}

In order to perform probabilistic or scenario damage calculations, it is
necessary to define a \gls{fragility function} for each building typology present in
the exposure model. The \glspl{fragility function} can be defined using either a
discrete or a continuous format, and the \gls{fragility model} file can include a
mix of both types of \glspl{fragility function}.

The following snippet from the above fragility model example file defines a
discrete fragility function:

\inputminted[firstline=11,firstnumber=11,lastline=17,fontsize=\footnotesize,frame=single,linenos,bgcolor=lightgray]{xml}{oqum/risk/Verbatim/input_fragility.xml}\\

The following attributes are needed to define a discrete \gls{fragility function}:

\begin{itemize}

    \item \Verb+id+: a unique key used to identify the \gls{taxonomy} for 
    which the function is being defined. This key is used to relate the 
    \gls{fragility function} with the relevant \gls{asset} in the 
    \gls{exposure model}.

    \item \Verb+format+: for discrete fragility functions, this attribute 
    should be set to ``\Verb+discrete+''

    \item \Verb+imls+: this attribute specifies the list of intensity levels
    for which the limit state probabilities of exceedance will be defined. 
    In addition, it is also necessary to define the intensity measure type 
    (\Verb+imt+). Optionally, a \Verb+noDamageLimit+ can be specified, which 
    defines the intensity level below which the probability of exceedance 
    for all limit states is taken to be zero.

    \item \Verb+poes+: this field is used to define the probabilities of 
    exceedance (\Verb+poes+) for each limit state for this 
    \gls{fragility function}. It is also necessary to specify which limit 
    state the exceedance probabilities are being defined for using the 
    attribute \Verb+ls+. The probabilities of exceedance for each limit state
    must be provided on a separate line; and the number of exceedance 
    probabilities for each limit state defined by the \Verb+poes+ attribute 
    must be equal to the number of intensity levels defined by the attribute 
    \Verb+imls+. Finally, the number and names of the limit states in each 
    fragility function must be equal to the number of limit states defined 
    earlier in the metadata section of the \gls{fragility model} using the 
    attribute \Verb+limitStates+.

\end{itemize}



The following snippet from the above \gls{fragility model} example file
defines a continuous \gls{fragility function}:

\inputminted[firstline=19,firstnumber=19,lastline=25,fontsize=\footnotesize,frame=single,linenos,bgcolor=lightgray]{xml}{oqum/risk/Verbatim/input_fragility.xml}\\

The following attributes are needed to define a continuous \gls{fragility function}:

\begin{itemize}

    \item \Verb+id+: a unique key used to identify the \gls{taxonomy} for 
    which the function is being defined. This key is used to relate the 
    \gls{fragility function} with the relevant \gls{asset} in the 
    \gls{exposure model}.

    \item \Verb+format+: for continuous fragility functions, this attribute 
    should be set to ``\Verb+continuous+''

    \item \Verb+shape+: for continuous fragility functions using the lognormal
    cumulative distrution, this attribute should be set to ``\Verb+logncdf+''.
    At present, only the lognormal cumulative distribution function can be 
    used for representing continuous fragility functions.

    \item \Verb+imls+: this element specifies various aspects related to the 
    intensity measure used by the the \gls{fragility function}. The range of 
    intensity levels for which the continuous fragility functions are valid 
    are specified using the attributes \Verb+minIML+ and \Verb+maxIML+. 
    In addition, it is also necessary to define the intensity measure type 
    \Verb+imt+. Optionally, a \Verb+noDamageLimit+ can be specified, which 
    defines the intensity level below which the probability of exceedance 
    for all limit states is taken to be zero.

    \item \Verb+params+: this field is used to define the parameters of 
    the (\Verb+params+) for each limit state for this 
    \gls{fragility function}. For a lognormal cumulative distrbution function, 
    the two parameters required to specify the function are the mean and 
    standard deviation of the intensity level. These parameters are defined for 
    each limit state using the attributes \Verb+mean+ and \Verb+stddev+ 
    respectively. The attribute \Verb+ls+ specifies the limit state for which 
    the parameters are being defined. The parameters for each limit state
    must be provided on a separate line. The number and names of the limit 
    states in each fragility function must be equal to the number of limit 
    states defined earlier in the metadata section of the \gls{fragility model}
    using the attribute \Verb+limitStates+.

\end{itemize}


Several methodologies to derive fragility functions are currently being
evaluated by \gls{acr:gem} and have been included as part of the Risk
Modeller's Toolkit, the code for which can be found on a public repository at
GitHub at the following address: 
\href{http://github.com/gemsciencetools/rmtk}{http://github.com/gemsciencetools/rmtk}.

Scripts to convert \glspl{fragility function} in CSV format or as Excel or
ASCII files into NRML are also under development, and can be found at the
OpenQuake platform at the following address:
\href{https://platform.openquake.org/risk_input_preparation_toolkit/}{https://platform.openquake.org/risk\_input\_preparation\_toolkit/}.
