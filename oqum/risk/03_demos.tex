This sections describes the set of demos that have been compile to exercise the OpenQuake-engine. These demos can be found in a public repository in GitHub at the following link \href{http://github.com/gem/oq-engine/tree/master/demos}{http://github.com/gem/oq-engine/tree/master/demos}. Furthermore, a folder containing all of these demonstrative examples is provided when an OATS (OpenQuake Alpha Testing Service) account is requested, and it is also part of the OpenQuake-engine virtual image package. These examples are purely demonstrative and do not intend to represent accurately the seismicity, vulnerability or exposure characteristics of the region of interest, but simply to provide example input files that can be used as a benchmark for users planning to employ the OpenQuake-engine in seismic risk and loss estimation studies. Is is also noted that in the demonstrative examples presented in this section, illustrations about the various messages from the engine displayed in the command line interface are presented. These messages often contain information about the calculation id and output id, which will certainly be different for each user.

The five demos use Nepal as the region of interest. An example \gls{exposure model} has been developed for this region, comprising 9144 assets distributed amongst 2221 locations (due to the existence of more than one \gls{asset} at the same location). A map with the distribution of the number of buildings throughout Nepal is presented in Figure~\ref{fig:expNepal}.

\begin{figure}[ht]
\centering
\includegraphics[width=12cm,height=8cm]{figures/risk/NepalExposure.pdf}
\caption{Distribution of number of buildings in Nepal.}
\label{fig:expNepal}
\end{figure}

The building portfolio was organised into four classes for the rural areas (adobe, dressed stone, unreinforced fired brick, wooden frames), and five classes for the urban areas (the aforementioned typologies, in addition to reinforced concrete buildings). For each one of these building typologies, \glspl{vulnerability function} and \glspl{fragility function} were collected from the literature. These input models are only for demonstrative purposes and for further information about the building characteristics of Nepal, users are advised to contact the National Society for Earthquake Technology of Nepal (NSET - \href{http://www.nset.org.np/}{http:www.nset.org.np/}).

This section includes instructions not only on how to run the risk calculations, but also on how to produce the necessary hazard input. Thus, each demo comprises the configuration file, exposure model and fragility/vulnerability models fundamental for the risk calculations, but also a configuration file and associated input models to produce the hazard input.


\subsection{Scenario Damage Demos}
\label{subsec:demos_scenario_damage}
\input{oqum/risk/03a_demos_scenario_damage}

\subsection{Scenario Risk Demos}
\label{subsec:demos_scenario_risk}
\input{oqum/risk/03a_demos_scenario_risk}

\subsection{Classical Probabilistic Seismic Damage Demos}
\label{subsec:demos_classical_damage}
\input{oqum/risk/03a_demos_classical_damage}

\subsection{Classical Probabilistic Seismic Risk Demos}
\label{subsec:demos_classical_risk}
\input{oqum/risk/03a_demos_classical_risk}

\subsection{Event-Based Probabilistic Seismic Risk Demos}
\label{subsec:demos_event_based_risk}
\input{oqum/risk/03a_demos_event_based_risk}

\subsection{Retrofit Benefit-Cost Ratio Demos}
\label{subsec:demos_benefit_cost}
The loss exceedance curves used within this demo are produced using the Classical PSHA-based Risk calculator. Thus, the process to produce the seismic hazard curves described in the respective section (\ref{subsec:demos_classical_risk}) can be employed here. Then, the risk calculations can be initiated using the following command:

\begin{Verbatim}[frame=single, commandchars=\\\{\}, samepage=true]
user@ubuntu:~\$ oq-engine --rr job_bcr.ini --hazard-output-id 27
\end{Verbatim}

which should produce the following output:

\begin{Verbatim}[frame=single, commandchars=\\\{\}, samepage=true]
Calculation 17 results:
id | output_type | name
37 | bcr_distribution | BCR Distribution for hazard 27
\end{Verbatim}