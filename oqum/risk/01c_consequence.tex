Starting from OpenQuake-engine v1.7, the Scenario Damage calculator also
accepts \glspl{consequence model} in addition to \glspl{fragility model}, in
order to estimate consequences based on the calculated damage distribution.
The user may provide one \gls{consequence model} file corresponding to each
loss type (amongst structural, nonstructural, contents, and business
interruption) for which a fragility model file is provided. Whereas providing
a \gls{fragility model} file for at least one loss type is mandatory for
running a Scenario Damage calculation, providing corresponding
\gls{consequence model} files is optional.

This section describes the schema currently used to store \glspl{consequence
model}, which are optional inputs for the Scenario Damage Calculator. A
\gls{consequence model} defines a set of \glspl{consequence function},
describing the distribution of the loss (or consequence) ratio conditional on
a set of discrete limit (or damage) states. These \gls{consequence function}
can be currently defined in OpenQuake-engine by specifying the parameters of
the continuous distribution of the loss ratio for each limit state specified
in the fragility model for the corresponding loss type, for each taxonomy
defined in the exposure model.

An example consequence model is shown in Listing~\ref{lst:input_consequence}.

\begin{listing}[htbp]
  \inputminted[firstline=1,firstnumber=1,fontsize=\footnotesize,frame=single,linenos,bgcolor=lightgray]{xml}{oqum/risk/Verbatim/input_consequence.xml}
  \caption{Example consequence model}
  \label{lst:input_consequence}
\end{listing}	

The initial portion of the schema contains general information that describes
some general aspects of the \gls{consequence model}. The information in this
metadata section is common to all of the functions in the \gls{consequence
model} and needs to be included at the beginning of every \gls{consequence
model} file. The parameters are described below:

\begin{itemize}

    \item \Verb+id+: a unique key used to identify the \gls{consequence model}

    \item \Verb+assetCategory+: an optional string used to specify the type of
    \glspl{asset} for which fragility functions will be defined in this file 
    (e.g: buildings, lifelines)

    \item \Verb+lossCategory+: valid strings for this attribute are 
    ``structural'', ``nonstructural'', ``contents'', and 
    ``business\_interruption''

    \item \Verb+description+: a brief string with further information about the
    \gls{consequence model}, for example, which building typologies are covered or 
    the source of the functions in the \gls{consequence model}

    \item \Verb+limitStates+: this field is used to define the number and 
    nomenclature of each limit state. Four limit states are employed in the 
    example above, but it is possible to use any number of discrete states,
    as long as a fragility curve is always defined for each limit state. The 
    limit states must be provided as a set of strings separated by whitespaces 
    between each limit state. Please ensure that there is no whitespace within 
    the name of any individual limit state.

\end{itemize}

\inputminted[firstline=4,firstnumber=4,lastline=9,fontsize=\footnotesize,frame=single,linenos,bgcolor=lightgray]{xml}{oqum/risk/Verbatim/input_consequence.xml}

The following snippet from the above consequence model example file defines a
consequence function using a lognormal distribution to model the uncertainty
in the consequence ratio for each limit state:

\inputminted[firstline=11,firstnumber=11,lastline=16,fontsize=\footnotesize,frame=single,linenos,bgcolor=lightgray]{xml}{oqum/risk/Verbatim/input_consequence.xml}

The following attributes are needed to define a consequence function:

\begin{itemize}

    \item \Verb+id+: a unique key used to identify the \gls{taxonomy} for 
    which the function is being defined. This key is used to relate the 
    \gls{consequence function} with the relevant \gls{asset} in the 
    \gls{exposure model}.

    \item \Verb+dist+: for vulnerability function which use a continuous 
    distribution to model the uncertainty in the conditional loss ratios, 
    this attribute should be set to either ``\Verb+LN+'' if using the lognormal
    distribution, or to ``\Verb+BT+'' if using the Beta distribution
    \footnote{Note that in OpenQuake-engine v1.7, the uncertainty in the 
    consequence ratios is ignored, and only the mean consequence ratios for the
    set of limit states is considered when computing the consequences from the
    damage distribution. Consideration of the uncertainty in the consequence
    ratios will be included in future releases of the OpenQuake-engine.}.

    \item \Verb+params+: this field is used to define the parameters of 
    the continuous distribution used for modelling the uncertainty in the
    loss ratios for each limit state for this 
    \gls{consequence function}. For a lognormal distrbution, 
    the two parameters required to specify the function are the mean and 
    standard deviation of the consequence ratio. These parameters are defined for 
    each limit state using the attributes \Verb+mean+ and \Verb+stddev+ 
    respectively. The attribute \Verb+ls+ specifies the limit state for which 
    the parameters are being defined. The parameters for each limit state
    must be provided on a separate line. The number and names of the limit 
    states in each \gls{consequence function} must be equal to the number of limit 
    states defined in the corresponding \gls{fragility model}
    using the attribute \Verb+limitStates+.

\end{itemize}