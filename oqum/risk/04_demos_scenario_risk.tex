A rupture of magnitude 7 Mw in the central part of Nepal was considered within
this demo. The characteristics of this rupture (geometry, dip, rake,
hypocentre,  upper and lower seismogenic depth) have been defined in the
\verb+rupture.xml+ file, whist the hazard calculation settings have been
established on the \verb+job_hazard.ini+ file. In order to calculate the set of
ground motion fields due to this rupture, users should navigate to the folder
where the demo files are located, and use the following command:

\begin{Verbatim}[frame=single, commandchars=\\\{\}, samepage=true]
user@ubuntu:~\$ oq-engine --run job_hazard.ini
\end{Verbatim}

which will produce the following hazard result:

\begin{Verbatim}[frame=single, commandchars=\\\{\}, samepage=true]
Calculation 10 results:
id | output_type | name
20 | gmf_scenario | gmf_scenario
\end{Verbatim}

Then, this hazard input can be used for the risk calculations using the
following command:

\begin{Verbatim}[frame=single, commandchars=\\\{\}, samepage=true]
user@ubuntu:~\$ oq-engine --run job_risk.ini --hazard-output-id 20
\end{Verbatim}

leading to the following results:

\begin{Verbatim}[frame=single, commandchars=\\\{\}, samepage=true]
Calculation 11 results:
id | output_type | name
21 | aggregate_loss | Aggregate Loss type=structural
22 | loss_map | loss maps. type=structural
\end{Verbatim}