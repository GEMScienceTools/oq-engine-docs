In order to run this calculator, the parameter \Verb+calculation_mode+ needs to be set to \Verb+classical_risk+. With this calculator it is also possible to extract loss maps, so the parameter \Verb+conditional_loss_poes+ needs to be defined as explained in the previous sub-section. The remaining parameters are illustrated below:
\begin{Verbatim}[frame=single, commandchars=\\\{\}, samepage=true]
...
structural_vulnerability_file = struct_vul_model.xml
nonstructural_vulnerability_file = nonstruct_vul_model.xml
contents_vulnerability_file = cont_vul_model.xml
business_interruption_vulnerability_file = bus_int_vul_model.xml
occupants_vulnerability_file = occ_vul_model.xml

lrem_steps_per_interval = 2
conditional_loss_poes = 0.01, 0.05, 0.10
\end{Verbatim}

\begin{itemize}
\item  \Verb+lrem_steps_per_interval+: this parameter controls the number of intermediate values between consecutive loss ratios (as defined in the \gls{vulnerability model}) that are considered in the risk calculations. A larger number of loss ratios than those defined in each \gls{vulnerability function} should be considered, in order to better account for the uncertainty in the loss ratio distribution. If this parameter is not defined in the configuration file, the OpenQuake-engine assumes the \Verb+lrem_steps_per_interval+ to be equal to 5. More details are provided in the OpenQuake-engine Book (Risk).
\end{itemize}