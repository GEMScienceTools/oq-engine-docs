In order to run this calculator, the parameter \Verb+calculation_mode+ needs
to be set to \Verb+classical_risk+.

Most of the job configuration parameters required for running a classical
probabilistic risk calculation are the same as those described in the previous
section for the classical probabilistic damage calculator. The remaining
parameters specific to the classical probabilistic risk calculator are
illustrated through the examples below.

\paragraph{Example 1}

This example illustrates a classical probabilistic risk calculation which uses
a single configuration file to first compute the hazard curves for the given
source model and ground motion model and then calculate loss exceedance curves
and maps based on the hazard curves. A minimal job configuration file required
for running a classical probabilistic risk calculation is shown below:

\inputminted[firstline=1,firstnumber=1,fontsize=\footnotesize,frame=single,linenos,bgcolor=lightgray,label=job.ini]{ini}{oqum/risk/verbatim/config_classical_risk_combined.ini}\\

Apart from the calculation mode, the only difference with the example job
configuration file shown in Example~1 of
Section~\ref{sec:config_classical_damage} is the use of a vulnerability model
instead of a fragility model.

As with the Scenario Risk calculator, it is possible to specify one or more
vulnerability model files in the same job configuration file, using the
parameters:

\begin{itemize}

  \item \Verb+structural_vulnerability_file+,

  \item \Verb+nonstructural_vulnerability_file+,

  \item \Verb+contents_vulnerability_file+,

  \item \Verb+business_interruption_vulnerability_file+, and/or

  \item \Verb+occupants_vulnerability_file+

\end{itemize}

It is important that the
\Verb+lossCategory+ parameter in the metadata section for each provided
vulnerability model file (``structural'', ``nonstructural'', ``contents'',
``business\_interruption'', or ``occupants'') should match the loss type
defined in the configuration file by the relevant keyword above.

In this case, the hazard curves will be computed at each of the locations of
the assets in the exposure model, for each of the intensity measure types
found in the provided set of vulnerability models. The above calculation can
be run using the command line:

\begin{minted}[fontsize=\footnotesize,frame=single,bgcolor=lightgray]{shell-session}
user@ubuntu:~\$ oq-engine --run job.ini
\end{minted}

After the calculation is completed, a message similar to the following will be
displayed:

\begin{minted}[fontsize=\footnotesize,frame=single,bgcolor=lightgray]{shell-session}
Calculation 2749 completed in 24 seconds. Results:
  id | output_type | name
5373 | Loss Curve  | loss curves. type=structural, hazard=5371
5374 | Loss Curve  | loss curves. type=nonstructural, hazard=5371
5375 | Loss Curve  | insured loss curves. type=structural hazard=5371
\end{minted}


\paragraph{Example 2}

This example illustrates a classical probabilistic risk calculation which uses
separate configuration files for the hazard and risk parts of a classical
probabilistic risk assessment. The first configuration file contains input
models and parameters required for the computation of the hazard curves. The
second configuration file contains input models and parameters required for
the calculation of the loss exceedance curves and probabilistic loss maps for
a portfolio of assets based on the hazard curves and vulnerability models.

\inputminted[firstline=1,firstnumber=1,fontsize=\footnotesize,frame=single,linenos,bgcolor=lightgray,label=job\_hazard.ini]{ini}{oqum/risk/verbatim/config_classical_hazard.ini}\\

\inputminted[firstline=1,firstnumber=1,fontsize=\footnotesize,frame=single,linenos,bgcolor=lightgray,label=job\_risk.ini]{ini}{oqum/risk/verbatim/config_classical_risk.ini}\\

Now, the above calculations described by the two configuration files
``job\_hazard.ini'' and ``job\_risk.ini'' can be run sequentially or
separately, as illustrated in Example~2 in
Section~\ref{sec:config_scenario_damage}. The new parameters introduced in the
above example configuration file are described below:

\begin{itemize}

	\item \Verb+lrem_steps_per_interval+: this parameter controls the number of
	  intermediate values between consecutive loss ratios (as defined in the 
	  \gls{vulnerability model}) that are considered in the risk calculations.
	  A larger number of loss ratios than those defined in each
	  \gls{vulnerability function} should be considered, in order to better
	  account for the uncertainty in the loss ratio distribution. If this
	  parameter is not defined in the configuration file, the OpenQuake-engine
	  assumes the \Verb+lrem_steps_per_interval+ to be equal to 5. More details
	  are provided in the OpenQuake-engine Book (Risk).

	\item \Verb+quantile_loss_curves+: this parameter can be used to request
	  the computation of quantile loss curves for computations involving
	  nontrivial logic trees. The quantiles for which the loss curves should
	  be computed must be provided as a comma separated list. If this parameter
	  is not included in the configuration file, quantile loss curves will not
	  be computed. 

	\item \Verb+conditional_loss_poes+: this parameter can be used to request
	  the computation of probabilistic loss maps, which give the loss levels
	  exceeded at the specified probabilities of exceedance over the time
	  period specified by \Verb+risk_investigation_time+. The probabilities of
	  exceedance for which the loss maps should be computed must be provided as
	  a comma separated list. If this parameter is not included in the
	  configuration file, probabilistic loss maps will not be computed.

\end{itemize}