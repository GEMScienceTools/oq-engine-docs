In order to run this calculator, the parameter \Verb+calculation_mode+ needs
to be set to \Verb+classical_risk+.

Most of the job configuration parameters required for running a classical
probabilistic risk calculation are the same as those described in the previous
section for the classical probabilistic damage calculator. The remaining
parameters specific to the classical probabilistic risk calculator are
illustrated through the examples below.

\paragraph{Example 1}

This example illustrates a classical probabilistic risk calculation which uses
a single configuration file to first compute the hazard curves for the given
source model and ground motion model and then calculate loss exceedance curves
and maps based on the hazard curves. A minimal job configuration file required
for running a classical probabilistic risk calculation is shown below:

\inputminted[firstline=1,firstnumber=1,fontsize=\footnotesize,frame=single,linenos,bgcolor=lightgray,label=job.ini]{ini}{oqum/risk/verbatim/config_classical_risk_combined.ini}\\

Apart from the calculation mode, the only difference with the example job
configuration file shown in Example~1 of
Section~\ref{sec:config_classical_damage} is the use of a vulnerability model
instead of a fragility model.

As with the Scenario Risk calculator, it is possible to specify one or more
vulnerability model files in the same job configuration file, using the
parameters:

\begin{itemize}

  \item \Verb+structural_vulnerability_file+,

  \item \Verb+nonstructural_vulnerability_file+,

  \item \Verb+contents_vulnerability_file+,

  \item \Verb+business_interruption_vulnerability_file+, and/or

  \item \Verb+occupants_vulnerability_file+

\end{itemize}

It is important that the
\Verb+lossCategory+ parameter in the metadata section for each provided
vulnerability model file (``structural'', ``nonstructural'', ``contents'',
``business\_interruption'', or ``occupants'') should match the loss type
defined in the configuration file by the relevant keyword above.

In this case, the hazard curves will be computed at each of the locations of
the assets in the exposure model, for each of the intensity measure types
found in the provided set of vulnerability models. The above calculation can
be run using the command line:

\begin{Verbatim}[frame=single, commandchars=\\\{\}, samepage=true]
user@ubuntu:~\$ oq-engine --run job.ini
\end{Verbatim}

After the calculation is completed, a message similar to the following will be
displayed:

\begin{Verbatim}[frame=single, commandchars=\\\{\}, samepage=true]
Calculation 2749 completed in 24 seconds. Results:
  id | output_type | name
5373 | Loss Curve | loss curves. type=structural, hazard=5371
5374 | Loss Curve | loss curves. type=nonstructural, hazard=5371
5375 | Loss Curve | insured loss curves. type=structural hazard=5371
\end{Verbatim}
