The seismic source model developed within the Global Seismic Hazard Assessment
Program (GSHAP) was used with the \cite{chiou2008} ground motion prediction
equation to produce the hazard input for this demo. No uncertainties are
considered in the seismic source model and since only one GMPE is being
considered, there will be only one possible path in the logic tree. Therefore,
only one set seismic hazard curves will be produced. To do so, the following
command needs to be employed:

\begin{Verbatim}[frame=single, commandchars=\\\{\}, samepage=true]
user@ubuntu:~\$ oq-engine --run job_hazard.ini
\end{Verbatim}

which will produce the following hazard output:

\begin{Verbatim}[frame=single, commandchars=\\\{\}, samepage=true]
Calculation 13 results:
id | output_type | name
27 | hazard_curve | hc-rlz-70
\end{Verbatim}

In this demo, loss exceedance curves for each asset and two probabilistic loss
maps (for probabilities of exceedance of 1\% and 10\%) are produced. The
following command launches these risk calculations:

\begin{Verbatim}[frame=single, commandchars=\\\{\}, samepage=true]
user@ubuntu:~\$ oq-engine --run job_risk.ini --hazard-output-id 27
\end{Verbatim}

and the following outputs are expected:

\begin{Verbatim}[frame=single, commandchars=\\\{\}, samepage=true]
Calculation 14 results:
id | output_type | name
28 | loss_curve | loss curves. type=structural, hazard=27
29 | loss_map | loss maps. type=structural poe=0.1, hazard=27
30 | loss_map | loss maps. type=structural poe=0.01, hazard=27
\end{Verbatim}