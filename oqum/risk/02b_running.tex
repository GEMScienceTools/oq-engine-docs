Using the command line interface, risk calculations can be launched and the
resulting outputs can be extracted. This section describes all the currently
implemented commands and presents examples for each of the calculators. One of
the first tasks that needs to be performed is the definition of the seismic
hazard input.

Whether a user chooses to load pre-computed ground motion fields, or calculate
this input using the hazard component of the OpenQuake-engine, a unique
\verb+id+ is associated to the set of ground motion fields, as depicted below.

\begin{Verbatim}[frame=single, commandchars=\\\{\}, samepage=true]
Calculation 3 results:
id | output_type | name
12 | gmf_scenario | gmf_scenario
\end{Verbatim}

This is the parameter that will be used when launching the risk calculations
to indicate which hazard input should be employed. In the case of the
scenario-based calculators, there is only a single hazard input (one or a set
of ground motion fields). For the remaining calculators, where probabilistic
seismic hazard is used, it is possible to have multiple hazard inputs due to
the employment of logic trees, as described in
Section~\ref{sec:hazard_logic_trees}. In the following illustration, a set of
hazard results produced using the Classical PSHA calculator is presented.

\begin{Verbatim}[frame=single, commandchars=\\\{\}, samepage=true]
Calculation 4 results:
id | output_type | name
32 | hazard_curve | hc-rlz-32-PGA
33 | hazard_curve | hc-rlz-33-PGA
34 | hazard_curve | hc-rlz-34-PGA
35 | hazard_curve | hc-rlz-35-PGA
36 | hazard_curve | mean curve for PGA
37 | hazard_curve | quantile curve (poe>= 0.15) for imt PGA
38 | hazard_curve | quantile curve (poe>= 0.85) for imt PGA
\end{Verbatim}

In this case, since the logic tree had four branches, fours sets of hazard
curves were produced, each one with its own \verb+id+. In addition, mean and
quantile hazard curves were also produced. A user may choose to run risk
calculations using results from one of the branches or mean/quantile curves.
To do so, the id of the respective hazard result should be employed when
launching the risk calculations, as depicted below.

\begin{Verbatim}[frame=single, commandchars=\\\{\}, samepage=true]
user@ubuntu:~\$ oq-engine --run-risk job.ini --hazard-output-id
<hazard_output_id>
\end{Verbatim}

or simply:

\begin{Verbatim}[frame=single, commandchars=\\\{\}, samepage=true]
user@ubuntu:~\$ oq-engine --rr job.ini --ho <hazard_output_id>
\end{Verbatim}

On the other hand, a user might also want to run the risk calculations
considering all the hazard results from a certain calculation run. In this
case, rather than providing the \verb+hazard-output-id+, users need to provide
the id of the hazard calculation as follows.

\begin{Verbatim}[frame=single, commandchars=\\\{\}, samepage=true]
user@ubuntu:~\$ oq-engine --run-risk job.ini --hazard-calculation-id
<hazard_calculation_id>
\end{Verbatim}

or simply:

\begin{Verbatim}[frame=single, commandchars=\\\{\}, samepage=true]
user@ubuntu:~\$ oq-engine --rr job.ini --co <hazard_calculation_id>
\end{Verbatim}