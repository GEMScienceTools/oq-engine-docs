In order to run this calculator, the parameter \Verb+calculation_mode+ needs
to be set to \Verb+scenario_risk+. 

Most of the job configuration parameters required for running a scenario risk
calculation are the same as those described in the previous section for the
scenario damage calculator. The remaining parameters specific to the scenario
risk calculator are illustrated through the examples below.


\paragraph{Example 1}

This example illustrates a scenario risk calculation which uses a single
configuration file to first compute the ground motion fields for the given
rupture model and then calculate loss statistics for structural losses,
nonstructural losses, and insured structural losses, based on the ground
motion fields. The job configuration file required for running this scenario
risk calculation is shown in Listing~\ref{lst:config_scenario_risk_combined}.

\begin{listing}[htbp]
  \inputminted[firstline=1,firstnumber=1,fontsize=\footnotesize,frame=single,linenos,bgcolor=lightgray,label=job.ini]{ini}{oqum/risk/verbatim/config_scenario_risk_combined.ini}
  \caption{Example combined configuration file for a scenario risk calculation (\href{https://raw.githubusercontent.com/GEMScienceTools/oq-engine-docs/master/oqum/risk/verbatim/config_scenario_risk_combined.xml}{Download example})}
  \label{lst:config_scenario_risk_combined}
\end{listing}

Whereas a scenario damage calculation requires one or more fragility and/or
consequence models, a scenario risk calculation requires the user to specify
one or more vulnerability model files. Note that one or more of the following
parameters can be used in the same job configuration file to provide the
corresponding vulnerability model files:

\begin{itemize}

  \item \Verb+structural_vulnerability_file+: this parameter is used to
    specify the path to the structural \gls{vulnerabilitymodel} file

  \item \Verb+nonstructural_vulnerability_file+: this parameter is used to
    specify the path to the nonstructural\gls{vulnerabilitymodel} file

  \item \Verb+contents_vulnerability_file+: this parameter is used to
    specify the path to the contents \gls{vulnerabilitymodel} file

  \item \Verb+business_interruption_vulnerability_file+: this parameter is
    used to specify the path to the business interruption
    \gls{vulnerabilitymodel} file

  \item \Verb+occupants_vulnerability_file+: this parameter is used to
    specify the path to the occupants \gls{vulnerabilitymodel} file

\end{itemize}

It is important that the \Verb+lossCategory+ parameter in the metadata section
for each provided vulnerability model file (``structural'', ``nonstructural'',
``contents'', ``business\_interruption'', or ``occupants'') should match the
loss type defined in the configuration file by the relevant keyword above.

The remaining new parameters introduced in this example are the following:

\begin{itemize}

  \item \Verb+master_seed+: this parameter is used to control the random
    number generator in the loss ratio sampling process. If the same
    \Verb+master_seed+ is defined at each calculation run, the same random loss
    ratios will be generated, thus allowing reproducibility of the results.

  \item \Verb+asset_correlation+: if the uncertainty in the loss ratios
    has been defined within the \gls{vulnerabilitymodel}, users can specify
    a coefficient of correlation that will be used in the Monte Carlo sampling
    process of the loss ratios, between the assets that share the same
    \gls{taxonomy}. If the \Verb+asset_correlation+ is set to one,
    the loss ratio residuals will be perfectly correlated. On the other hand,
    if this parameter is set to zero, the loss ratios will be sampled
    independently. Any value between zero and one will lead to increasing
    levels of correlation. If this parameter is not defined, the
    \glsdesc{acr:oqe} will assume zero correlation in the vulnerability. As of
    \glsdesc{acr:oqe18}, \Verb+asset_correlation+ applies only to continuous
    \glspl{vulnerabilityfunction} using the lognormal or Beta distribution; 
    it does not apply to \glspl{vulnerabilityfunction} defined using the PMF
    distribution.

  \item \Verb+insured_losses+: this parameter specifies whether insured losses
    should be calculated; the default value of this parameter is \Verb+false+.
    In order for the \glsdesc{acr:oqe} to be able to compute insured losses, the
    insurance limits and deductibles must be listed for each asset in the 
    exposure model, as described in Example~5 in Section~\ref{sec:exposure}.

\end{itemize}

In this case, the ground motion fields will be computed at each of the
locations of the assets in the exposure model and for each of the intensity
measure types found in the provided set of vulnerability models. The above
calculation can be run using the command line:

\begin{minted}[fontsize=\footnotesize,frame=single,bgcolor=lightgray]{shell-session}
user@ubuntu:~\$ oq-engine --run job.ini
\end{minted}

After the calculation is completed, a message similar to the following will be
displayed:

\begin{minted}[fontsize=\footnotesize,frame=single,bgcolor=lightgray]{shell-session}
Calculation 2735 completed in 10 seconds. Results:
  id | output_type | name
5328 | datastore | agglosses-rlzs
5329 | datastore | loss_map-rlzs
\end{minted}

All of the different ways of running a scenario damage calculation as
illustrated through the examples of the previous section are also applicable
to the scenario risk calculator, though the examples are not repeated here.