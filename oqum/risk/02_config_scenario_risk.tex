In order to run this calculator, the parameter \Verb+calculation_mode+ needs
to be set to \Verb+scenario_risk+. Most of the job configuration parameters 
required for running a scenario risk calculation are the same as those 
described in the previous section for the scenario damage calculator.
The remaining parameters specific to the scenario risk calculator 
are illustrated below.

\begin{Verbatim}[frame=single, commandchars=\\\{\}, samepage=true]
...
structural_vulnerability_file = structural_vulnerability_model.xml
nonstructural_vulnerability_file = nonstructural_vulnerability_model.xml
contents_vulnerability_file = contents_vulnerability_model.xml
business_interruption_vulnerability_file = downtime_vulnerability_model.xml
occupants_vulnerability_file = occupants_vulnerability_model.xml

asset_correlation = 0.7
master_seed = 3
insured_losses = true
\end{Verbatim}

\begin{itemize}

  \item \Verb+structural_vulnerability_file+: this parameter is used to
    specify the path to the structural \gls{vulnerability model} file

  \item \Verb+nonstructural_vulnerability_file+: this parameter is used to
    specify the path to the nonstructural\gls{vulnerability model} file

  \item \Verb+contents_vulnerability_file +: this parameter is used to
    specify the path to the contents \gls{vulnerability model} file

  \item \Verb+business_interruption_vulnerability_file +: this parameter is
    used to specify the path to the business interruption
    \gls{vulnerability model} file

  \item \Verb+occupants_vulnerability_file+: this parameter is used to
    specify the path to the occupants \gls{vulnerability model} file

  \item \texttt{asset\_correlation} if the uncertainty in the loss ratios
    has been defined within the \gls{vulnerability model}, users can specify
    a coefficient of correlation that will be used in the Monte Carlo sampling
    process of the loss ratios, between the assets that share the same
    \gls{taxonomy}. If the \texttt{asset\_correlation} is set to one,
    the loss ratio residuals will be perfectly correlated. On the other hand,
    if this parameter is set to zero, the loss ratios will be sampled
    independently. Any value between zero and one will lead to increasing
    levels of correlation. If this parameter is not defined, the
    OpenQuake-engine will assume zero correlation in the vulnerability.

  \item \Verb+master_seed+: this parameter is used to control the random
    number generator in the loss ratio sampling process. If the same
    \Verb+master_seed+ is defined at each calculation run, the same random loss
    ratios will be generated, thus allowing replicability of the results.

  \item \Verb+insured_losses+: this parameter specifies whether insured losses
    should be calculated (\Verb+true+) or not (\Verb+false+).

\end{itemize}