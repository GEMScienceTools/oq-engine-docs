\emph{All} risk calculators in the OpenQuake-engine require an \gls{exposure
model} that needs to be provided in the NRML format. The information included
in an exposure model comprises a metadata section listing general information
about the exposure, followed by a cost conversions section that describes how
the different areas, costs, and occupancies for the assets will be specified,
followed by data regarding each individual asset in the portfolio.

A minimal exposure model comprising a single asset is shown in the example
below:

\inputminted[firstline=1,firstnumber=1,fontsize=\footnotesize,frame=single,linenos,bgcolor=lightgray]{xml}{oqum/risk/Verbatim/input_exposure_minimal.xml}\\

Let us take a look at each of the sections in the above example file in turn.
The first part of the file contains the metadata section:

\inputminted[firstline=5,firstnumber=5,lastline=8,fontsize=\footnotesize,frame=single,linenos,bgcolor=lightgray]{xml}{oqum/risk/Verbatim/input_exposure_minimal.xml}

The information in the metadata section is common to all of the assets in the
portfolio and needs to be incorporated at the beginning of every exposure
model file. There are a number of parameters that compose the metadata
section, which is intended to provide general information regarding the
\glspl{asset} within the \gls{exposure model}. These parameters are described
below:

\begin{itemize}

    \item \Verb+id+: a unique key used to identify the \gls{exposure model}

    \item \Verb+category+: an optional string used to define the type of
    \glspl{asset} being stored (e.g: buildings, lifelines)

    \item \Verb+taxonomySource+: an optional attribute used to define the
    \gls{taxonomy} being used to classify the \glspl{asset}

    \item \Verb+description+: a brief string with further information about the
    \gls{exposure model}

\end{itemize}


Next, let us look at the part of the file describing the area and cost
conversions:

\inputminted[firstline=10,firstnumber=10,lastline=15,fontsize=\footnotesize,frame=single,linenos,bgcolor=lightgray]{xml}{oqum/risk/Verbatim/input_exposure_minimal.xml}\\

Notice that the \Verb+costType+ element defines a \Verb+name+, a \Verb+type+, 
and a \Verb+unit+ attribute.

The NRML schema for the exposure model allows the definition of a structural
cost, a nonstructural components cost, a contents cost, and a business
interruption or downtime cost for each asset in the portfolio. Thus, the valid
values for the \Verb+name+ attribute of the \Verb+costType+ element are the
following:

\begin{itemize}

    \item \Verb+structural+: used to specify the structural replacement cost
    of assets

    \item \Verb+nonstructural+: used to specify the replacement cost for the
    nonstructural components of assets

    \item \Verb+contents+: used to specify the contents replacement cost

    \item \Verb+business_interruption+: used to specify the cost that will be 
    incurred per unit time that a damaged asset remains closed following an 
    earthquake

\end{itemize}

The exposure model shown in the example above defines only the structural
values for the assets. However, multiple cost types can be defined for each
asset in the same exposure model.

The \Verb+unit+ attribute of the \Verb+costType+ element is used for
specifying the currency unit for the corresponding cost type. Note that the
OpenQuake-engine itself is agnostic to the currency units; the \Verb+unit+ is
thus a descriptive attribute which is used by OpenQuake-engine to annotate the
results of a risk assessment. This attribute can be set to any valid Unicode
string.

The \Verb+type+ attribute of the \Verb+costType+ element specifies whether the
costs will be provided as an aggregated value for an asset, or per building or
unit comprising an asset, or per unit area of an asset. The valid values for
the \Verb+type+ attribute of the \Verb+costType+ element are the following:

\begin{itemize}

    \item \Verb+aggregated+: indicates that the replacement costs will be 
    provided as an aggregated value for each asset 

    \item \Verb+per_asset+: indicates that the replacement costs will be 
    provided per building or unit comprising each asset

    \item \Verb+per_area+: indicates that the replacement costs will be 
    provided per unit area for each asset

\end{itemize}

If the costs are to be specified \Verb+per_area+ for any of the
\Verb+costTypes+, the \Verb+area+ element will also need to be defined in the
conversions section. The \Verb+area+ element defines a \Verb+type+, and a
\Verb+unit+ attribute.

The \Verb+unit+ attribute of the \Verb+area+ element is used for specifying
the units for the area of an asset. The OpenQuake-engine itself is agnostic to the
area units; the \Verb+unit+ is thus a descriptive attribute which is used by the
OpenQuake-engine to annotate the results of a risk assessment. This attribute
can be set to any valid Unicode string.

The \Verb+type+ attribute of the \Verb+area+ element specifies whether the
area will be provided as an aggregated value for an asset, or per building or
unit comprising an asset. The valid values for the \Verb+type+ attribute of
the \Verb+area+ element are the following:

\begin{itemize}

    \item \Verb+aggregated+: indicates that the area will be provided as an 
    aggregated value for each asset

    \item \Verb+per_asset+: indicates that the area will be provided per 
    building or unit comprising each asset

\end{itemize}


The way the information about the characteristics of the \glspl{asset} in an
\gls{exposure model} are stored can vary strongly depending on how and why the
data was compiled. As an example, if national census information is used to
estimated the distribution of assets in a given region, it is likely that the
number of buildings within a given geographical area will be used to define
the dataset, and will be used for estimating the number of collapsed buildings
for a scenario earthquake. On the other hand, if simplified methodologies
based on proxy data such as population distribution are used to develop the
exposure model, then it is likely that the built up area or economic cost of
each building typology will be directly derived, and will be used for the
estimation of economic losses.


Finally, let us look at the part of the file describing the set of assets in
the portfolio to be used in seismic damage or risk calculations:

\inputminted[firstline=17,firstnumber=17,lastline=27,fontsize=\footnotesize,frame=single,linenos,bgcolor=lightgray]{xml}{oqum/risk/Verbatim/input_exposure_minimal.xml}\\

Each asset definition involves specifiying a set of mandatory and optional
attributes concerning the asset. The following set of attributes can be
assigned to each asset based on the current schema for the exposure model:

\begin{itemize}

    \item \Verb+id+: mandatory; a unique key used to identify the given
    \gls{asset}, which is used by the OpenQuake-engine to relate each asset 
    with its associated results.

    \item \Verb+taxonomy+: mandatory; specifies the building typology of the 
    given \gls{asset}. The taxonomy strings can be user-defined, or based on
    an existing classification scheme such as 

    \item \Verb+number+: mandatory; the number of independent units or 
    individual structures comprising a given \gls{asset}

    \item \Verb+location+: mandatory; specifies the longitude 
    (between -180$^{\circ}$ to 180$^{\circ}$) and latitude 
    (between -90$^{\circ}$ to 90 $^{\circ}$) of the given \gls{asset}, both
    specified in decimal degrees\footnote{Within the OpenQuake-engine, 
    longitude and latitude coordinates are internally rounded to a precision
    of 5 digits after the decimal point.}.

    \item \Verb+area+: area of the \gls{asset}, at a given location. As 
    mentioned earlier, the area is a mandatory attribute only if any one of the 
    costs for the \gls{asset} is specified per unit area.

    \item \Verb+costs+: specifies a set of costs for the given \gls{asset}. 
    The replacement value for different cost types must be provided on 
    separate lines within the \Verb+costs+ element. As shown in the example 
    above, each cost entry must define the \Verb+type+ and the \Verb+value+. 
    Currently supported valid options for the cost \Verb+type+ are: 
    \Verb+structural+,  \Verb+nonstructural+, \Verb+contents+, and 
    \Verb+business_interruption+.

    \item \Verb+occupancies+: mandatory only for probabilistic or scenario 
    damage calculations. Each entry within this element specifies the number of
    occupants for the asset for a particular period of the day. As shown in 
    the example above, each occupancy entry must define the \Verb+period+ and 
    the \Verb+occupants+. Currently supported valid options for the 
    \Verb+period+ are: \Verb+day+, \Verb+transit+, and \Verb+night+. Currently,
    the number of \Verb+occupants+ for an asset can only be provided as an 
    aggregated value for the asset.

\end{itemize}

For the purposes of performing a retrofitting benefit/cost analysis, it
is also necessary to define the retrofitting cost (\Verb+reco+). The
combination between the possible options in which these three attributes can
be defined leads to four ways of storing the information about the assets. For
each of these cases a brief explanation and example is provided in this
section.


\paragraph{Example 1}

This example illustrates an \gls{exposure model} in which the aggregated
cost (structural, nonstructural, contents and business interruption) of the
buildings of each taxonomy for a set of locations is directly provided. Thus,
in order to indicate how the various costs will be defined, the following
information needs to be stored in the exposure model file:

\inputminted[firstline=8,firstnumber=8,lastline=18,fontsize=\footnotesize,frame=single,linenos,bgcolor=lightgray]{xml}{oqum/risk/Verbatim/input_exposure_cagg.xml}\\

In this case, the cost \Verb+type+ of each component as been defined as
\Verb+aggregated+. Once the way in which each cost is going to be defined has
been established, the values for each asset can be stored according to the
following format:

\inputminted[firstline=19,firstnumber=19,lastline=29,fontsize=\footnotesize,frame=single,linenos,bgcolor=lightgray]{xml}{oqum/risk/Verbatim/input_exposure_cagg.xml}\\

Each \gls{asset} is uniquely identified by its \Verb+id+, (e.g. loss
exceedance curves). Then, a pair of coordinates (latitude and longitude) for a
\Verb+location+ where the asset is assumed to exist is defined. Each asset
must be classified according to a \Verb+taxonomy+, so that the OpenQuake-
engine is capable of employing the appropriate \gls{vulnerability function} or
\gls{fragility function} in the risk calculations. Finally, the cost values of
each \Verb+type+ are stored within the \Verb+costs+ attribute. In this
example, the aggregated value for all units (within a given asset) at each
location is provided directly, so there is no need to define other attributes
such as \Verb+number+ or \Verb+area+. This mode of representing an exposure
model is probably the simplest one.


\paragraph{Example 2}

In this example an \gls{exposure model} containing the number of units
(buildings) and the associated costs per unit of each building typology is
presented.

\inputminted[firstline=8,firstnumber=8,lastline=18,fontsize=\footnotesize,frame=single,linenos,bgcolor=lightgray]{xml}{oqum/risk/Verbatim/input_exposure_cunit.xml}\\

For this case, the cost \Verb+type+ has been set to \Verb+per_asset+. Then, the
information from each asset can be stored following the format below:

\inputminted[firstline=19,firstnumber=19,lastline=29,fontsize=\footnotesize,frame=single,linenos,bgcolor=lightgray]{xml}{oqum/risk/Verbatim/input_exposure_cunit.xml}\\

In this example, the various costs for each asset is not provided directly, as
happened in the previous example. In order to carry out the risk calculations
in which the economic cost of each asset is required, the OpenQuake-engine
multiplies, for each asset, the number of units (buildings) by the ``per
asset'' replacement cost. Note that in this case, there is no need to specify
the attribute \Verb+area+.


\paragraph{Example 3}

This example is comprised of an \gls{exposure model} containing the built up
area of each building typology for a set of locations, and the associated
costs are provided per unit area.

\inputminted[firstline=8,firstnumber=8,lastline=20,fontsize=\footnotesize,frame=single,linenos,bgcolor=lightgray]{xml}{oqum/risk/Verbatim/input_exposure_carea_aagg.xml}\\

In order to compile an \gls{exposure model} with this structure, it is
required to set the cost \Verb+type+ to \Verb+per_area+. In addition, it is
also necessary to specify if the \Verb+area+ that is being store represents
the aggregated area of number of units within an asset, or the average area of
a single unit. In this particular case, the \Verb+area+ that is being stored
is the aggregated built up area per asset, and thus this attribute was set to
\Verb+aggregated+.

\inputminted[firstline=21,firstnumber=21,lastline=31,fontsize=\footnotesize,frame=single,linenos,bgcolor=lightgray]{xml}{oqum/risk/Verbatim/input_exposure_carea_aagg.xml}\\

Once again, the OpenQuake-engine needs to carry out some calculations in order
to compute the different costs per asset. In this case, this value is computed
by multiplying the aggregated built up \Verb+area+ of each building typology
by the associated cost per unit of area. Notice that in this case, there is no
need to specify the attribute \Verb+number+.


\paragraph{Example 4}

This example is comprised of an \gls{exposure model} containing the number of
buildings for each location, the average built up area per building unit and
the associated costs per unit area.

\inputminted[firstline=8,firstnumber=8,lastline=20,fontsize=\footnotesize,frame=single,linenos,bgcolor=lightgray]{xml}{oqum/risk/Verbatim/input_exposure_carea_aunit.xml}\\

Similarly to what was described in the previous example, the various costs
\Verb+type+ also need to be establish as \Verb+per_area+, but the \Verb+type+
of area is now defined as \Verb+per_asset+.

\inputminted[firstline=21,firstnumber=21,lastline=31,fontsize=\footnotesize,frame=single,linenos,bgcolor=lightgray]{xml}{oqum/risk/Verbatim/input_exposure_carea_aunit.xml}\\

In this example, the OpenQuake-engine will make use of all the parameters to
estimate the various costs of each asset, by multiplying the number of
buildings by its average built up area, and then by the respective cost per
unit of area.


\paragraph{Example 5}

In this example, additional information will be included, which is required
for other risk analysis besides loss estimation, such as the calculation of
insured losses or benefit/cost analysis. For the former assessment, it is
necessary to establish how the insured limit and deductible is going to be
define, according to the format below.

\inputminted[firstline=8,firstnumber=8,lastline=21,fontsize=\footnotesize,frame=single,linenos,bgcolor=lightgray]{xml}{oqum/risk/Verbatim/input_exposure_ins_rel.xml}\\

In this example, both the insurance limit and the deductible were defined as a
fraction of the costs, by setting the attribute \Verb+isAbsolute+ to
\Verb+false+. Then, for each type of cost, the limit and deductible value
can be stored for each asset.

\inputminted[firstline=22,firstnumber=22,lastline=32,fontsize=\footnotesize,frame=single,linenos,bgcolor=lightgray]{xml}{oqum/risk/Verbatim/input_exposure_ins_rel.xml}\\

On the other hand, a user could define one or both of these parameters as the
absolute threshold, by setting the aforementioned attribute to \Verb+true+.
This is shown in the example file below:

\inputminted[firstline=1,firstnumber=1,fontsize=\footnotesize,frame=single,linenos,bgcolor=lightgray]{xml}{oqum/risk/Verbatim/input_exposure_ins_abs.xml}\\

Moreover, in order to perform a benefit/cost
assessment, it is also fundamental to indicate the retrofitting cost. This
parameter is handled in the same manner as the structural cost, and it should
be stored according to the following structure:

\inputminted[firstline=1,firstnumber=1,fontsize=\footnotesize,frame=single,linenos,bgcolor=lightgray]{xml}{oqum/risk/Verbatim/input_exposure_retrofit.xml}\\

Despite the fact that for the demonstration of how the insurance parameters
and retrofitting cost can be stored the per building type of cost structure
described in Example 1 was used, it is important to mention that any of the
other cost storing approaches can also be employed (Examples 2--4).


\paragraph{Example 6}

The OpenQuake-engine is also capable of estimating human losses, based on a
number of occupants within an asset, at a certain time of the day. This
example describes how this parameter is defined for each asset. In addition,
this example also serves the purpose of presenting an \gls{exposure model} in
which three cost types have been defined following different structures.

As previously mentioned, in this example only three costs are being stored,
and each one follows a different approach. The \Verb+structural+ cost is being
defined as the aggregate replacement cost for all of the buildings comprising
the asset (Example 1), the \Verb+nonstructural+ value is defined as the
replacement cost per unit area where the area is defined per building
comprising the asset (Example 4), and the \Verb+contents+ and
\Verb+business_interruption+ values are provided per building comprising the
asset (Example 2). The  occupants at different times of the day are also
provided as aggregated values for all of the buildings comprising the asset.

\inputminted[firstline=1,firstnumber=1,fontsize=\footnotesize,frame=single,linenos,bgcolor=lightgray]{xml}{oqum/risk/Verbatim/input_exposure_occupants.xml}


Scripts to convert an \gls{exposure model} in CSV format or as Excel or
ASCII files into NRML are also under development, and can be found at the
OpenQuake platform at the following address:
\href{https://platform.openquake.org/risk_input_preparation_toolkit/}{https://platform.openquake.org/risk\_input\_preparation\_toolkit/}.