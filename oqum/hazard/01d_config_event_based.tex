In the following we describe the sections of the configuration file that are
required to complete event based PSHA calculations


\textbf{Calculation type and model info}

This part is almost identical to the corresponding one described in
Section~\ref{subsec:config_classical_psha}. Note the setting of the
\texttt{calculation\_mode} parameter which now corresponds to
\texttt{event\_based}.

\begin{Verbatim}[frame=single, commandchars=\\\{\}, fontsize=\small,
    numbers=left, numbersep=2pt]
[general]
description = A demo OpenQuake-engine .ini file for classical PSHA
calculation_mode = event_based
random_seed = 1024
\end{Verbatim}



\textbf{Event based parameters}

This is section is used to specify the number of stochastic event sets to be
generated for each logic tree realisation  (each stochastic event set
represents a potential realisation of seismicity during the
\texttt{investigation\_time} specified in the
\texttt{calculation\_configuration} part). Additionally, in this section the
user can specify the spatial correlation model to be used in case for the
generation of ground motion fields.

\begin{Verbatim}[frame=single, commandchars=\\\{\}, fontsize=\small]
[event_based_params]
ses_per_logic_tree_path = 5
ground_motion_correlation_model = JB2009
ground_motion_correlation_params = {"vs30_clustering": True}
\end{Verbatim}

The acceptable flags for the parameter \verb+vs30_clustering+ are \verb+False+
and \verb+True+, with a capital \verb+F+ and \verb+T+ respectively. \verb+0+
and \verb+1+ are also acceptable flags.



\textbf{Output}

This part substitutes the \texttt{Output} part described in  the configuration
file example described in the Section~ \ref{subsec:config_classical_psha} at
page~\pageref{subsec:config_classical_psha}.

\begin{Verbatim}[frame=single, commandchars=\\\{\}, fontsize=\small]
[output]
export_dir = /tmp/xxx
ground_motion_fields = true
# post-process ground motion fields into hazard curves,
# given the specified `intensity_measure_types_and_levels`
hazard_curves_from_gmfs = true
mean_hazard_curves = true
quantile_hazard_curves = 0.15, 0.5, 0.85
poes = 0.1, 0.2
\end{Verbatim}