In this Chapter we provide a desciption of the main commands available for
running hazard with the \gls{acr:oqe} and the file formats used to represent
the results of the analyses.

A general introduction on the use of OpenQuake-engine is provided in
Section~\ref{sec:running_oq_engine} at page~\pageref{sec:running_oq_engine}. The
reader is invited to consult this part before diving into the following
sections.


% -----------------------------------------------------------------------------
\section{Running OpenQuake-engine for hazard calculations}
\label{sec:running_hazard_calculations}
\index{Running OpenQuake!hazard}

The execution of a hazard analysis using the OpenQuake-engine is
straightforward. Below we provide an example of the simplest command that can be
used to launch a hazard calculation. It consists in the invocation of \texttt
{oq-engine} together with the \texttt{--rh} option which stands for ``run
hazard'' and the name of a configuration file (in the example below it
corresponds to \texttt{job.ini}):

\begin{Verbatim}[frame=single, commandchars=\\\{\}, fontsize=\small]
user@ubuntu:~$ oq-engine --rh job.ini
\end{Verbatim}

The amount of information prompted during the execution of the analysis can be
controlled through the \texttt{--log-level} flag as shown in the example below:

\begin{Verbatim}[frame=single, commandchars=\\\{\}, fontsize=\small]
user@ubuntu:~$ oq-engine --rh job.ini --log-level debug
\end{Verbatim}

In this example we ask the engine to provide an extensive amount of information
(usually not justified for a standard analysis). Alternative options are:
\texttt{debug}, \texttt{info}, \texttt{progress}, \texttt{warn}, \texttt{error},
\texttt{critical}.


% -----------------------------------------------------------------------------
\section{Exporting results from a hazard calculation}
\label{sec:exporting_hazard_results}

There are two alternative ways to get results from the OpenQuake-engine:
directly through the calculation or by exporting them from the internal
\gls{acr:oqe} database once a calculation is completed.

The first option is defined at the OpenQuake-engine invocation through the
flag \texttt{--exports xml}, as shown in the example below:

\begin{Verbatim}[frame=single, commandchars=\\\{\}, fontsize=\small]
user@ubuntu:~$ oq-engine --rh job.ini \textcolor{red}{--exports xml}
\end{Verbatim}

The second option allows the user to export the computed results or just a
subset of them whenever they want. In order to obtain the list of results of
the hazard calculations stored in the \gls{acr:oqe} database the user can
utilize the following command:

\begin{Verbatim}[frame=single, commandchars=\\\{\}, fontsize=\small]
user@ubuntu:~$ oq-engine --lhc
\end{Verbatim}

The execution of this command will produce a list similar to the one provided
below (the numbers in red are the calculations IDs):

\begin{Verbatim}[frame=single, commandchars=\\\{\}, fontsize=\small]
user@ubuntu:~$ oq engine --lhc
job_id | status | start_time | description
\textcolor{red}{1} | failed | 2013-03-01 09:49:34 | Classical PSHA
\textcolor{red}{2} | successful | 2013-03-01 09:49:56 | Classical PSHA
\textcolor{red}{3} | failed | 2013-03-01 10:24:04 | Classical PSHA
\textcolor{red}{4} | failed | 2013-03-01 10:28:16 | Classical PSHA
\textcolor{red}{5} | failed | 2013-03-01 10:30:04 | Classical PSHA
\textcolor{red}{6} | successful | 2013-03-01 10:31:53 | Classical PSHA
\textcolor{red}{7} | failed | 2013-03-09 08:15:14 | Classical PSHA
\textcolor{red}{8} | successful | 2013-03-09 08:18:04 | Classical PSHA
\end{Verbatim}

Subsequently the user can get the list of result stored for a specific hazard
analysis as in the example below (note that the number in blue emphasizes the
result ID):

\begin{Verbatim}[frame=single, commandchars=\\\{\}, fontsize=\small]
user@ubuntu:~$ oq-engine --lho <calc_id>
id | output_type | name
\textcolor{blue}{3} | hazard_curve | hc-rlz-6
\end{Verbatim}

and finally extract an xml file for a specific hazard result:

\begin{Verbatim}[frame=single, commandchars=\\\{\}, fontsize=\small]
user@ubuntu:~$ oq-engine --eh <result_id> <path_to_the_output_folder>
\end{Verbatim}


% -----------------------------------------------------------------------------
\section{Description of hazard outputs}
\label{sec:hazard_outputs}

The results generated by the OpenQuake-engine are fundamentally of two
distinct typologies differentiated by the presence (or absence) of epistemic
uncertainty in the PSHA input model.

When epistemic uncertainty is incorporated into the calculation, the
OpenQuake-engine calculators (e.g. Classical PSHA, Event Based PSHA,
Disaggregation, UHS) produce a set of results (i.e. hazard curves, ground
motion fields, disaggregation matrices, UHS, for each logic-tree realisation)
which reflects epistemic uncertainties introduced in the PSHA input model.

For each logic tree sample, results are computed and stored. Calculation of
results statistics (mean, standard deviation, quantiles) are supported by all
the calculators, with the exception of the disaggregation calculator.

\subsection{Outputs from Classical PSHA}
\label{subsec:output_classical_psha}
By default, the classical PSHA calculator computes and stores hazard curves
for each logic tree sample considered.

When the PSHA input model doesn't contain epistemic uncertainties the results
is a set of hazard curves (one for each investigated site). The command below
illustrates how is possible to retrieve the group of hazard curves obtained
for a calculation with a given identifier \texttt{<calc\_id>} (see
Section~\ref{sec:exporting_hazard_results} for an explanation about how to
obtain the list of calculations performed with their corresponding ID):

\begin{Verbatim}[frame=single, commandchars=\\\{\}, fontsize=\small]
user@ubuntu:~$ oq-engine --lo <calc_id>
id | output_type | name
\textcolor{red}{3 | hazard_curve | hc-rlz-6}
\end{Verbatim}

In this case the \gls{acr:oqe} computed a group of hazard curves with result
ID equal to \texttt{3}. On the contrary, if the parameter
\texttt{number\_of\_logic\_tree\_samples} in the configuration file is
different than zero, then N hazard curves files are generated. The example
below shows this case:

\begin{Verbatim}[frame=single, commandchars=\\\{\}, fontsize=\small]
user@ubuntu:~$ oq engine --lo <calc_id>
id | output_type | name
\textcolor{red}{5 | datastore | hcurves}
\textcolor{red}{6 | datastore | hcurves}
\textcolor{red}{7 | datastore | hcurves}
\textcolor{red}{8 | datastore | hcurves}
\textcolor{red}{9 | datastore | hcurves}
\textcolor{red}{10 | datastore | hcurves}
\end{Verbatim}

If we export from the database the hazard curves contained in one of the
items above using the following command:

\begin{Verbatim}[frame=single, commandchars=\\\{\}, fontsize=\small]
user@ubuntu:~$ oq-engine --export-outputs <output_id> <output_directory>
\end{Verbatim}

we obtain a nrml formatted file as represented in the example in the inset
below:

\begin{Verbatim}[frame=single, commandchars=\\\{\}, fontsize=\small]
<?xml version='1.0' encoding='UTF-8'?>
<nrml xmlns:gml="http://www.opengis.net/gml"
      xmlns="http://openquake.org/xmlns/nrml/0.5">
  <hazardCurves \textcolor{red}{sourceModelTreePath="b1|b212"}
      \textcolor{red}{gsimTreePath="b2" IMT="PGA" investigationTime="50.0"}>
    \textcolor{green}{<IMLs>0.005 0.007 0.0098 ... 1.09 1.52 2.13</IMLs>}
    <hazardCurve>
      <gml:Point>
      \textcolor{blue}{<gml:pos>10.0 45.0</gml:pos>}
      </gml:Point>
      <poEs>1.0 1.0 1.0 ... 0.000688359310522 0.0 0.0</poEs>
    </hazardCurve>
    ...
    <hazardCurve>
      <gml:Point>
      \textcolor{blue}{<gml:pos>lon lat</gml:pos>}
      </gml:Point>
      <poEs>poe1 poe2 ... poeN</poEs>
    </hazardCurve>
  </hazardCurves>
</nrml>
\end{Verbatim}

Notwithstanding the intuitiveness of this file, let's have a brief
overview of the information included.

The overall content of this file is a list of hazard curves, one for each
investigated site, computed using a PSHA input model representing one possible
realisation obtained using the complete logic tree structure.

The attributes of the \texttt{hazardCurves} element (see text in red) specify
the path of the logic tree used to create the seismic source model
(\texttt{source\-Model\-TreePath}) and the ground motion model
(\texttt{gsim\-Tree\-Path}) plus the intensity measure type and the
investigation time used to compute the probability of exceedance.

The \texttt{IMLs} element (in green in the example) contains the values of
shaking used by the engine to compute the probability of exceedance in the
investigation time. For each site this file contains a \texttt{hazardCurve}
element which has the coordinates (longitude and latitude in decimal degrees)
of the site and the values of the probability of exceedance for all the
intensity measure levels specified in the \texttt{IMLs} element.

If in the configuration file the calculation of mean hazard curves and hazard
curves corresponding to one or several percentiles have been specified, the
list of outputs that we should expect from the \glsdesc{acr:oqe} corresponds
to:

\begin{Verbatim}[frame=single, commandchars=\\\{\}, fontsize=\small]
user@ubuntu:~$ oq-engine --lho <calc_id>
id | output_type | name
17 | hazard_curve | hc-rlz-17
18 | hazard_curve | hc-rlz-18
19 | hazard_curve | hc-rlz-13
20 | hazard_curve | hc-rlz-14
21 | hazard_curve | hc-rlz-15
22 | hazard_curve | hc-rlz-16
\textcolor{red}{23 | hazard_curve | quantile(0.5)-curves-PGA}
24 | hazard_map | hazard-map(0.1)-PGA-rlz-17
25 | hazard_map | hazard-map(0.1)-PGA-rlz-18
26 | hazard_map | hazard-map(0.1)-PGA-rlz-13
27 | hazard_map | hazard-map(0.1)-PGA-rlz-14
28 | hazard_map | hazard-map(0.1)-PGA-rlz-15
29 | hazard_map | hazard-map(0.1)-PGA-rlz-16
\textcolor{red}{30 | hazard_map | hazard-map(0.1)-PGA-quantile(0.5)}
\end{Verbatim}

In this example the \gls{acr:oqe} produced hazard curves and hazard maps for
six logic tree realisations plus median hazard curves and the median hazard
map (both highlighted in red).

The following inset shows a sample of the nrml file used to describe a hazard
map:

\begin{Verbatim}[frame=single, commandchars=\\\{\}, fontsize=\small]
<?xml version='1.0' encoding='UTF-8'?>
<nrml xmlns:gml="http://www.opengis.net/gml"
      xmlns="http://openquake.org/xmlns/nrml/0.4">
  \textcolor{red}{<hazardMap sourceModelTreePath="b1" gsimTreePath="b1"}
        \textcolor{red}{IMT="PGA" investigationTime="50.0" poE="0.1">}
    <node lon="119.596690957" lat="21.5497682591" iml="0.204569990197"/>
    <node lon="119.596751048" lat="21.6397004197" iml="0.212391638188"/>
    <node lon="119.596811453" lat="21.7296325803" iml="0.221407505615"/>
    ...
  </hazardMap>
</nrml>
\end{Verbatim}

\subsection{Outputs from Hazard Disaggregation}
\label{subsec:output_hazard_disaggregation}
\section{Output from Disaggregation}
The output from an \gls{acr:oqe} disaggregation analysis  
correspond to the combination of a hazard curve and a multidimensional 
matrix containing the results of the disaggregation.

The example below shows the list of disaggregation results obtained 
for four logic tree realisations. 
For each realisation, disaggregation has been completed for two  
intensity measure levels corresponding to different probabilities of 
exceedence in the specified \texttt{investigation time}.
\begin{Verbatim}[frame=single, commandchars=\\\{\}, samepage=true]
user@ubuntu:~$ openquake --lho <calc_id> 
id | output_type | name
19 | hazard_curve | hc-rlz-3
20 | hazard_curve | hc-rlz-3
21 | hazard_curve | hc-rlz-4
22 | hazard_curve | hc-rlz-4
23 | disagg_matrix | disagg(0.02)-rlz-3-SA(0.025)-POINT(10.1 40.1)
24 | disagg_matrix | disagg(0.1)-rlz-3-SA(0.025)-POINT(10.1 40.1)
25 | disagg_matrix | disagg(0.02)-rlz-3-PGA-POINT(10.1 40.1)
26 | disagg_matrix | disagg(0.1)-rlz-3-PGA-POINT(10.1 40.1)
27 | disagg_matrix | disagg(0.02)-rlz-4-SA(0.025)-POINT(10.1 40.1)
28 | disagg_matrix | disagg(0.1)-rlz-4-SA(0.025)-POINT(10.1 40.1)
29 | disagg_matrix | disagg(0.02)-rlz-4-PGA-POINT(10.1 40.1)
30 | disagg_matrix | disagg(0.1)-rlz-4-PGA-POINT(10.1 40.1)
\end{Verbatim}

In the following (see Inset \ref{vrb:disaggr}) we show 
%\begin{nrmlsmp}
\begin{Verbatim}[frame=single, commandchars=\\\{\}, fontsize=\small]
<?xml version='2.0' encoding='UTF-8'?>
<nrml xmlns:gml="http://www.opengis.net/gml" 
      xmlns="http://openquake.org/xmlns/nrml/0.4">
  <disaggMatrices sourceModelTreePath="b1" gsimTreePath="b1" IMT="PGA" 
        investigationTime="50.0" lon="10.1" lat="40.1" 
        magBinEdges="5.0, 6.0, 7.0, 8.0" 
        distBinEdges="0.0, 25.0, 50.0, 75.0, 100.0" 
        lonBinEdges="9.0, 10.5, 12.0" 
        latBinEdges="39.0, 40.5" 
        epsBinEdges="-3.0, -1.0, 1.0, 3.0" 
        tectonicRegionTypes="Active Shallow Crust">
    <disaggMatrix type="Mag" dims="3" poE="0.1" iml="0.033424622602">
    <disaggMatrix type="Mag" dims="3" poE="0.1" iml="0.033424622602">
      <prob index="0" value="0.987374744394"/>
      <prob index="1" value="0.704295394366"/>
      <prob index="2" value="0.0802318409498"/>
    </disaggMatrix>
    <disaggMatrix type="Dist" dims="4" poE="0.1" iml="0.033424622602">
      <prob index="0" value="0.700851969171"/>
      <prob index="1" value="0.936680387051"/>
      <prob index="2" value="0.761883595568"/>
      <prob index="3" value="0.238687565571"/>
    </disaggMatrix>
    <disaggMatrix type="TRT" dims="1" poE="0.1" iml="0.033424622602">
      <prob index="0" value="0.996566187011"/>
    </disaggMatrix>
    <disaggMatrix type="Mag,Dist" dims="3,4" poE="0.1" iml="0.033424622602">
      <prob index="2,3" value="0.0"/>
    </disaggMatrix>
    <disaggMatrix type="Mag,Dist,Eps" dims="3,4,3" poE="0.1" iml="0.033424622602">
      <prob index="0,0,0" value="0.0785857271425"/>
\end{Verbatim}
%\caption{Hazard map: nrml sample file}
%\label{vrb:disaggr}
%\end{nrmlsmp}



\subsection{Outputs from Event Based PSHA}
\label{subsec:output_event_based_psha}
\section{Output from Event Based PSHA}\label{EventBasedOutput}
%
The Event Based PSHA calculator computes and stores stochastic 
event sets and the corresponding ground motion fields. In case,
this calculator can also compute hazard curves and hazard maps
exactly in the same way is done using the Classical PSHA calculator.

The inset below shows an example of the list of results provided by 
\gls{acr:oqe} at the end of an event-based PSHA calculation:
%
\begin{Verbatim}[frame=single, commandchars=\\\{\}, fontsize=\small]
user@ubuntu:~$ oq-engine --lho <calc_id> 
id | output_type | name
\textcolor{red}{31 | ses | ses-coll-rlz-19}
\textcolor{green}{32 | gmf | gmf-rlz-19}
\textcolor{red}{33 | ses | ses-coll-rlz-20}
\textcolor{green}{34 | gmf | gmf-rlz-20}
35 | hazard_curve | hazard-curve-rlz-19-SA(0.1)
36 | hazard_curve | hazard-curve-rlz-20-SA(0.1)
37 | hazard_curve | hazard-curve-rlz-19-PGA
38 | hazard_curve | hazard-curve-rlz-20-PGA
39 | hazard_curve | mean curve for SA(0.1)
40 | hazard_curve | quantile curve (poe >= 0.15) for imt SA(0.1)
41 | hazard_curve | quantile curve (poe >= 0.5) for imt SA(0.1)
42 | hazard_curve | quantile curve (poe >= 0.85) for imt SA(0.1)
43 | hazard_curve | mean curve for PGA
44 | hazard_curve | quantile curve (poe >= 0.15) for imt PGA
45 | hazard_curve | quantile curve (poe >= 0.5) for imt PGA
46 | hazard_curve | quantile curve (poe >= 0.85) for imt PGA
\end{Verbatim}
This list contains two sets of stochastic events (in red) two sets of ground motion fields (in blue).

The whole group of stochastic event set and ground motion fields can 
be exported immediately using the results with \texttt{id} 35 and 25,
respectively.

In the remaining part of this Section we show an example of a file
containing a stochastic event set and a file containing a ground 
motion field.

This is an example showing a nrml file containing two a collection of 
stochastic event sets 
\begin{Verbatim}[frame=single, commandchars=\\\{\}, fontsize=\small]
<?xml version='1.0' encoding='UTF-8'?>
<nrml xmlns:gml="http://www.opengis.net/gml" 
	  xmlns="http://openquake.org/xmlns/nrml/0.4">
  <stochasticEventSetCollection sourceModelTreePath="b1" 
          gsimTreePath="b1">
    \textcolor{red}{<stochasticEventSet id="12" investigationTime="50.0">}
      <rupture id="533" magnitude="4.55" strike="90.0" dip="90.0" 
              rake="90.0" tectonicRegion="Active Shallow Crust">
        <planarSurface>
          <topLeft lon="12.233903801" lat="43.256198599" 
                  depth="11.3933265259"/>
          <topRight lon="12.263958243" lat="43.2562025344" 
                  depth="11.3933265259"/>
          <bottomLeft lon="12.233903801" lat="43.256198599" 
                  depth="12.6066734741"/>
          <bottomRight lon="12.263958243" lat="43.2562025344" 
                  depth="12.6066734741"/>
        </planarSurface>
      </rupture>
      <rupture id="535" magnitude="4.65" strike="135.0" dip="90.0" 
              rake="90.0" tectonicRegion="Active Shallow Crust">
        <planarSurface>
          <topLeft lon="11.45858812" lat="42.7429056814" 
                  depth="11.3208667302"/>
          <topRight lon="11.4822820715" lat="42.7256333907" 
                  depth="11.3208667302"/>
          <bottomLeft lon="11.45858812" lat="42.7429056814" 
                  depth="12.6791332698"/>
          <bottomRight lon="11.4822820715" lat="42.7256333907" 
                  depth="12.6791332698"/>
        </planarSurface>
      </rupture>
    \textcolor{red}{</stochasticEventSet>}
  </stochasticEventSetCollection>
</nrml>
\end{Verbatim}
This is an example showing a nrml file containing a collection of
ground motion fields:
\begin{Verbatim}[frame=single, commandchars=\\\{\}, fontsize=\small]
<?xml version='1.0' encoding='UTF-8'?>
<nrml xmlns:gml="http://www.opengis.net/gml" 
      xmlns="http://openquake.org/xmlns/nrml/0.4">
  <gmfCollection sourceModelTreePath="b1" gsimTreePath="b1">
    <gmfSet investigationTime="50.0" stochasticEventSetId="12">
      <gmf IMT="PGA" ruptureId="533">
        <node gmv="0.0105891230432" lon="11.1240023202" 
            lat="43.5107462335"/>
        <node gmv="0.00905803920023" lon="11.1241875202" 
            lat="43.6006783941"/>
        <node gmv="0.00637664420977" lon="11.124373581" 
            lat="43.6906105547"/>
        <node gmv="0.00476533134789" lon="11.1245605075" 
            lat="43.7805427153"/>
        <node gmv="0.00452594698469" lon="11.1247483046" 
            lat="43.8704748759"/>
        ...
        <node gmv="0.000173010769646" lon="11.3782630185" 
            lat="44.5"/>
      </gmf>
    </gmfSet>
  </gmfCollection>
</nrml>
\end{Verbatim}

