% -----------------------------------------------------------------------------
\section{Calculation workflows}
\index{OpenQuake-engine!Hazard calculation workflows} 
% Three types of analysis
The hazard component of the OpenQuake-engine can compute seismic hazard 
using various approaches. 
%
Three types of analysis are currently supported:
\begin{itemize}
\item \textit{Classical Probabilistic Seismic Hazard Analysis (PSHA)}, 
allowing calculation of hazard curves and hazard maps following the 
classical integration procedure 
(\cite{cornell1968}, \citet{mcguire1976}) as formulated by \cite{field2003}.
%
\item \textit{Event-Based Probabilistic Seismic Hazard Analysis}, 
    allowing calculation of ground-motion fields from 
    stochastic event sets. Traditional results - 
    such as hazard curves - can be obtained by post-processing the 
    set of computed ground-motion fields.
\item \textit{\gls{acr:ssha}}, allowing the calculation of 
    ground motion fields from a single earthquake rupture scenario 
    taking into account ground-motion aleatory variability.
\end{itemize}
%
Each workflow has a modular structure, so that intermediate results 
can be exported and analyzed. 
Each calculator can be extended independently of the others so that 
additional calculation options and methodologies can be easily 
introduced, without affecting the overall calculation workflow. 
%
%  - - - - - - - - - - - - - - - - - - - - - - - - - - - - - - - - - - - - - - -
\subsection{Classical Probabilistic Seismic Hazard Analysis}
\index{OpenQuake-engine!Hazard calculation workflows!Classical PSHA} 
\label{section:classicalPSHA}
%
Input data for the classical \gls{acr:psha} consist of a PSHA input model
provided together with calculation settings. 

The main calculators used to perform this analysis are the following:
\begin{enumerate}
\item \emph{Logic Tree Processor} \hfill \\
The Logic Tree Processor (LTP) takes as an input the \gls{acr:psha} 
Input Model and creates a Seismic Source Model. The LTP uses the 
information in the Initial Seismic Source Models and  
the Seismic Source Logic Tree to create a Seismic Source Input
Model (i.e. a model describing geometry and activity rates of each 
source without any epistemic uncertainty). 
%
Following a procedure similar to the one just described the Logic Tree 
Processor creates a Ground Motion model (i.e. a data structure that 
associates to each tectonic region considered in the calculation a 
\gls{acr:gmpe}).
%
\item \emph{Earthquake Rupture Forecast Calculator} \hfill \\
The produced Seismic Source Input Model becomes an input information for 
the Earthquake Rupture Forecast (ERF) calculator which creates a list 
earthquake ruptures admitted by the source model, each one characterized
by a probability of occurrence over a specified time span.
\item \emph{Classical PSHA Calculator} \hfill \\
The classical PSHA calculator uses the ERF and the Ground Motion model 
to compute hazard curves on each site specified in the calculation settings.
\end{enumerate} 
%
%  - - - - - - - - - - - - - - - - - - - - - - - - - - - - - - - - - - - - - - -
\subsection{Event-Based Probabilistic Seismic Hazard Analysis}
\index{OpenQuake-engine!Hazard calculation workflows!Event-based PSHA} 
\label{section:event-basedPSHA}
Input data for the Event-Based PSHA - as in the case of the 
Classical \gls{acr:psha} calculator - consist of a PSHA Input Model 
and calculation settings.

The main calculators  used to perform this analysis are:
\begin{enumerate}
%
\item \emph{Logic Tree Processor} \hfill \\
The Logic Tree Processor works in the same way described in 
the description of the Classical \gls{acr:psha} workflow 
(see section \ref{section:classicalPSHA} at page 
\pageref{section:classicalPSHA}).
%
\item \emph{Earthquake Rupture Forecast Calculator} \hfill \\ 
The Earthquake Rupture Forecast Calculator was already 
introduced in the description of the PSHA workflow (see section 
\ref{section:classicalPSHA} at page \pageref{section:classicalPSHA}).
%
\item \emph{Stochastic Event Set Calculator} \hfill \\
The Stochastic Event Set Calculator generates a collection of Stochastic 
event sets by sampling the ruptures contained in the ERF according to their 
probability of occurrence. 
%
A Stochastic Event Set (SES) thus represents a potential realisation of the 
seismicity (i.e. a list of ruptures) produced by the set of seismic sources 
considered in the analysis over the time span fixed for the 
calculation of hazard. 
%
\item \emph{Ground Motion Field Calculator} \hfill \\
The Ground Motion Field Calculator computes for each event contained in a 
Stochastic Event Set a realization of the geographic distribution of the 
shaking by taking into account the aleatory uncertainties in 
the ground-motion model. Eventually, the Ground Motion Field calculator 
can consider the spatial correlation of the ground-motion during the 
generation of the \gls{acr:gmf}.
%
\item \emph{Event-based PSHA Calculator} \hfill \\
The event-based PSHA calculator takes a (large) set of ground-motion 
fields representative of the possible shaking scenarios that the investigated
area can experience over a (long) time span and for each 
site computes the corresponding hazard curve. 
%
This procedure is computationally intensive and is not recommended for 
investigating the hazard over large areas. 
\end{enumerate}
%
%  - - - - - - - - - - - - - - - - - - - - - - - - - - - - - - - - - - - - - - -
\subsection{Scenario based Seismic Hazard Analysis}
\index{OpenQuake-engine!Hazard calculation workflows!Scenario-based SHA} 
\label{section:deterministicSHA}
In case of \gls{acr:ssha}, the input data consist of a single earthquake 
rupture model and a single ground-motion model. Using the Ground Motion Field 
Calculator, multiple realizations of ground shaking can be computed, each 
realization sampling the aleatory uncertainties in the ground-motion model.

The main calculators used to perform this analysis are:
\begin{enumerate}
\item \emph{Ground Motion Field Calculator} \hfill \\
The Ground Motion Field Calculator was already 
introduced during the description of the event based PSHA workflow (see 
section \ref{section:event-basedPSHA} at page \pageref{section:classicalPSHA}).
\end{enumerate}
% ..............................................................................
% . . . . . . . . . . . . . . . . . . . . . . . . . . . . . . . . . . . > Figure
% \begin{figure}[!hb]
% \centering
% \includegraphics[width=14cm]{./figures/deterministic_workflow.eps}
% \caption{Workflow for deterministic SHA. Given a rupture scenario model, 
% consisting of an earthquake rupture model, plus a GMPE, the ground-motion 
% field calculator can compute multiple ground-motion field realizations (by 
% taking into account GMPE aleatory uncertainties).}
% \label{deterministic_workflow}
% \end{figure}
% ..............................................................................
\cleardoublepage
