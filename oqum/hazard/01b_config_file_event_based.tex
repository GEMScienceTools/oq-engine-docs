\subsubsection{Event based PSHA}
%
In the following we describe the sections of the logic tree that are 
specific for event based PSHA calculations (the remaining sections 
\begin{enumerate}
\item \textbf{Calculation type and model info}
    This part is almost identical to the corresponding one 
    describe in section \ref{sec:config_classical_PSHA}. Note
    the setting of the \texttt{calculation\_mode} parameter.
\begin{Verbatim}[frame=single, commandchars=\\\{\}, fontsize=\small,
    numbers=left, numbersep=2pt]
[general]
description = A demo OpenQuake-engine .ini file for classical PSHA
calculation_mode = event_based
random_seed = 1024
\end{Verbatim}
%
\item \textbf{Event based} \hfill \\
This is an additional part used to specify the number of stochastic 
event sets (each one representing a potential realisation of seismicity
during the \texttt{investigation\_time} specified in the 
\texttt{calculation\_configuration} part.
\begin{Verbatim}[frame=single, commandchars=\\\{\}, fontsize=\small]
[event_based_params]
ses_per_logic_tree_path = 5
ground_motion_correlation_model = JB2009
ground_motion_correlation_params = {"vs30_clustering": true}
\end{Verbatim}
%
\item \textbf{Output} \hfill \\
This part substitutes the \texttt{Output} part described in 
configuration file example descibed in section \ref{sec:config_classical_PSHA}
at page \pageref{sec:config_classical_PSHA}.
\begin{Verbatim}[frame=single, commandchars=\\\{\}, fontsize=\small]
[output]
export_dir = /tmp/xxx
complete_logic_tree_ses = true
complete_logic_tree_gmf = true
ground_motion_fields = true
# post-process ground motion fields into hazard curves,
# given the specified `intensity_measure_types_and_levels`
hazard_curves_from_gmfs = true
mean_hazard_curves = true
quantile_hazard_curves = 0.15, 0.5, 0.85
poes_hazard_maps = 0.1, 0.2
\end{Verbatim}
%
\end{enumerate}
