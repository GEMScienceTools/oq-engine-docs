\subsubsection{Calculation of a hazard map and hazard curves using 
    the classical PSHA methodology}
\label{sec:config_classical_PSHA}
%
In the following we describe the overall structure and the
most typical parameters of a configuration file to be used for the 
computation of a seismic hazard map using a classical PSHA methodology.
\begin{itemize}
\item \textbf{Calculation type and model info}
\begin{Verbatim}[frame=single, commandchars=\\\{\}, fontsize=\small,
    numbers=left, numbersep=2pt]
[general]
description = A demo OpenQuake-engine .ini file for classical PSHA
calculation_mode = classical
random_seed = 1024
\end{Verbatim}
In this section the user specifies the following parameters:
\begin{itemize}
    \item \texttt{description}: a parameter that can be used to designate 
        the model 
    \item \texttt{calculation\_mode}: it is used to set the kind 
        of calculation. In this case it corresponds to \texttt{classical}.
        We'll describe alternative options later on.
    \item \texttt{random\_seed}: is used to control the random generator 
        so that when montecarlo procedures are used calculations are 
        replicable (if the same \texttt{random\_seed} is used you get 
        exactly the same results).
\end{itemize}
%
\item \textbf{Geometry of the area (or the sites) where hazard is computed}
    \hfill \\
This section is used to specify where hazard will be computed. Two 
option are available. 

The first one consists on defining a polygon 
(usually a rectangle) and a distance (in km) used to discretize the 
polygon area. The polygon is defined by a list of longitude-latitude tuples.

An example is provided below.
\begin{Verbatim}[frame=single, commandchars=\\\{\}, fontsize=\small,
    firstnumber=5, numbers=left, numbersep=2pt]
[geometry]
region = 10.0 43.0, 12.0 43.0, 12.0 46.0, 10.0 46.0
\# km
region_grid_spacing = 10.0
\end{Verbatim}

The second option allows the definition of a number of sites where 
the hazard will be computed. An example is provided below.
\begin{Verbatim}[frame=single, commandchars=\\\{\}, fontsize=\small,
    firstnumber=5, numbers=left, numbersep=2pt]
[geometry]
sites = 10.0 43.0, 12.0 43.0, 12.0 46.0, 10.0 46.0


\end{Verbatim}
If the list of sites is too long the user can specify the name 
of a .csv file as it is shown below
\begin{Verbatim}[frame=single, commandchars=\\\{\}, fontsize=\small,
    firstnumber=5, numbers=left, numbersep=2pt]
[geometry]
sites\_csv = <name_of_the_csv_file>


\end{Verbatim}
The format of the csv file containing the list of sites 
is a sequence of points (one per row) specified in terms 
of the longitude, latitude tuple. An example is provided below,
\begin{Verbatim}[frame=single, commandchars=\\\{\}, fontsize=\small,
    firstnumber=1, numbers=left, numbersep=2pt]
179.0,90.0
178.0,89.0
177.0,88.0
\end{Verbatim}


%
\item \textbf{Logic tree sampling} \hfill \\
    The \gls{acr:oqe} provides two options for processing the whole 
    logic tree structure. The first option uses Montecarlo sampling;
    the user in this case specifies a number of realizations. 

    In the second option all the possible realizations are created. 
    Below we provide an example for the latter option.
    In this case we set the \texttt{number\-\_of\-\_logic\_tree\_samples}
    to 0. \gls{acr:oqe} will perform a complete enumeration of all 
    the possible paths from the roots to the leaves of the logic tree 
    structure. 
\begin{Verbatim}[frame=single, commandchars=\\\{\}, fontsize=\small,
    firstnumber=9, numbers=left, numbersep=2pt]
[logic_tree]
number_of_logic_tree_samples = 0
\end{Verbatim}
    If the seismic source logic tree and the ground motion
    logic tree do not contain epistemic uncertainties the engine will
    create a single PSHA input.
%
\item \textbf{Parameters controlling the construction of the earthquake 
    rupture forecast}
\begin{Verbatim}[frame=single, commandchars=\\\{\}, fontsize=\small,
    firstnumber=11, numbers=left, numbersep=2pt]
[erf]
# km
rupture_mesh_spacing = 5
width_of_mfd_bin = 0.1
# km
area_source_discretization = 10
\end{Verbatim}
This section of the configuration file is used to specify the 
level of discretization of the mesh representing faults, the grid
used to delineate the area sources and, the magnitude-frequency 
distribution. 
Note that the smaller is the mesh spacing (or the bin width) the larger 
are (1) the precision in the calculation and (2) the computation demand.
%
\item \textbf{Parameters describing site conditions}
\begin{Verbatim}[frame=single, commandchars=\\\{\}, fontsize=\small,
    firstnumber=17, numbers=left, numbersep=2pt]
[site_params]
reference_vs30_type = measured
reference_vs30_value = 760.0
reference_depth_to_2pt5km_per_sec = 5.0
reference_depth_to_1pt0km_per_sec = 100.0
\end{Verbatim}
In this section the user specifies local soil conditions. The simplest
solution is to define uniform site conditions (i.e. all the sites have 
the same characteristics). Alternatively it's possible to define 
spatially variable soil properties in a separate file; the engine will
then assign to each investigation location the values of the closest point
used to specify site conditions.
%
\begin{Verbatim}[frame=single, commandchars=\\\{\}, fontsize=\small,
    firstnumber=17, numbers=left, numbersep=2pt]
[site_params]
site_model_file = ../_site_model/site_model.xml



\end{Verbatim}
The file containing the site model has the following 
structure:
\begin{Verbatim}[frame=single, commandchars=\\\{\}, fontsize=\small]
<?xml version="1.0" encoding="utf-8"?>
<nrml xmlns:gml="http://www.opengis.net/gml"
      xmlns="http://openquake.org/xmlns/nrml/0.4">
    <siteModel>
        <site lon="10.0" lat="40.0" vs30="800.0" 
            vs30Type="inferred" 
            z1pt0="19.367196734" z2pt5="0.588625072259" />
        <site lon="10.1" lat="40.0" vs30="800.0" 
            vs30Type="inferred" 
            z1pt0="19.367196734" z2pt5="0.588625072259" />
        <site lon="10.2" lat="40.0" vs30="800.0" 
            vs30Type="inferred" 
            z1pt0="19.367196734" z2pt5="0.588625072259" />
        <site lon="10.3" lat="40.0" vs30="800.0" 
            vs30Type="inferred" 
            z1pt0="19.367196734" z2pt5="0.588625072259" />
        <site lon="10.4" lat="40.0" vs30="800.0" 
            vs30Type="inferred" 
            z1pt0="19.367196734" z2pt5="0.588625072259" />
        ...
    </siteModel>
</nrml>
\end{Verbatim}

%
\item \textbf{Calculation configuration}
\phantomsection
\label{sec:calculation_configuration}
\begin{Verbatim}[frame=single, commandchars=\\\{\}, fontsize=\small,
     firstnumber=22, numbers=left, numbersep=2pt]
[calculation]
source_model_logic_tree_file = source_model_logic_tree.xml
gsim_logic_tree_file = gmpe_logic_tree.xml
# years
investigation_time = 50.0
intensity_measure_types = PGA, SA(0.1)
intensity_measure_types_and_levels = \{"PGA": [0.005, ..., 2.13]\} 
truncation_level = 3
# km
maximum_distance = 200.0
\end{Verbatim}
This section of the \gls{acr:oqe} configuration file specifies the 
parameters that
are relevant for the calculation of hazard. These include the names of
the two files containing the Seismic Source System and the Ground 
Motion System, the duration of the time window used to compute the 
hazard, the ground motion intensity measure types and levels for 
which the probability of exceedence will be computed, the level of
truncation of the gaussian 
distribution of the logarithm of ground motion used in the calculation 
of hazard and the maximum integration distance (i.e. the distance within 
which sources will contribute to the computation of the hazard).
%
\item \textbf{Output}
\begin{Verbatim}[frame=single, commandchars=\\\{\}, fontsize=\small,
    firstnumber=32, numbers=left, numbersep=2pt]
[output]
export_dir = out/
# given the specified `intensity_measure_types_and_levels`
quantile_hazard_curves =
poes_hazard_maps = 0.1
\end{Verbatim}
The final section of the configuration file is the one that contains 
the parameters controlling the typology of output to be produced.
%
\end{itemize}
%
%\input{config_file_event_based.tex}
%
