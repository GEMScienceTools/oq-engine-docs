\subsubsection{Seismic hazard disaggregation}
%
In this section we describe the structure of the configutation 
file to be used to complete a seismic hazard disaggregation. 
Since only a few parts of the standard configuration file need to 
be changed we'll use the description given in Section 
\ref{sec:config_classical_PSHA} at page 
\pageref{sec:config_classical_PSHA} as a reference and we'll 
emphasize herein major differences.
%
\begin{itemize}
%
\item \textbf{Calculation type and model info}
\begin{Verbatim}[frame=single, commandchars=\\\{\}, fontsize=\small]
[general]
description = A demo .ini file for PSHA disaggregation
calculation_mode = disaggregation
random_seed = 1024
\end{Verbatim}
The calculation mode parameter in this case is set as 
\texttt{disaggregation}.
%
\item \textbf{Geometry of the area (or the sites) where hazard is computed}
\begin{Verbatim}[frame=single, commandchars=\\\{\}, fontsize=\small]
[geometry]
sites = 11.0 44.5
\end{Verbatim}

In the section it will be necessary to specify the geographic 
coordinates of the site (or the sites) where the disaggregation
will be performed.
%
\item \textbf{Disaggregation parameters}
\begin{Verbatim}[frame=single, commandchars=\\\{\}, fontsize=\small]
[disaggregation]
poes_disagg = 0.02, 0.1
mag_bin_width = 1.0
distance_bin_width = 25.0
# decimal degrees
coordinate_bin_width = 1.5
num_epsilon_bins = 3
\end{Verbatim}
With the disaggregation settings shown above we'll disaggregate the intensity
measure levels with 10\% and 2\% probability of exceedance using the
\texttt{in\-ves\-ti\-gation\_time} and the intensity measure types 
defined in the ``Calculation configuration'' section of the OpenQuake
configuration file (see page \pageref{sec:calculation_configuration}). 

The parameters \texttt{mag\_bin\_width},  \texttt{distance\_bin\_width},
\texttt{coordinate\_bin\_width} control the level of discretization of the
disaggregation matrix computed. \texttt{num\_epsilon\_bins} indicates the 
number of bins used to represent the contributions provided by different
values of epsilon.

If the user is interested in a specific type of disaggregation,
we suggest to use a very coarse gridding for the parameters that are 
not necessary. For example, if the user is interested in a magnitude-distance 
disaggregation, we suggest the use of very large value for the 
\texttt{coordinate\_\-bin\_\-width} and to set  \texttt{num\_epsilon\_bins}
equal to 1.
\end{itemize}
