By default, the classical PSHA calculator computes and stores hazard curves
for each logic tree sample considered.

When the PSHA input model doesn't contain epistemic uncertainties the results
is a set of hazard curves (one for each investigated site). The command below
illustrates how is possible to retrieve the group of hazard curves obtained
for a calculation with a given identifier \texttt{<calc\_id>} (see
Section~\ref{sec:exporting_hazard_results} for an explanation about how to
obtain the list of calculations performed with their corresponding ID):

\begin{Verbatim}[frame=single, commandchars=\\\{\}, fontsize=\small]
user@ubuntu:~$ oq-engine --lho <calc_id>
id | output_type | name
\textcolor{red}{3 | hazard_curve | hc-rlz-6}
\end{Verbatim}

In this case the \gls{acr:oqe} computed a group of hazard curves with result
ID equal to \texttt{3}. On the contrary, if the parameter
\texttt{number\_of\_logic\_tree\_samples} in the configuration file is
different than zero, then N hazard curves files are generated. The example
below shows this case:

\begin{Verbatim}[frame=single, commandchars=\\\{\}, fontsize=\small]
user@ubuntu:~$ oq engine --lo <calc_id>
id | output_type | name
\textcolor{red}{5 | datastore | hcurves}
\textcolor{red}{6 | datastore | hcurves}
\textcolor{red}{7 | datastore | hcurves}
\textcolor{red}{8 | datastore | hcurves}
\textcolor{red}{9 | datastore | hcurves}
\textcolor{red}{10 | datastore | hcurves}
\end{Verbatim}

If we export from the database the hazard curves contained in one of the
items above using the following command:

\begin{Verbatim}[frame=single, commandchars=\\\{\}, fontsize=\small]
user@ubuntu:~$ oq-engine --eh <output_id> <output_directory>
\end{Verbatim}

we obtain a nrml formatted file as represented in the example in the inset
below:

\begin{Verbatim}[frame=single, commandchars=\\\{\}, fontsize=\small]
<?xml version='1.0' encoding='UTF-8'?>
<nrml xmlns:gml="http://www.opengis.net/gml"
      xmlns="http://openquake.org/xmlns/nrml/0.5">
  <hazardCurves \textcolor{red}{sourceModelTreePath="b1|b212"}
      \textcolor{red}{gsimTreePath="b2" IMT="PGA" investigationTime="50.0"}>
    \textcolor{green}{<IMLs>0.005 0.007 0.0098 ... 1.09 1.52 2.13</IMLs>}
    <hazardCurve>
      <gml:Point>
      \textcolor{blue}{<gml:pos>10.0 45.0</gml:pos>}
      </gml:Point>
      <poEs>1.0 1.0 1.0 ... 0.000688359310522 0.0 0.0</poEs>
    </hazardCurve>
    ...
    <hazardCurve>
      <gml:Point>
      \textcolor{blue}{<gml:pos>lon lat</gml:pos>}
      </gml:Point>
      <poEs>poe1 poe2 ... poeN</poEs>
    </hazardCurve>
  </hazardCurves>
</nrml>
\end{Verbatim}

Notwithstanding the intuitiveness of this file, let's have a brief
overview of the information included.

The overall content of this file is a list of hazard curves, one for each
investigated site, computed using a PSHA input model representing one possible
realisation obtained using the complete logic tree structure.

The attributes of the \texttt{hazardCurves} element (see text in red) specify
the path of the logic tree used to create the seismic source model
(\texttt{source\-Model\-TreePath}) and the ground motion model
(\texttt{gsim\-Tree\-Path}) plus the intensity measure type and the
investigation time used to compute the probability of exceedance.

The \texttt{IMLs} element (in green in the example) contains the values of
shaking used by the engine to compute the probability of exceedance in the
investigation time. For each site this file contains a \texttt{hazardCurve}
element which has the coordinates (longitude and latitude in decimal degrees)
of the site and the values of the probability of exceedance for all the
intensity measure levels specified in the \texttt{IMLs} element.

If in the configuration file the calculation of mean hazard curves and hazard
curves corresponding to one or several percentiles have been specified, the
list of outputs that we should expect from the \glsdesc{acr:oqe} corresponds
to:

\begin{Verbatim}[frame=single, commandchars=\\\{\}, fontsize=\small]
user@ubuntu:~$ oq-engine --lho <calc_id>
id | output_type | name
17 | hazard_curve | hc-rlz-17
18 | hazard_curve | hc-rlz-18
19 | hazard_curve | hc-rlz-13
20 | hazard_curve | hc-rlz-14
21 | hazard_curve | hc-rlz-15
22 | hazard_curve | hc-rlz-16
\textcolor{red}{23 | hazard_curve | quantile(0.5)-curves-PGA}
24 | hazard_map | hazard-map(0.1)-PGA-rlz-17
25 | hazard_map | hazard-map(0.1)-PGA-rlz-18
26 | hazard_map | hazard-map(0.1)-PGA-rlz-13
27 | hazard_map | hazard-map(0.1)-PGA-rlz-14
28 | hazard_map | hazard-map(0.1)-PGA-rlz-15
29 | hazard_map | hazard-map(0.1)-PGA-rlz-16
\textcolor{red}{30 | hazard_map | hazard-map(0.1)-PGA-quantile(0.5)}
\end{Verbatim}

In this example the \gls{acr:oqe} produced hazard curves and hazard maps for
six logic tree realisations plus median hazard curves and the median hazard
map (both highlighted in red).

The following inset shows a sample of the nrml file used to describe a hazard
map:

\begin{Verbatim}[frame=single, commandchars=\\\{\}, fontsize=\small]
<?xml version='1.0' encoding='UTF-8'?>
<nrml xmlns:gml="http://www.opengis.net/gml"
      xmlns="http://openquake.org/xmlns/nrml/0.4">
  \textcolor{red}{<hazardMap sourceModelTreePath="b1" gsimTreePath="b1"}
        \textcolor{red}{IMT="PGA" investigationTime="50.0" poE="0.1">}
    <node lon="119.596690957" lat="21.5497682591" iml="0.204569990197"/>
    <node lon="119.596751048" lat="21.6397004197" iml="0.212391638188"/>
    <node lon="119.596811453" lat="21.7296325803" iml="0.221407505615"/>
    ...
  </hazardMap>
</nrml>
\end{Verbatim}