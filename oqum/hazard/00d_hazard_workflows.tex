The hazard component of the OpenQuake-engine can compute seismic hazard using
various approaches. Three types of analysis are currently supported:

\begin{itemize}

	\item \textit{Classical Probabilistic Seismic Hazard Analysis (PSHA)},
	allowing calculation of hazard curves and hazard maps following the
	classical integration procedure (\cite{cornell1968}, \citet{mcguire1976})
	as formulated by \cite{field2003}.

	\item \textit{Event-Based Probabilistic Seismic Hazard Analysis},
	allowing calculation of ground-motion fields from stochastic event sets.
	Traditional results - such as hazard curves - can be obtained by post-
	processing the set of computed ground-motion fields.

	\item \textit{\gls{acr:ssha}}, allowing the calculation of ground
	motion fields from a single earthquake rupture scenario taking into
	account ground-motion aleatory variability.

\end{itemize}

Each workflow has a modular structure, so that intermediate results can be
exported and analyzed. Each calculator can be extended independently of the
others so that additional calculation options and methodologies can be easily
introduced, without affecting the overall calculation workflow.



\subsection{Classical Probabilistic Seismic Hazard Analysis}
\index{OpenQuake-engine!Hazard calculation workflows!Classical PSHA}
\label{subsec:classical_psha}
Input data for the classical \gls{acr:psha} consist of a PSHA input model
provided together with calculation settings.

The main calculators used to perform this analysis are the following:

\begin{enumerate}

	\item \emph{Logic Tree Processor}

	The Logic Tree Processor (LTP) takes as an input the \gls{acr:psha} Input
	Model and creates a Seismic Source Model. The LTP uses the information in
	the Initial Seismic Source Models and the Seismic Source Logic Tree to
	create a Seismic Source Input Model (i.e. a model describing geometry and
	activity rates of each source without any epistemic uncertainty).

	Following a procedure similar to the one just described the Logic Tree
	Processor creates a Ground Motion model (i.e. a data structure that
	associates to each tectonic region considered in the calculation a
	\gls{acr:gmpe}).

	\item \emph{Earthquake Rupture Forecast Calculator}

	The produced Seismic Source Input Model becomes an input information for
	the Earthquake Rupture Forecast (ERF) calculator which creates a list
	earthquake ruptures admitted by the source model, each one characterized
	by a probability of occurrence over a specified time span.

	\item \emph{Classical PSHA Calculator}

	The classical PSHA calculator uses the ERF and the Ground Motion model to
	compute hazard curves on each site specified in the calculation settings.

\end{enumerate}

\subsection{Event-Based Probabilistic Seismic Hazard Analysis}
\index{OpenQuake-engine!Hazard calculation workflows!Event-based PSHA}
\label{subsec:event_based_psha}
Input data for the Event-Based PSHA - as in the case of the Classical
\gls{acr:psha} calculator - consists of a PSHA Input Model and a set of
calculation settings.

The main calculators used to perform this analysis are:

\begin{enumerate}

	\item \emph{Logic Tree Processor}

	The Logic Tree Processor works in the same way described in  the
	description of the Classical \gls{acr:psha} workflow  (see
	Section~\ref{subsec:classical_psha} at
	page~\pageref{subsec:classical_psha}).

	\item \emph{Earthquake Rupture Forecast Calculator}

	The Earthquake Rupture Forecast Calculator was already  introduced in the
	description of the PSHA workflow (see Section~\ref{subsec:classical_psha}
	at page~\pageref{subsec:classical_psha}).

	\item \emph{Stochastic Event Set Calculator}

	The Stochastic Event Set Calculator generates a collection of stochastic
	event sets by sampling the ruptures contained in the ERF according to
	their probability of occurrence.

	A Stochastic Event Set (SES) thus represents a potential realisation of
	the seismicity (i.e. a list of ruptures) produced by the set of seismic
	sources considered in the analysis over the time span fixed for the
	calculation of hazard.

	\item \emph{Ground Motion Field Calculator}

	The Ground Motion Field Calculator computes for each event contained in a
	Stochastic Event Set a realization of the geographic distribution of the
	shaking by taking into account the aleatory uncertainties in the ground-
	motion model. Eventually, the Ground Motion Field calculator can consider
	the spatial correlation of the ground-motion during the generation of the
	\gls{acr:gmf}.

	\item \emph{Event-based PSHA Calculator}

	The event-based PSHA calculator takes a (large) set of ground-motion
	fields representative of the possible shaking scenarios that the
	investigated area can experience over a (long) time span and for each
	site computes the corresponding hazard curve.

	This procedure is computationally intensive and is not recommended for
	investigating the hazard over large areas.

\end{enumerate}

\subsection{Scenario based Seismic Hazard Analysis}
\index{OpenQuake-engine!Hazard calculation workflows!Scenario-based SHA}
\label{subsec:scenario_hazard}
In case of \gls{acr:ssha}, the input data consist of a single earthquake
rupture model and one or more ground-motion models. Using the Ground Motion Field Calculator, multiple realizations of ground shaking can be computed, each realization sampling the aleatory uncertainties in the ground-motion model. The main calculator used to perform this analysis is the \emph{Ground Motion Field Calculator}, which was already introduced during the description of the event based PSHA workflow (see Section~\ref{subsec:event_based_psha} at page~\pageref{subsec:event_based_psha}).

As the scenario calculator does not need to determine the probability of occurrence of the specific rupture, but only sufficient information to parameterise the location (as a three-dimensional surface), the magnitude and the style-of-faulting of the rupture, a more simplified NRML structure is needed. A \emph{rupture model} XML can be defined in the following formats:

\begin{enumerate}
    \item \emph{Simple Fault Rupture} - in which the geometry is defined by the trace of the fault rupture, the dip and the upper and lower seismogenic depths. An example is shown below:
\begin{minted}[firstline=1,firstnumber=1,fontsize=\footnotesize,frame=single,bgcolor=lightgray]{xml}
<?xml version='1.0' encoding='utf-8'?>
<nrml xmlns:gml="http://www.opengis.net/gml"
      xmlns="http://openquake.org/xmlns/nrml/0.5">
    <simpleFaultRupture>
        <magnitude>7.0</magnitude>
        <rake>90.0</rake>
        <hypocenter lat="0.0" lon="0.0" depth="10.0"/>
        <simpleFaultGeometry>
            <gml:LineString>
                <gml:posList>
                    0.0 -0.3
                    0.0  0.3
                </gml:posList>
            </gml:LineString>
            <dip>90.0</dip>
            <upperSeismoDepth>2.0</upperSeismoDepth>
            <lowerSeismoDepth>20.0</lowerSeismoDepth>
        </simpleFaultGeometry>
    </simpleFaultRupture>
</nrml>
\end{minted}
\\
    \item \emph{Planar \& Multi-Planar Rupture} - in which the geometry is defined as a collection of one or more rectangular planes, each defined by four corners.

    \begin{minted}[firstline=1,firstnumber=1,fontsize=\footnotesize,frame=single,bgcolor=lightgray]{xml}
<?xml version='1.0' encoding='utf-8'?>
<nrml xmlns:gml="http://www.opengis.net/gml"
      xmlns="http://openquake.org/xmlns/nrml/0.5">
    <multiPlanesRupture>
        <magnitude>8.0</magnitude>
        <rake>90.0</rake>
        <hypocenter lat="-1.4" lon="1.1" depth="10.0"/>
            <planarSurface strike="90.0" dip="45.0">
                <topLeft lon="-0.8" lat="-2.3" depth="0.0" />
                <topRight lon="-0.4" lat="-2.3" depth="0.0" />
                <bottomLeft lon="-0.8" lat="-2.3890" depth="10.0" />
                <bottomRight lon="-0.4" lat="-2.3890" depth="10.0" />
            </planarSurface>
            <planarSurface strike="30.94744" dip="30.0">
                <topLeft lon="-0.42" lat="-2.3" depth="0.0" />
                <topRight lon="-0.29967" lat="-2.09945" depth="0.0" />
                <bottomLeft lon="-0.28629" lat="-2.38009" depth="10.0" />
                <bottomRight lon="-0.16598" lat="-2.17955" depth="10.0" />
            </planarSurface>
    </multiPlanesRupture>
</nrml> 
\end{minted}
\\
    \item \emph{Complex Fault Rupture} - in which the geometry is defined by the upper, lower and (if applicable) intermediate edges of the fault rupture.

\begin{minted}[firstline=1,firstnumber=1,fontsize=\footnotesize,frame=single,bgcolor=lightgray]{xml}
    <?xml version='1.0' encoding='utf-8'?>
<nrml xmlns:gml="http://www.opengis.net/gml"
      xmlns="http://openquake.org/xmlns/nrml/0.5">
    <complexFaultRupture>
        <magnitude>8.0</magnitude>
        <rake>90.0</rake>
        <hypocenter lat="-1.4" lon="1.1" depth="10.0"/>
        <complexFaultGeometry>
            <faultTopEdge>
                <gml:LineString>
                    <gml:posList>
                        0.6 -1.5 2.0
                        1.0 -1.3 5.0
                        1.5 -1.0 8.0
                    </gml:posList>
                </gml:LineString>
            </faultTopEdge>
            <intermediateEdge>
                <gml:LineString>
                    <gml:posList>
                        0.65 -1.55 4.0
                        1.1  -1.4  10.0
                        1.5  -1.2  20.0
                    </gml:posList>
                </gml:LineString>
            </intermediateEdge>
            <faultBottomEdge>
                <gml:LineString>
                    <gml:posList>
                        0.65 -1.7 8.0
                        1.1  -1.6 15.0
                        1.5  -1.7 35.0
                    </gml:posList>
                </gml:LineString>
            </faultBottomEdge>
        </complexFaultGeometry>
    </complexFaultRupture>
</nrml>
\end{minted}
\end{enumerate}


\cleardoublepage