The Event Based PSHA calculator computes and stores stochastic event sets and
the corresponding ground motion fields.

This calculator can also produce hazard curves and hazard maps exactly in the
same way as done using the Classical PSHA calculator.

The inset below shows an example of the list of results provided by the
\gls{acr:oqe} at the end of an event-based PSHA calculation:

\begin{Verbatim}[frame=single, commandchars=\\\{\}, fontsize=\small]
user@ubuntu:~$ oq-engine --lho <calc_id>
id | output_type | name
\textcolor{red}{31 | ses | ses-coll-rlz-19}
\textcolor{green}{32 | gmf | gmf-rlz-19}
\textcolor{red}{33 | ses | ses-coll-rlz-20}
\textcolor{green}{34 | gmf | gmf-rlz-20}
35 | hazard_curve | hazard-curve-rlz-19-SA(0.1)
36 | hazard_curve | hazard-curve-rlz-20-SA(0.1)
37 | hazard_curve | hazard-curve-rlz-19-PGA
38 | hazard_curve | hazard-curve-rlz-20-PGA
39 | hazard_curve | mean curve for SA(0.1)
40 | hazard_curve | quantile curve (poe >= 0.15) for imt SA(0.1)
41 | hazard_curve | quantile curve (poe >= 0.50) for imt SA(0.1)
42 | hazard_curve | quantile curve (poe >= 0.85) for imt SA(0.1)
43 | hazard_curve | mean curve for PGA
44 | hazard_curve | quantile curve (poe >= 0.15) for imt PGA
45 | hazard_curve | quantile curve (poe >= 0.50) for imt PGA
46 | hazard_curve | quantile curve (poe >= 0.85) for imt PGA
\end{Verbatim}

This list in the inset above contains two sets of stochastic events (in red)
and two sets of ground motion fields (in blue).

The whole group of stochastic event set and ground motion fields can be
exported immediately using the results with \texttt{id} 35 and 25, respectively.

Below is an example showing a nrml file containing a collection of stochastic
event sets (2 ruptures):

\input{oqum/hazard/verbatim/output_ses}

The text in red shows the part which describes the id of the generated
stochastic event set and the investigation time covered.

The text in green emphasises the portion of the text used to describe a
rupture. The information provided describes entirely the geometry of the rupture as well as its rupturing properties (e.g. rake, magnitude). The rupture ID is a string that represents each rupture uniquely (including the case where the same rupture is sampled multiple times). The exact format may be subject to change from time-to-time; however, the ID will usually contain information regarding the specific logic tree branch sample (realisation), the stochastic event set counter, the ID of the source from which the rupture was samples and a unique number indicating the rupture identifier within the earthquake rupture forecast of the specific source.

This is an example of a nrml file containing one ground motion field:

\input{oqum/hazard/verbatim/output_gmf}
