\section{Output from Event Based PSHA}\label{EventBasedOutput}
%
The Event Based PSHA calculator computes and stores stochastic 
event sets and the corresponding ground motion fields. 
%
This calculator can also produce hazard curves and hazard maps
exactly in the same way is done using the Classical PSHA calculator.

The inset below shows an example of the list of results provided by 
the \gls{acr:oqe} at the end of an event-based PSHA calculation:
%
\begin{Verbatim}[frame=single, commandchars=\\\{\}, fontsize=\small]
user@ubuntu:~$ openquake --lho <calc_id> 
id | output_type | name
\textcolor{red}{31 | ses | ses-coll-rlz-19}
\textcolor{green}{32 | gmf | gmf-rlz-19}
\textcolor{red}{33 | ses | ses-coll-rlz-20}
\textcolor{green}{34 | gmf | gmf-rlz-20}
\textcolor{red}{35 | complete_lt_ses | complete logic tree SES}
\textcolor{green}{36 | complete_lt_gmf | complete logic tree GMF}
37 | hazard_curve | hazard-curve-rlz-19-SA(0.1)
38 | hazard_curve | hazard-curve-rlz-20-SA(0.1)
39 | hazard_curve | hazard-curve-rlz-19-PGA
40 | hazard_curve | hazard-curve-rlz-20-PGA
41 | hazard_curve | mean curve for SA(0.1)
42 | hazard_curve | quantile curve (poe >= 0.15) for imt SA(0.1)
43 | hazard_curve | quantile curve (poe >= 0.5) for imt SA(0.1)
44 | hazard_curve | quantile curve (poe >= 0.85) for imt SA(0.1)
45 | hazard_curve | mean curve for PGA
46 | hazard_curve | quantile curve (poe >= 0.15) for imt PGA
47 | hazard_curve | quantile curve (poe >= 0.5) for imt PGA
48 | hazard_curve | quantile curve (poe >= 0.85) for imt PGA
\end{Verbatim}
This list in the inset above contains two sets of stochastic events 
(in red) and two sets of ground motion fields (in blue).

The whole group of stochastic event set and ground motion fields can 
be exported immediately using the results with \texttt{id} 35 and 25,
respectively.

\subsection{Example of files containing a stochastic event set and a 
ground motion field}

This is an example showing a nrml file containing a collection of 
stochastic event sets (2 ruptures)
\begin{Verbatim}[frame=single, commandchars=\\\{\}, fontsize=\small]
<?xml version='1.0' encoding='UTF-8'?>
<nrml xmlns:gml="http://www.opengis.net/gml" 
	  xmlns="http://openquake.org/xmlns/nrml/0.4">
  <stochasticEventSetCollection sourceModelTreePath="b1" 
          gsimTreePath="b1">
    \textcolor{red}{<stochasticEventSet id="12" investigationTime="50.0">}
\textcolor{green}{     <rupture id="533" magnitude="4.55" strike="90.0" dip="90.0"}
\textcolor{green}{              rake="90.0" tectonicRegion="Active Shallow Crust">}
\textcolor{green}{       <planarSurface>}
\textcolor{green}{         <topLeft lon="12.233903801" lat="43.256198599"}
\textcolor{green}{                  depth="11.3933265259"/>}
\textcolor{green}{         <topRight lon="12.263958243" lat="43.2562025344"}
\textcolor{green}{                  depth="11.3933265259"/>}
\textcolor{green}{         <bottomLeft lon="12.233903801" lat="43.256198599"}
\textcolor{green}{                  depth="12.6066734741"/>}
\textcolor{green}{         <bottomRight lon="12.263958243" lat="43.2562025344"}
\textcolor{green}{                  depth="12.6066734741"/>}
\textcolor{green}{       </planarSurface>}
\textcolor{green}{     </rupture>}
      <rupture id="535" magnitude="4.65" strike="135.0" dip="90.0" 
              rake="90.0" tectonicRegion="Active Shallow Crust">
        <planarSurface>
          <topLeft lon="11.45858812" lat="42.7429056814" 
                  depth="11.3208667302"/>
          <topRight lon="11.4822820715" lat="42.7256333907" 
                  depth="11.3208667302"/>
          <bottomLeft lon="11.45858812" lat="42.7429056814" 
                  depth="12.6791332698"/>
          <bottomRight lon="11.4822820715" lat="42.7256333907" 
                  depth="12.6791332698"/>
        </planarSurface>
      </rupture>
    \textcolor{red}{</stochasticEventSet>}
  </stochasticEventSetCollection>
</nrml>
\end{Verbatim}

The text in red shows the part which describes the id of the generated
stochastic event set and the investigation time covered.
%
The text in green emphasises the portion of the text used to describe 
a rupture. The informtion provided describes entirely the geometry of
the rupture as well as its rupturing properties (e.g. rake, magnitude).

This is an example of a nrml file containing one 
ground motion field:
\begin{Verbatim}[frame=single, commandchars=\\\{\}, fontsize=\small]
<?xml version='1.0' encoding='UTF-8'?>
<nrml xmlns:gml="http://www.opengis.net/gml" 
      xmlns="http://openquake.org/xmlns/nrml/0.4">
  <gmfCollection sourceModelTreePath="b1" gsimTreePath="b1">
    <gmfSet investigationTime="50.0" stochasticEventSetId="12">
      <gmf IMT="PGA" ruptureId="533">
        <node gmv="0.0105891230432" lon="11.1240023202" 
            lat="43.5107462335"/>
        <node gmv="0.00905803920023" lon="11.1241875202" 
            lat="43.6006783941"/>
        <node gmv="0.00637664420977" lon="11.124373581" 
            lat="43.6906105547"/>
        <node gmv="0.00476533134789" lon="11.1245605075" 
            lat="43.7805427153"/>
        <node gmv="0.00452594698469" lon="11.1247483046" 
            lat="43.8704748759"/>
        ...
        <node gmv="0.000173010769646" lon="11.3782630185" 
            lat="44.5"/>
      </gmf>
    </gmfSet>
  </gmfCollection>
</nrml>
\end{Verbatim}
