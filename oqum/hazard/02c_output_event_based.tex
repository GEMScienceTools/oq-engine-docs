\section{Output from Event Based PSHA}\label{EventBasedOutput}
%
The Event Based PSHA calculator computes and stores stochastic 
event sets and the corresponding ground motion fields. In case,
this calculator can also compute hazard curves and hazard maps
exactly in the same way is done using the Classical PSHA calculator.

A stochastic event set is a set of ruptures representing a possible
realisation of the seismicity produced by the seismic sources 
included in the PSHA input model, given an \texttt{investigation time}.
%
A ground motion field is defined in a \Verb+GMF+ element, which 
consists of a sequence of \Verb+GMFNode+ elements, 
each containing a geographical location and the associated ground 
motion value.

The inset below shows an example of the list of results provided by 
\gls{acr:oqe} at the end of an event-based PSHA calculation:
\begin{Verbatim}[frame=single, commandchars=\\\{\}]
user@ubuntu:~$ openquake --lho <calc_id> 
id | output_type | name
\textcolor{red}{31 | ses | ses-coll-rlz-19}
\textcolor{green}{32 | gmf | gmf-rlz-19}
\textcolor{red}{33 | ses | ses-coll-rlz-20}
\textcolor{green}{34 | gmf | gmf-rlz-20}
\textcolor{red}{35 | complete_lt_ses | complete logic tree SES}
\textcolor{green}{36 | complete_lt_gmf | complete logic tree GMF}
37 | hazard_curve | hazard-curve-rlz-19-SA(0.1)
38 | hazard_curve | hazard-curve-rlz-20-SA(0.1)
39 | hazard_curve | hazard-curve-rlz-19-PGA
40 | hazard_curve | hazard-curve-rlz-20-PGA
41 | hazard_curve | mean curve for SA(0.1)
42 | hazard_curve | quantile curve (poe >= 0.15) for imt SA(0.1)
43 | hazard_curve | quantile curve (poe >= 0.5) for imt SA(0.1)
44 | hazard_curve | quantile curve (poe >= 0.85) for imt SA(0.1)
45 | hazard_curve | mean curve for PGA
46 | hazard_curve | quantile curve (poe >= 0.15) for imt PGA
47 | hazard_curve | quantile curve (poe >= 0.5) for imt PGA
48 | hazard_curve | quantile curve (poe >= 0.85) for imt PGA
\end{Verbatim}
This list contains two sets of stochastic events (in red in the example 
above) two sets of ground motion fields (in blue in the example above).
The whole group of stochastic event set and ground motion fields can 
be exported immediately using the results with \texttt{id} 35 and 25,
respectively.
