\section{Output from Event Based PSHA}\label{EventBasedOutput}
%
The Event Based PSHA calculator computes and stores stochastic 
event sets and the corresponding ground motion fields. 

A ground motion field is defined in a \Verb+GMF+ element, which 
consists of a sequence of \Verb+GMFNode+ elements, 
each containing a geographical location and the associated ground 
motion value.
\begin{Verbatim}[frame=single, commandchars=\\\{\}, samepage=true]
user@ubuntu:~$ openquake --lho <calc_id> 
id | output_type | name
31 | ses | ses-coll-rlz-19
32 | gmf | gmf-rlz-19
33 | ses | ses-coll-rlz-20
34 | gmf | gmf-rlz-20
35 | complete_lt_ses | complete logic tree SES
36 | complete_lt_gmf | complete logic tree GMF
37 | hazard_curve | hazard-curve-rlz-19-SA(0.1)
38 | hazard_curve | hazard-curve-rlz-20-SA(0.1)
39 | hazard_curve | hazard-curve-rlz-19-PGA
40 | hazard_curve | hazard-curve-rlz-20-PGA
41 | hazard_curve | mean curve for SA(0.1)
42 | hazard_curve | quantile curve (poe >= 0.15) for imt SA(0.1)
43 | hazard_curve | quantile curve (poe >= 0.5) for imt SA(0.1)
44 | hazard_curve | quantile curve (poe >= 0.85) for imt SA(0.1)
45 | hazard_curve | mean curve for PGA
46 | hazard_curve | quantile curve (poe >= 0.15) for imt PGA
47 | hazard_curve | quantile curve (poe >= 0.5) for imt PGA
48 | hazard_curve | quantile curve (poe >= 0.85) for imt PGA
\end{Verbatim}

