% -----------------------------------------------------------------------------
\section{OpenQuake Hazard Demos}
With the \gls{acr:oqe} installation a number of demos are provided showing practical examples
of input and configuration files definition for different use cases.
The following demos illustrate how to use \gls{acr:oqe} for various seismic hazard analysis:
\begin{itemize}
\item PointSourceClassicalPSHA
\item AreaSourceClassicalPSHA
\item SimpleFaultSourceClassicalPSHA
\item ComplexFaultSourceClassicalPSHA
\item CharacteristicFaultSourceCase1ClassicalPSHA
\item CharacteristicFaultSourceCase2ClassicalPSHA
\item CharacteristicFaultSourceCase3ClassicalPSHA
\item LogicTreeCase1ClassicalPSHA
\item LogicTreeCase2ClassicalPSHA
\item LogicTreeCase3ClassicalPSHA
\item EventBasedPSHA
\item Disaggregation
\end{itemize}

\subsection{Classical PSHA Demos}
A number of demos have been designed to show how to perform a classical PSHA calculation using the different available
source typologies and how to define non-trivial logic trees. It should be noted that the input files that will be illustrated are valid
not only for a classical PSHA calculation but also for event based and disaggregation analysis.\\

\subsubsection{Classical PSHA with different source typologies}
All the classical PSHA demos illustrating the different source typologies (all demos but the ones about Logic Tree definition)
share the following GSIM logic tree file:
\begin{Verbatim}[frame=single, commandchars=\\\{\}, fontsize=\normalsize]
<?xml version="1.0" encoding="UTF-8"?>

<nrml xmlns:gml="http://www.opengis.net/gml"
      xmlns="http://openquake.org/xmlns/nrml/0.4">
    <logicTree logicTreeID='lt1'>

        <logicTreeBranchingLevel branchingLevelID="bl1">
            <logicTreeBranchSet
               uncertaintyType="gmpeModel"
               applyToTectonicRegionType="Active Shallow Crust"
               branchSetID="bs1">

                <logicTreeBranch branchID="b1">
                     <uncertaintyModel>
                        BooreAtkinson2008
                     </uncertaintyModel>
                     <uncertaintyWeight>1.0</uncertaintyWeight>
                </logicTreeBranch>

            </logicTreeBranchSet>
        </logicTreeBranchingLevel>

    </logicTree>
</nrml>
\end{Verbatim}
Each source in the source model file is defined as belonging to \texttt{Active Shallow Crust}, and therefore the GSIM Logic Tree file
defines for simplicity a single GMPE (Boore and Atkinson 2008) which is associated to an \texttt{Active Shallow Crust} tectonic region type.\\
The configuration file is also common (each demo differs in terms of the \texttt{region} key only):
\begin{Verbatim}[frame=single, commandchars=\\\{\}, fontsize=\normalsize]
[general]

description = ...
calculation_mode = classical
random_seed = 23

[geometry]

region = ...
region_grid_spacing = 5.0

[logic_tree]

number_of_logic_tree_samples = 0

[erf]

rupture_mesh_spacing = 2
width_of_mfd_bin = 0.1
area_source_discretization = 5.0

[site_params]

reference_vs30_type = measured
reference_vs30_value = 600.0
reference_depth_to_2pt5km_per_sec = 5.0
reference_depth_to_1pt0km_per_sec = 100.0

[calculation]

source_model_logic_tree_file = source_model_logic_tree.xml
gsim_logic_tree_file = gmpe_logic_tree.xml
investigation_time = 50.0
intensity_measure_types_and_levels =\{
"PGV": [2, 4, 6 ,8, 10, ...], 
"PGA": [0.005, 0.007, ...], 
"SA(0.025)": [...], 
"SA(0.05)": [...],
"SA(0.1)": [...], 
"SA(0.2)": [...], 
"SA(0.5)": [...], 
"SA(1.0)": [...],
"SA(2.0)": [...]\}
truncation_level = 3
maximum_distance = 200.0

[output]

export_dir = ...
mean_hazard_curves = false
quantile_hazard_curves =
hazard_maps = true
uniform_hazard_spectra = true
poes = 0.1 0.02
\end{Verbatim}
The configuration file sets the calculation type to \texttt{classical} and requests hazard curves for PGV, PGA, and a number of
spectral acceleration periods. Hazard maps and uniform hazard spectra are also requested (the keys \texttt{hazard\_\-maps}
and \texttt{uniform\_\-hazard\_\-spectra} are both set to \texttt{true} ) and are computed for probability levels equal to 0.1 and 0.02).
Each source is then described according to the schema as explained in \label{chap:oqhazintro}.

\subsubsection{Classical PSHA with non trivial logic trees}
Three demos are provided to illustrate how the logic tree formalism can be used to express epistemic uncertainties in seismic hazard analysis.\\

LogicTreeCase1ClassicalPSHA shows an example of logic tree defining two alternative source models, with sources belonging to two different
tectonic region types, and with two alternative GMPEs for each tectonic region type.
The source model logic tree is therefore defined in the following way:
\begin{Verbatim}[frame=single, commandchars=\\\{\}, fontsize=\normalsize]
<?xml version="1.0" encoding="UTF-8"?>
<nrml xmlns:gml="http://www.opengis.net/gml"
      xmlns="http://openquake.org/xmlns/nrml/0.4">
    <logicTree logicTreeID="lt1">

        <logicTreeBranchingLevel branchingLevelID="bl1">

            <logicTreeBranchSet uncertaintyType="sourceModel"
                                branchSetID="bs1">
                <logicTreeBranch branchID="b1">
                    <uncertaintyModel>
                      source_model_1.xml
                    </uncertaintyModel>
                    <uncertaintyWeight>0.5</uncertaintyWeight>
                </logicTreeBranch>
                <logicTreeBranch branchID="b2">
                    <uncertaintyModel>
                       source_model_2.xml
                    </uncertaintyModel>
                    <uncertaintyWeight>0.5</uncertaintyWeight>
                </logicTreeBranch>
            </logicTreeBranchSet>

        </logicTreeBranchingLevel>

    </logicTree>
</nrml>
\end{Verbatim}
The two source models are defined in two different source model files \texttt{source\_\-model\_\-1.xml} and \texttt{source\_\-model\_\-2.xml} each
associated to the corresponding weight (0.5 in both cases).\\
The GSIM logic tree file contains the following structure:
\begin{Verbatim}[frame=single, commandchars=\\\{\}, fontsize=\normalsize]
<?xml version="1.0" encoding="UTF-8"?>

<nrml xmlns:gml="http://www.opengis.net/gml"
      xmlns="http://openquake.org/xmlns/nrml/0.4">
    <logicTree logicTreeID='lt1'>

        <logicTreeBranchingLevel branchingLevelID="bl1">
            <logicTreeBranchSet uncertaintyType="gmpeModel"
               applyToTectonicRegionType="Active Shallow Crust"
               branchSetID="bs1">
                <logicTreeBranch branchID="b11">
                   <uncertaintyModel>
                      BooreAtkinson2008
                   </uncertaintyModel>
                   <uncertaintyWeight>0.5</uncertaintyWeight>
                </logicTreeBranch>
                <logicTreeBranch branchID="b12">
                   <uncertaintyModel>
                      ChiouYoungs2008
                   </uncertaintyModel>
                   <uncertaintyWeight>0.5</uncertaintyWeight>
                </logicTreeBranch>
            </logicTreeBranchSet>
        </logicTreeBranchingLevel>

        <logicTreeBranchingLevel branchingLevelID="bl2">
            <logicTreeBranchSet uncertaintyType="gmpeModel"
              applyToTectonicRegionType="Stable Continental Crust"
              branchSetID="bs2">
              <logicTreeBranch branchID="b21">
                <uncertaintyModel>
                   ToroEtAl2002</uncertaintyModel>
                <uncertaintyWeight>0.5</uncertaintyWeight>
                </logicTreeBranch>
                <logicTreeBranch branchID="b22">
                  <uncertaintyModel>
                     Campbell2003</uncertaintyModel>
                  <uncertaintyWeight>0.5</uncertaintyWeight>
                </logicTreeBranch>
            </logicTreeBranchSet>
        </logicTreeBranchingLevel>

    </logicTree>
</nrml>
\end{Verbatim}
The source model contains sources belonging to Active Shallow Crust and Stable Continental Crust, therefore the
GSIM logic tree defines two branching levels, one for each considered tectonic region type. Moreover for each tectonic
region type a branch set with two GMPEs is defined: Boore and Atkinson 2008 and Chiou and Youngs 2008 for Active
Shallow Crust and Toro et al. 2003 and Campbell 2003 for Stable Continental Crust. By processing the above logic tree
files using the logic tree path enumeration mode (enabled by setting in the configuration file \texttt{number\_\-of\_\-logic\_\-tree\_\-samples = 0})
hazard results are obtained for 8 logic tree paths (2 source models x 2 GMPEs for Active x 2 GMPEs for Stable).\\

LogicTreeCase2ClassicalPSHA defines a single source model consisting of only two sources (area and simple fault) belonging to different
tectonic region types (Active Shallow Crust and Stable Continental Region) and both characterized by a truncated Gutenberg-Richter distribution.
The logic tree defines uncertainties for G-R a, b values (three possible pairs for each source) and maximum magnitude (three values for each source) 
and uncertainties on the GMPEs for each tectonic region type (two GMPE per region type).\\
To accomodate such a structure the GSIM logic tree is defined in the following way:
\begin{Verbatim}[frame=single, commandchars=\\\{\}, fontsize=\normalsize]
<?xml version="1.0" encoding="UTF-8"?>
<nrml xmlns:gml="http://www.opengis.net/gml"
      xmlns="http://openquake.org/xmlns/nrml/0.4">
    <logicTree logicTreeID="lt1">

        <logicTreeBranchingLevel branchingLevelID="bl1">
            <logicTreeBranchSet uncertaintyType="sourceModel"
                                branchSetID="bs1">
                <logicTreeBranch branchID="b11">
                    <uncertaintyModel>
                     source_model.xml
                    </uncertaintyModel>
                    <uncertaintyWeight>1.0</uncertaintyWeight>
                </logicTreeBranch>
            </logicTreeBranchSet>
        </logicTreeBranchingLevel>

        <logicTreeBranchingLevel branchingLevelID="bl2">
            <logicTreeBranchSet uncertaintyType="abGRAbsolute"
                                applyToSources="1"
                                branchSetID="bs21">
                <logicTreeBranch branchID="b21">
                    <uncertaintyModel>4.6 1.1</uncertaintyModel>
                    <uncertaintyWeight>0.333</uncertaintyWeight>
                </logicTreeBranch>
                <logicTreeBranch branchID="b22">
                    <uncertaintyModel>4.5 1.0</uncertaintyModel>
                    <uncertaintyWeight>0.333</uncertaintyWeight>
                </logicTreeBranch>
                <logicTreeBranch branchID="b23">
                    <uncertaintyModel>4.4 0.9</uncertaintyModel>
                    <uncertaintyWeight>0.334</uncertaintyWeight>
                </logicTreeBranch>
            </logicTreeBranchSet>
        </logicTreeBranchingLevel>

        <logicTreeBranchingLevel branchingLevelID="bl3">
            <logicTreeBranchSet uncertaintyType="abGRAbsolute"
                                applyToSources="2"
                                branchSetID="bs31">
                <logicTreeBranch branchID="b31">
                    <uncertaintyModel>3.3 1.0</uncertaintyModel>
                    <uncertaintyWeight>0.333</uncertaintyWeight>
                </logicTreeBranch>
                <logicTreeBranch branchID="b32">
                    <uncertaintyModel>3.2 0.9</uncertaintyModel>
                    <uncertaintyWeight>0.333</uncertaintyWeight>
                </logicTreeBranch>
                <logicTreeBranch branchID="b33">
                    <uncertaintyModel>3.1 0.8</uncertaintyModel>
                    <uncertaintyWeight>0.334</uncertaintyWeight>
                </logicTreeBranch>
            </logicTreeBranchSet>
        </logicTreeBranchingLevel>

        <logicTreeBranchingLevel branchingLevelID="bl4">
            <logicTreeBranchSet uncertaintyType="maxMagGRAbsolute"
                                applyToSources="1"
                                branchSetID="bs41">
                <logicTreeBranch branchID="b41">
                    <uncertaintyModel>7.0</uncertaintyModel>
                    <uncertaintyWeight>0.333</uncertaintyWeight>
                </logicTreeBranch>
                <logicTreeBranch branchID="b42">
                    <uncertaintyModel>7.3</uncertaintyModel>
                    <uncertaintyWeight>0.333</uncertaintyWeight>
                </logicTreeBranch>
                <logicTreeBranch branchID="b43">
                    <uncertaintyModel>7.6</uncertaintyModel>
                    <uncertaintyWeight>0.334</uncertaintyWeight>
                </logicTreeBranch>
            </logicTreeBranchSet>
        </logicTreeBranchingLevel>

        <logicTreeBranchingLevel branchingLevelID="bl5">
            <logicTreeBranchSet uncertaintyType="maxMagGRAbsolute"
                                applyToSources="2"
                                branchSetID="bs51">
                <logicTreeBranch branchID="b51">
                    <uncertaintyModel>7.5</uncertaintyModel>
                    <uncertaintyWeight>0.333</uncertaintyWeight>
                </logicTreeBranch>
                <logicTreeBranch branchID="b52">
                    <uncertaintyModel>7.8</uncertaintyModel>
                    <uncertaintyWeight>0.333</uncertaintyWeight>
                </logicTreeBranch>
                <logicTreeBranch branchID="b53">
                    <uncertaintyModel>8.0</uncertaintyModel>
                    <uncertaintyWeight>0.334</uncertaintyWeight>
                </logicTreeBranch>
            </logicTreeBranchSet>
        </logicTreeBranchingLevel>

    </logicTree>
</nrml>
\end{Verbatim}
The first branching level defines the source model. For each source, two branching levels are created, one defining
uncertainties on G-R a and b values (defined by setting \texttt{uncertaintyType="abGRAbsolute"}) and G-R maximum
magnitude (\texttt{uncertaintyType="maxMagGRAbsolute"}). It is important to notice that each branch set is applied
to a specific source by defining the attribute \texttt{applyToSources}, followed by the source ID. The GSIM logic tree file is
the same as used for LogicTreeCas1ClassicalPSHA. By setting in the configuration file \texttt{number\_\-of\_\-logic\_\-tree\_\-samples = 0},
hazard results are obtained for 324 paths (1 source model x 3 (a, b) pairs for source 1 x  3 (a, b) pairs for source 2 x 3 max magnitude values
for source 1 x 3 max magnitude values for source 2 x 2 GMPEs for Active Shallow Crust X 2 GMPEs for Stable Continental Crust).\\

LogicTreeCas3ClassicalPSHA illustrates an example of logic tree defining relative uncertainties on G-R maximum magnitude and b value.
A single source model is considered containing two sources belonging to different tectonic region types and both characterized by a G-R 
magnitude frequency distribution. The source model logic tree is as follows:
\begin{Verbatim}[frame=single, commandchars=\\\{\}, fontsize=\normalsize]
<?xml version="1.0" encoding="UTF-8"?>
<nrml xmlns:gml="http://www.opengis.net/gml"
      xmlns="http://openquake.org/xmlns/nrml/0.4">
    <logicTree logicTreeID="lt1">

        <logicTreeBranchingLevel branchingLevelID="bl1">
            <logicTreeBranchSet uncertaintyType="sourceModel"
                                branchSetID="bs1">
                <logicTreeBranch branchID="b11">
                    <uncertaintyModel>
                     source_model.xml
                    </uncertaintyModel>
                    <uncertaintyWeight>1.0</uncertaintyWeight>
                </logicTreeBranch>
            </logicTreeBranchSet>
        </logicTreeBranchingLevel>

        <logicTreeBranchingLevel branchingLevelID="bl2">
            <logicTreeBranchSet uncertaintyType="bGRRelative"
                                branchSetID="bs21">
                <logicTreeBranch branchID="b21">
                    <uncertaintyModel>+0.1</uncertaintyModel>
                    <uncertaintyWeight>0.333</uncertaintyWeight>
                </logicTreeBranch>
                <logicTreeBranch branchID="b22">
                    <uncertaintyModel>0.0</uncertaintyModel>
                    <uncertaintyWeight>0.333</uncertaintyWeight>
                </logicTreeBranch>
                <logicTreeBranch branchID="b23">
                    <uncertaintyModel>-0.1</uncertaintyModel>
                    <uncertaintyWeight>0.334</uncertaintyWeight>
                </logicTreeBranch>
            </logicTreeBranchSet>
        </logicTreeBranchingLevel>

        <logicTreeBranchingLevel branchingLevelID="bl3">
            <logicTreeBranchSet uncertaintyType="maxMagGRRelative"
                                branchSetID="bs31">
                <logicTreeBranch branchID="b31">
                    <uncertaintyModel>0.0</uncertaintyModel>
                    <uncertaintyWeight>0.333</uncertaintyWeight>
                </logicTreeBranch>
                <logicTreeBranch branchID="b32">
                    <uncertaintyModel>+0.5</uncertaintyModel>
                    <uncertaintyWeight>0.333</uncertaintyWeight>
                </logicTreeBranch>
                <logicTreeBranch branchID="b33">
                    <uncertaintyModel>+1.0</uncertaintyModel>
                    <uncertaintyWeight>0.334</uncertaintyWeight>
                </logicTreeBranch>
            </logicTreeBranchSet>
        </logicTreeBranchingLevel>

    </logicTree>
</nrml>
\end{Verbatim}
After the first branching level defining the source model, two additional branching levels are defined, one defining
relative uncertainties on b value (\texttt{bGRRelative} applied consistently to all sources in the source model)
and the second uncertainties on maximum magnitude (\texttt{maxMagGRRelative}). Similarly to the other cases,
two GMPEs are considered for each tectonic region type and therefore the total number of logic tree path is 36
(1 source model x 3 b value increments x 3 maximum magnitude increments x 2 GMPE for Active x 2 GMPEs for Stable)

\subsubsection{Event Based PSHA}
A demo showing an example of Event Based calculation is provided with the following configuration file:
\begin{Verbatim}[frame=single, commandchars=\\\{\}, fontsize=\normalsize]
[general]

description = Event Based PSHA using Area Source
calculation_mode = event_based
random_seed = 23

[geometry]

sites = 0.5 -0.5

[logic_tree]

number_of_logic_tree_samples = 0

[erf]

rupture_mesh_spacing = 2
width_of_mfd_bin = 0.1
area_source_discretization = 5.0

[site_params]

reference_vs30_type = measured
reference_vs30_value = 600.0
reference_depth_to_2pt5km_per_sec = 5.0
reference_depth_to_1pt0km_per_sec = 100.0

[calculation]

source_model_logic_tree_file = source_model_logic_tree.xml
gsim_logic_tree_file = gmpe_logic_tree.xml
investigation_time = 50.0
intensity_measure_types = PGA
intensity_measure_types_and_levels = {"PGA": [...]}
truncation_level = 3
maximum_distance = 200.0

[event_based_params]

ses_per_logic_tree_path = 100
ground_motion_correlation_model =
ground_motion_correlation_params =

[output]

export_dir = ...
ground_motion_fields = true
hazard_curves_from_gmfs = true
mean_hazard_curves = false
quantile_hazard_curves =
hazard_maps = true
poes = 0.1
\end{Verbatim}
The source model consist of one source (area). 100 stochastic event sets are generated (\texttt{ses\_\-per\_\-logic\_\-tree\_\-path = 100}). Ground motion fields
are computed (\texttt{ground\_\-motion\_\-fields = true}) and also hazard curves from ground motion fields are extracted (\texttt{hazard\_\-curves\_\-from\_\-gmfs = true}).
Corresponding hazard maps for 0.1 probability are additionally calculated(\texttt{hazard\_\-maps = true})

\subsubsection{Disaggregation}
An example of disaggregation calculation is given considering a source model consisting of two sources (area and simple fault) belonging to two different tectonic region types.
The calculation is described by the following configuration file:
\begin{Verbatim}[frame=single, commandchars=\\\{\}, fontsize=\normalsize]
[general]

description = ...
calculation_mode = disaggregation
random_seed = 23

[geometry]

sites = 0.5 -0.5

[logic_tree]

number_of_logic_tree_samples = 0

[erf]

rupture_mesh_spacing = 2
width_of_mfd_bin = 0.1
area_source_discretization = 5.0

[site_params]

reference_vs30_type = measured
reference_vs30_value = 600.0
reference_depth_to_2pt5km_per_sec = 5.0
reference_depth_to_1pt0km_per_sec = 100.0

[calculation]

source_model_logic_tree_file = source_model_logic_tree.xml
gsim_logic_tree_file = gmpe_logic_tree.xml
investigation_time = 50.0
intensity_measure_types_and_levels = {"PGA": [...]}
truncation_level = 3
maximum_distance = 200.0

[disaggregation]

poes_disagg = 0.1
mag_bin_width = 1.0
distance_bin_width = 10.0
coordinate_bin_width = 0.2
num_epsilon_bins = 3

[output]

export_dir = ...
\end{Verbatim}
Disaggregation matrices are computed for a single site (located between the two sources) for a ground motion value corresponding to a probability value equal to 0.1
(\texttt{poes\_\-disagg = 0.1}). Magnitude values are classified in one magnitude unit bins (\texttt{mag\_\-bin\_\-width = 1.0}), distances in bins of 10 km
(\texttt{distance\_\-bin\_\-width = 10.0}), coordinates in bins of 0.2 degrees (\texttt{coordinate\_\-bin\_\-width = 0.2}). 3 epsilons bins are considered (\texttt{num\_\-epsilon\_\-bins = 3}).

\begin{comment}
The demo allows to compute 

\section{Demo 01 - Classical PSHA}

% -----------------------------------------------------------------------------
\section{Demo 02 - Classical PSHA: simple logic tree}
This demo contains simple logic tree structures accounting for epistemic
uncertainties in the seismic source and ground motion intensity models.

The seismic source model incorporates epistemic uncertainty about the 
value of the maximum magnitude of the magnitude-frequency distribution 
used.
%
The ground motion intensity model includes uncertainty about the ground
motion prediction equations to be used in the calculation of hazard.

Given that the overall structure of the logic tree is not particularly
complex and assuming that uncertainties are fully correlated we decide 
to compute all the possible realisations of the logic tree by fixing 
the \texttt{number\_\-of\_\-logic\_\-tree\_\-samples} parameter in the 
configuration file to zero.
\begin{Verbatim}[frame=single, commandchars=\\\{\}, fontsize=\normalsize]
[logic_tree]
number_of_logic_tree_samples = 0
\end{Verbatim}

Let's now run the OpenQuake:
\begin{Verbatim}[frame=single, fontsize=\normalsize]
user@ubuntu:~/demos/classical_psha_simple_lt$ openquake \ 
--rh job_1strike.ini 
\end{Verbatim}

This is the list of results that we get at the end of this calculation:
\begin{Verbatim}[frame=single, commandchars=\\\{\}, fontsize=\normalsize]
Calculation 8 results:
id | output_type | name
5 | hazard_curve | hc-rlz-10
6 | hazard_curve | hc-rlz-7
7 | hazard_curve | hc-rlz-8
8 | hazard_curve | hc-rlz-9
9 | hazard_curve | hc-rlz-11
10 | hazard_curve | hc-rlz-12
11 | hazard_map | hazard-map(0.1)-PGA-rlz-10
12 | hazard_map | hazard-map(0.1)-PGA-rlz-7
13 | hazard_map | hazard-map(0.1)-PGA-rlz-8
14 | hazard_map | hazard-map(0.1)-PGA-rlz-9
15 | hazard_map | hazard-map(0.1)-PGA-rlz-11
16 | hazard_map | hazard-map(0.1)-PGA-rlz-12
\end{Verbatim}
OpenQuake produced six hazard curves and six hazard maps i.e. one
result for each leaf of the logic tree. 
\end{comment}