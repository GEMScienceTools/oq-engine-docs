% -----------------------------------------------------------------------------
\section{Demo 01 - Classical PSHA}

% -----------------------------------------------------------------------------
\section{Demo 02 - Classical PSHA: simple logic tree}
This demo contains simple logic tree structures accounting for epistemic
uncertainties in the seismic source and ground motion intensity models.

The seismic source model incorporates epistemic uncertainty about the 
value of the maximum magnitude of the magnitude-frequency distribution 
used.
%
The ground motion intensity model includes uncertainty about the ground
motion prediction equations to be used in the calculation of hazard.

Given that the overall structure of the logic tree is not particularly
complex and assuming that uncertainties are fully correlated we decide 
to compute all the possible realisations of the logic tree by fixing 
the \texttt{number\_\-of\_\-logic\_\-tree\_\-samples} parameter in the 
configuration file to zero.
\begin{Verbatim}[frame=single, commandchars=\\\{\}, fontsize=\normalsize]
[logic_tree]
number_of_logic_tree_samples = 0
\end{Verbatim}

Let's now run the OpenQuake:
\begin{Verbatim}[frame=single, fontsize=\normalsize]
user@ubuntu:~/demos/classical_psha_simple_lt$ openquake \ 
--rh job_1strike.ini 
\end{Verbatim}

This is the list of results that we get at the end of this calculation:
\begin{Verbatim}[frame=single, commandchars=\\\{\}, fontsize=\normalsize]
Calculation 8 results:
id | output_type | name
5 | hazard_curve | hc-rlz-10
6 | hazard_curve | hc-rlz-7
7 | hazard_curve | hc-rlz-8
8 | hazard_curve | hc-rlz-9
9 | hazard_curve | hc-rlz-11
10 | hazard_curve | hc-rlz-12
11 | hazard_map | hazard-map(0.1)-PGA-rlz-10
12 | hazard_map | hazard-map(0.1)-PGA-rlz-7
13 | hazard_map | hazard-map(0.1)-PGA-rlz-8
14 | hazard_map | hazard-map(0.1)-PGA-rlz-9
15 | hazard_map | hazard-map(0.1)-PGA-rlz-11
16 | hazard_map | hazard-map(0.1)-PGA-rlz-12
\end{Verbatim}
OpenQuake produced six hazard curves and six hazard maps i.e. one
result for each leaf of the logic tree. 
