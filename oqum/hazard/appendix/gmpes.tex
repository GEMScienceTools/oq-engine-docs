\label{sec:gmpes_list}
We provide below a list of the ground motion prediction equations 
implemented in the \gls{acr:hazlib}. All the implemented \gls{acr:gmpe}
use moment magnitude as the reference magnitude. For each GMPE,
the \gls{acr:oqe} name, a short description, and the corresponding reference
are given.
%
\section[GMPEs for shallow earthquakes in active tectonic regions]{Ground motion prediction equations for shallow earthquakes in active tectonic regions}
\begin{itemize} 
    \item AbrahamsonSilva2008 \hfill \\ A ground motion prediction equation 
        developed in the context of the NGA West project 
		\footnote{\href{http://peer.berkeley.edu/ngawest/}{http://peer.berkeley.edu/ngawest}}.
        The model is applicable to magnitudes 5-8.5, distances 0-200 km, and
        spectral periods 0-10 sec (\cite{abrahamson2008}).
    \item AkkarBommer2010 \hfill \\ A ground motion prediction equation 
        developed using mostly data from Europe and the Middle East. The 
		dataset 
        used to derive these equations contains events with moment 
        magnitude between 5 and 7.6 and distances up to 100 km 
		(\cite{akkar2010}).
    \item AkkarCagnan2010 \hfill \\ A ground motion prediction equation for 
		shallow
        earthquakes in active tectonic regions developed using data from the 
        Turkish strong-motion database. Equations are valid for a distance 
        %(\gls{acr:rjb}) range of 0–200 km and are derived for moment magnitudes 
        range of 0–200 km and are derived for moment magnitudes 
        between 5 and 7.6 (\cite{akkar2010a}).
    \item BooreAtkinson2008 \hfill \\ A ground motion prediction equation 
        for shallow earthquakes in active tectonic regions developed in 
        the context of the NGA West project.
        The model is applicable to magnitude in the range 5-8, 
		distances $<$ 200 km,
        and spectral periods 0-10 (\cite{boore2008}).
    \item CauzziFaccioli2008 \hfill \\ A ground motion prediction equation 
        derived from global data base of shallow crustal earthquakes (vast 
		majority coming from Japan) with magnitudes in
        range 5-7.2 and distances $<$ 150.0 (\cite{cauzzi2008}).
    \item ChiouYoungs2008 \hfill \\ A ground motion prediction equation 
        for shallow earthquakes in active tectonic regions developed in 
        the context of the NGA West \footnote{	
		\href{http://peer.berkeley.edu/ngawest/}{http://peer.berkeley.edu/ngawest}}.
        The model is supposed to be applicable for magnitude in range 4-8.5 (if 
		strike-slip), 4-8 (if normal or reverse) and distances 0-200 km.
    \item FaccioliEtAl2010 \hfill \\ Based on the same functional form of 
	    \cite{cauzzi2008} but using closest distance to the rupture instead of 
		hypocentral distance (\cite{faccioli2010})
    \item SadighEtAl1997 \hfill \\ A ground motion prediction based primarily 
	    on strong motion data 
        from California and applicable for magnitude in range 4-8 and distances 
		$<$ 100 km (\cite{sadigh1997}).
    \item ZhaoEtAl2006Asc \hfill \\ A ground motion prediction equation 
        for active shallow crust events developed using mostly japanese strong 
		ground motion recordings (\cite{zhao2006}).
\end{itemize}
%
\section[GMPEs for subduction sources]{Ground motion prediction equations for subduction sources}
\begin{itemize}
    \item AtkinsonBoore2003SInter, AtkinsonBoore2003SSlab \hfill \\ Ground 
	    motion prediction equations for 
        subduction interface and in-slab events obtained using a global 
        dataset of subduction earthquakes with moment magnitude between 
        5.0 and 8.3 (\cite{atkinson2003}).
    \item LinLee2008SInter, LinLee2008SSlab \hfill \\ Ground motion 
	    prediction equations for subduction
        interface and in-slab events created using strong motion
        data included in the the Taiwanese database (\cite{lin2008}).
    \item YoungsEtAl1997SInter, YoungsEtAl1997SSlab \hfill \\ One of the most  
	    well known ground motion 
        prediction equations for subduction earthquakes. Published in 1997,
        is still currently used for the calculation of the ground motion 
        in subduction tectonic environments. This GMPE covers events of 
        moment magnitude greater than 5 occurred at a distance between 5
        and 500 km. The source-site distance metric is the \gls{acr:rrup}.      
		(\cite{youngs1997})
    \item ZhaoEtAl2006SInter, ZhaoEtAl2006SSlab \hfill \\ 
	    Ground motion pre\-dic\-tion
	    equa\-tions for subduction interface and in-slab developed using mostly
	    japanese strong ground motion recordings (\cite{zhao2006}).
\end{itemize}
%
\section[GMPEs for stable continental regions]{Ground motion prediction equations for stable continental regions}
\begin{itemize}
    \item AtkinsonBoore2006 \hfill \\ 
	A ground motion prediction equation for Eastern 
	North America derived from a stochastic finite fault model 
	(\cite{atkinson2006}).
    \item Campbell2003 \hfill \\ 
	Ground motion prediction equation calibrated for Eastern North America 
	applicable for events with magnitude greater than 5 and distances $<$ 70 km 
	(\cite{campbell2003}).
    \item ToroEtAl2002 \hfill \\ Ground motion prediction equation for rock 	
	sites in central and eastern North America
    based on the prediction of a stochastic ground-motion model. The model is 
	applicable for magnitudes in range 5-8 and distances in 1-500 km 
	(\cite{toro2002}).
\end{itemize}
