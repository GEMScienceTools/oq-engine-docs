<<<<<<< HEAD
\begin{itemize} 
    \item \cite{wells1994} - Probably the most well known magnitude-scaling 
    relationship.
\end{itemize}
=======
\label{sec:msr_list}

\section{Magnitude-scaling relationships for shallow earthquakes 
    in active tectonic regions}
We provide below a list of the magnitude-area scaling relationships 
implemented in the \gls{acr:hazlib}. 
\begin{itemize} 
    \item \cite{wells1994} - One of the most well known magnitude scaling
        relationship based on a global database of historical earthquake 
        ruptures. The implemented relationship is the one linking magnitude 
        to rupture area.
\end{itemize}
%
%\section{Magnitude-scaling relationships for subduction earthquakes}
%\begin{itemize}
%    \item 
%\end{itemize}
%%
%\section{Magnitude-scaling relationships stable continental regions}
%\begin{itemize}
%    \item
%\end{itemize}
%
%\section{Ground motion prediction equations for volcanic areas}
%\begin{itemize}
%    \item 
%\end{itemize}
>>>>>>> haz-marco
