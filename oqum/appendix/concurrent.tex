The OpenQuake-engine works by splitting a computation into a number of tasks
which are then processed in parallel. The user has the ability to
control the splitting procedure, at least to a certain extent, by
setting the parameter `concurrent\_tasks` in the job.ini file. The
engine will try to produce a number of tasks close to
`concurrent\_tasks`: it could be more, it could be less. The details
of the algorithm used can change depending on the release of the engine and
this is why they are not documented here. Instead, we will document
how you can set the parameter to a sensible value.

For instance, suppose you have a standard PC with an i7 processor with
8 hyperthreaded cores, i.e. 4 real cores.  You could set:

concurrent\_tasks = 16

and then each hyperthreaded core would process around 2 tasks, which
is a reasonable value. If your computation consumes a lot of memory,
you could increase `concurrent\_tasks`, thus producing more tasks
of smaller size, requiring less memory.

If you don't set the parameter, a default value is used. Currently the
default is set to 8 times the number of cores in your controller machine.
This default value for `concurrent\_tasks` is likely to change in the future and you should not rely on it if you are using a computer cluster. If you are not using a cluster, the default value should be a reasonable choice.

Now, suppose you have have a cluster with a controller node and 10 workers, each
of which has 8 hyperthreaded cores, making for 80 cores in total. In this scenario you could set:

concurrent\_tasks = 160

If you did not set the parameter, the default (assuming 8 cores on the
controller machine) would be 8 * 8 = 64 tasks, which is
not enough. The number of available cores on the workers is 80, so 16 cores
will remain unused. Our suggestion is to provide a value:

concurrent\_tasks = 2 * number of (hyperthread) cores in the workers

or more, if the computation has memory issues. With more tasks, less
memory is used, but more data is transferred and the computation
becomes slower.

If `concurrent\_tasks` is set to zero, the parallelization
is disabled and the job is executed by using a single core. This
is useful when debugging errors.