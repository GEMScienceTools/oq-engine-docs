OpenQuake-engine is the seismic hazard and risk calculation software developed by
the \glsdesc{acr:gem}. By following current standards in software
developments like test-driven development and continuous integration, the
\glsdesc{acr:oqe} aims at becoming an open, and community-driven tool for
seismic hazard and risk analysis.

The source code of the \glsdesc{acr:oqe} is available on a public web-based
repository at the following address:

\href{http://github.com/gem/oq-engine}{http://github.com/gem/oq-engine}.


% ------------------------------------------------------------------------------
\section{Running the OpenQuake-engine}
\index{Running OpenQuake!introduction}
\label{sec:running_oq_engine}

An \gls{acr:oqe} analysis is launched from the command line of a terminal.

A schematic list of the options that can be used for the execution of the
\gls{acr:oqe} can be obtained with the following command:

\begin{minted}[fontsize=\footnotesize,frame=single,bgcolor=lightgray]{shell-session}
user@ubuntu:~\$ oq-engine --help
\end{minted}

The result is the following:

\inputminted[firstline=1,fontsize=\footnotesize,frame=single]{shell-session}{oqum/help.txt}\\

% ------------------------------------------------------------------------------
\section{Concurrent Computing with OpenQuake}
\label{sec:concurrent_tasks}

The \glsdesc{acr:oqe} supports concurrent computing on both standalone
computers and computer clusters.

The \glsdesc{acr:oqe} works by splitting a computation into a number of tasks
which are then processed in parallel. The user has the ability to control the
splitting procedure, at least to a certain extent, by setting the parameter
\Verb+concurrent_tasks+ in the job.ini file. The \glsdesc{acr:oqe} will try to
produce a number of tasks close to \Verb+concurrent_tasks+: it could be more, it
could be less. The details of the algorithm used can change depending on the
release of the engine and this is why they are not documented here. Our
suggestion is to provide a value:

\Verb+concurrent_tasks+ = 2 * number of (hyperthread) cores in the workers

or more, if the computation has memory issues. With more tasks, less memory is
used, but more data is transferred and the computation becomes slower.

If \Verb+concurrent_tasks+ is set to zero, the parallelization is disabled and the
job is executed by using a single core. This is useful when debugging errors.
