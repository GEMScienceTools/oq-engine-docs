\setlength{\parskip}{0.5\baselineskip}%
\setlength{\parindent}{0pt}%

\setcounter{secnumdepth}{3}
\setcounter{tocdepth}{3}
% This is used to create the cover and to plot trees
\usepackage{pst-tree}
\usepackage{pstricks,pstricks-add,multido}
%
\usepackage{geometry}
\usepackage{moresize}

%\usepackage{chappg}

\usepackage{float}
\floatstyle{plaintop}
\newfloat{nrmlsmp}{th}{lop}[chapter]
\floatname{nrmlsmp}{Nrml sample}

%
%\usepackage{algorithmic}
% 
\usepackage{fancyvrb}
\usepackage{listings}
\usepackage{alltt}
\usepackage{gensymb}
%
\usepackage[section]{placeins}
% 
\usepackage{pbox}
% http://en.wikibooks.org/wiki/LaTeX/Indexing
\usepackage{makeidx} 
\makeindex
%
%\usepackage{subfigure}
% Figure caption settings
\usepackage[textfont=it,margin=10pt,font=small,labelfont=bf,labelsep=endash]{caption}
\usepackage{subcaption}
%
\usepackage{bm}
% Landscape package
\usepackage{lscape}
%
\usepackage{hyperref} 
\hypersetup{colorlinks=true}
\hypersetup{breaklinks=true}
% package for multiline comments
\usepackage{verbatim}

\usepackage{xcolor}
\usepackage{framed}
\usepackage[utf8]{inputenc}

%
% Package to create a glossary - It must be uploaded after hyperref
% to produce the glossary: makeglossaries OQB
%\usepackage[toc,acronym,nonumberlist,style=altlist]{glossaries}
\usepackage[toc,nonumberlist,style=altlist,acronym]{glossaries}
\glstoctrue
\makeglossaries
%
% - - - - - - - - - - - - - - - - - - - - - - - - - - - - - - - - Setting Fonts
% \renewcommand{\encodingdefault}{OT1}
\renewcommand{\encodingdefault}{OT1}
\renewcommand{\familydefault}{ppl}
% \renewcommand{\familydefault}{cmss}
% \renewcommand{\seriesdefault}{m}
% \renewcommand{\shapedefault}{up}

% - - - - - - - - - - - - - - - - - - - - - - - - - - - - - - - - - - - - - - -
\usepackage{amsmath}
% - - - - - - - - - - - - - - - - - - - - - - - - - - - - - - - - - - - - - - -
\usepackage{titlesec}
\usepackage[dvips]{graphicx}
% - - - - - - - - - - - - - - - - - - - - - - - - - - - - - - - - - - - - - - -
\usepackage{type1cm,eso-pic,color}

%\makeatletter
%\AddToShipoutPicture{
%    \setlength{\@tempdimb}{.5\paperwidth}
%    \setlength{\@tempdimc}{.5\paperheight}
%    \setlength{\unitlength}{1pt}
%    \put(\strip@pt\@tempdimb,\strip@pt\@tempdimc){
%        \makebox(0,0){\rotatebox{55}{
%        	\textcolor[gray]{0.85}{
%        		\fontsize{5cm}{5cm}
%        		\selectfont{DRAFT}}
%        	}
%        }
%	}
%}
%\makeatother

%
% Solves problems with margin notes
\usepackage{mparhack} 
	\setlength{\marginparwidth}{1.1in}
	\let\oldmarginpar\marginpar
	\renewcommand\marginpar[1]{\-\oldmarginpar[\raggedright\color{red01}
	\footnotesize #1]%
	{\raggedright\footnotesize #1}}
% Define some colors
	\definecolor{azure}{RGB}{240,255,255}
	\definecolor{honeydew}{RGB}{240,255,240}
	\definecolor{blue01}{RGB}{4,64,116}
	\definecolor{blue02}{RGB}{0,62,113}
	\definecolor{gray01}{rgb}{0.1,0.1,0.1}
	\definecolor{gray02}{rgb}{0.8,0.8,0.8}
	\definecolor{red01}{rgb}{0.5,0.0,0.0}
	\definecolor{orange00}{rgb}{1.0,0.74,0.53}
	\definecolor{orange01}{rgb}{0.9137,0.5882,0.0980}
	\definecolor{orange02}{rgb}{0.7608,0.4157,0.1804}
	\definecolor{orange03}{rgb}{0.6941,0.1843,0.1333}
\usepackage[english]{babel}
% Bibliography settings
\usepackage[square,colon]{natbib} % Extend bibligraphy functions
% Page numbering by Chapter
%\usepackage[auto]{chappg} 
%\pagenumbering{bychapter}
% 
% Define page properties
\usepackage{scrpage2}
	\pagestyle{scrheadings}
	\lofoot[]{}
	\refoot[]{}
	%\lofoot[]{\includegraphics[width=1.0cm]{./figures/openquake_logo.eps}}
	%\refoot[]{\includegraphics[width=1.0cm]{./figures/openquake_logo.eps}}
	%\renewcommand{\partpagestyle}{empty}
% - - - - - - - - - - - - - - - - - - - - - - - - - -  Reformatting PART Titles
\titleformat{\part}[display]
{\filleft\normalfont\sffamily}
{\textcolor{blue01}{\bfseries\large PART}\hspace{4pt}
	\bfseries\Huge\textcolor{blue01}{\thepart}}
{1pc}
{\Huge\bfseries\textcolor{blue01}}
[]
% - - - - - - - - - - - - - - - - - - - - - - - - - Reformatting CHAPTER Titles
% Titles: CHAPTER
\titleformat{\chapter}
	[display] % shape
	{\filleft\normalfont\sffamily} % format
	{\textcolor{blue01}{\bfseries\MakeUppercase{\chaptertitlename}} % label
	\hspace{4pt}\huge\bfseries\textcolor{blue01}{\thechapter}} 
	{1pc} % sep
	{\huge\bfseries\textcolor{blue01}} % Before
	[]
% - - - - - - - - - - - - - - - - - - - - - - - - - Reformatting SECTION Titles
% Titles: SECTION
\titleformat{\section}
	[hang] % shape
	{\vspace{.8ex}\Large\bfseries\color{blue01}} % format 
	{\textcolor{blue01}{\thesection.}} % label
	{.5em} % sep
	{} % before
	[] % after
% - - - - - - - - - - - - - - - - - - - - - - -  Reformatting SUBSECTION Titles
% Title: SUBSECTION
\titleformat{\subsection}
	[hang] % shape
	{\vspace{.8ex}\large\bfseries\color{blue01}} % format 
	{\textcolor{blue01}{\thesubsection.}} % label
	{.5em} % sep
	{} % before
	[] % after
%  - - - - - - - - - - - - - - - - - - - - -  Reformatting SUBSUBSECTION Titles 
% Title: SUBSUBSECTION
\titleformat{\subsubsection}
	[hang] % shape
	{\vspace{.8ex}\normalfont\bfseries\color{blue01}} % format 
	{\textcolor{blue01}{\thesubsubsection.}} % label
	{.5em} % sep
	{} % before
	[] % after
% - - - - - - - - - - - - - - - - - - - - - - -  Reformatting PARAGRAPH Titles 
% Title: PARAGRAPH
\titleformat{\paragraph}
	[hang] % shape
	{\vspace{.2ex}\normalfont\color{blue01}} % format 
	{} % label
	{} % sep
	{} % before
	[] % after
%
