This book aims to provide an explanation of the scientific basis 
and the methodologies adopted in the implementation of the OpenQuake engine, an open source code for seismic hazard and physical risk calculation. 
%
The book follows the traditional openness and transparency features of the 
\gls{acr:gem} as clearly indicated in the development principles of 
the OpenQuake engine. 

%
The \gls{acr:gem} initiative is a global collaborative effort with the aim to provide organisations and people with tools and resources for transparent assessment of earthquake risk anywhere in the world.
%
The OpenQuake engine is a fully integrated, flexible and scalable hazard and physical risk 
calculation engine whose development is at the core of \gls{acr:gem}'s
overall objectives.
% ------------------------------------------------------------------------------
\section{Basics of the Engine}
The implementation of the OpenQuake software officially started in Summer 2010 
following the experience gained in \gls{acr:gem}'s kick-off project GEM1 
\citep{gemfoundation2010}, during which an extensive appraisal of existing hazard 
and physical risk codes was performed \citep{danciu2010,crowley2010}
and prototype hazard and risk software were selected, designed and
implemented \citep{pagani2010,crowley2010a}.

The current version of the OpenQuake engine is Python code developed 
following the most common requirements of Open Source 
software development, such as a public repository, IRC channel and open mailing lists. 
The source code, released under an open source software license,
is freely and openly accessible on a web based repository 
(see \href{http://github.com/gem}{github.com/gem}) while the 
development process is managed so that the community can participate 
to the day by day development as well as in the mid- and long-term 
design process. 
%
The software development also leverages on a number of open source projects 
such as \href {http://celeryproject.org}{Celeryd} and \href{http://www.rabbitmq.com}{RabbitMQ}, just to mention a few.

The hazard component of the engine largely relies on classes belonging to the OpenQuake Hazard library (see \href{https://github.com/gem/oq-hazardlib}{oq-hazardlib}) a comprehensive library for performing state-of-the-art PSHA. This library has been designed and implemented following the successful collaboration and important lessons learnt working with the \href{http://www.opensha.org}{OpenSHA} software and the developing teams at \gls{acr:usgs} and 
\gls{acr:scec} in GEM1. 
%
The risk component of the engine was designed in GEM1, prototyped in Java and eventually
coded in Python by the team operating at the \gls{acr:gem} 
Model Facility. This scientific code was originally integrated with the engine, but in late 2012 it was extracted to form the OpenQuake Risk Library (see \href{https://github.com/gem/oq-risklib}{oq-risklib}).

%A schema that illustrates the structure of the engine is 
%represented in Figure \ref{fig:openquake_schema}; the schema contains:
%purple boxes representing the main modules of the hazard component, 
%green boxes showing the modules of the risk component, white boxes
%with main outputs computed by the distinct modules and orange rectangles
%displaying the main input information that should be entered into the calculation engine. 
%
% ------------------------------------------------------------------------------
%\subsection{Brief description of the OpenQuake IT architecture}
%% ------------------------------------------------------------------------------
Openquake is the core of the OpenGEM system. 
%
As it appears in Figure \ref{fig:oq_it}, OpenQuake is powered by a 
number of open source software projects, of which OpenSHA-lite and 
RiskLib are currently the most essential ones. 
OpenSHA-lite is a ‘light’ version of the comprehensive software for 
probabilistic seismic hazard assessment OpenSHA and whose code serves
as a basis for the hazard component of the OpenQuake engine. RiskLib 
is a global collaboration project that is aimed at common development 
of a code ‘library’ for the risk all types of natural hazards, including
an API, on wh ich ‘apps’ can be built, such as tools that support risk
mitigation/reduction. The World Bank’s GFDRR, OpenGeo, AIFDR and GEM 
jointly work on RiskLib and its code repository, currently part of 
the OpenQuake project, will be broken out soon.
%
% . . . . . . . . . . . . . . . . . . . . . . . . . . . . . . . . . . . > Figure
\begin{figure}
\includegraphics[width=\textwidth,angle=0]{./Figures/Part_Introduction/oq_system_architecture.eps}
\caption{Openquake system architecture.}
\label{fig:oq_it}
\end{figure}
% . . . . . . . . . . . . . . . . . . . . . . . . . . . . . . . . . . . < Figure
%
Implementation of a standard format for data exchange for natural hazard
and risk,  is closely related to what is described above. The NRML 
(Natural hazards’ Risk Markup Language) format has been developed, which
is capable of encompassing a wide variety of risk and hazard data formats.
More information can be found in the detailed background documentation 
for OpenQuake.

The data services that are mentioned above are globally accessible 
read/write collaborations of core data sets that are exposed with a 
REST API. Some of the core modules of GEM will be available through 
the services and are currently being developed by GEM’s Model Facility 
development team, in collaboration with a number of IT partners and the 
scientists coordinating the modules. Examples are the development of a 
global exposure database, development of a global portal for active 
faults, global vulnerability functions and an earthquake consequences
database.






%
% ------------------------------------------------------------------------------
\section{Structure of the Book}
The OpenQuake Engine Book is organized into two volumes, one which describes the science behind the hazard component of the engine
and another which illustrates the theory of the physical risk calculators incorporated into the software. Readers that are interested in learning how to run calculations using the OpenQuake engine are referred to the OpenQuake Engine User Manual.

\hfill \\
%\emph{Part II: Hazard}
The GEM Hazard Team is currently updating the volume on hazard. In the meantime, interested readers are referred to the OpenQuake Engine User Manual, or they can contact the coordinator, Dr Marco Pagani, for further information. 
%\begin{itemize}
%\item Chapter \ref{chap:inthaz} offers an introduction to the hazard 
%topics discussed in the following chapters. In particular, in this 
%Chapter we discuss the main OpenQuake concepts and we illustrate the 
%calculation workflows currently available in the hazard component of 
%OpenQuake.
%\item Chapter \ref{chap:hazinp} focuses on the structure and the 
%characteristics of the information necessary to define a 
%comprehensive PSHA input model. This Chapter also includes
%descriptions of the main seismic source typologies and of the logic 
%tree structure.
%\item We dedicate Chapter \ref{chap:erf} to the explanation of the
%methodology adopted for the processing of the logic tree structures
%supported by OpenQuake and for the creation of the 
%\gls{earthquakeruptureforecast}
%\item The last Chapter of the hazard part (chapter \ref{chap:hazcalc}) 
%illustrates the main calculators available: the classical-PSHA calculator,
%the event-based calculator and the disaggregation calculator. 
%\end{itemize}
\hfill \\
\hfill \\
%\emph{Part III: Risk}
This volume on risk is organised as follows:
\begin{itemize}
\item Chapter \ref{chap:intrisk} introduces the main physical risk concepts and workflows. 
\item Chapter \ref{chap:riskinput} contains an explanation of the exposure, physical vulnerability and fragility concepts, which are input to the calculations.
\item Chapter \ref{chap:scenario_risk} describes the scenario risk 
methodology, for estimating loss distributions for single events.
\item Chapter \ref{chap:scenario_damage} describes the scenario damage 
methodology, for estimating damage ditributions for single events.
\item Chapter \ref{chap:risk_prob_event_based} provides an overview of the probabilistic event-based risk calculation methodology, which produces loss exceedance curves for portfolios of buildings.
\item Chapter \ref{chap:risk_psha_based} illustrates single-asset risk calculations based on the hazard curves from classical PSHA.
\item Chapter \ref{chap:bcr} outlines the benefit/cost ratio calculations based on loss curves for structures with and without retrofitting.
\end{itemize}
%
%In the closing part, the Book contains a glossary that aims to define a 
%clear and unique terminology.
%
% . . . . . . . . . . . . . . . . . . . . . . . . . . . . . . . . . . . > Figure
%\begin{landscape}
%\begin{figure}
%\includegraphics[width=20cm,angle=0]{./Figures/Part_Introduction/engine9_20110130.eps}
%\caption{OpenQuake Engine schema. Purple boxes are the calculators included in the  
%the hazard part of OQ; green boxes are the physical risk calculators.}
%\label{fig:openquake_schema}
%\end{figure}
%\end{landscape}
% . . . . . . . . . . . . . . . . . . . . . . . . . . . . . . . . . . . < Figure