%%%%%%%%%%%%%%%%%%%%%%%%%%%%%%%%%%%%%%%%%%%%%%%%%%%%%%%%%%%%%%%%%%%%%%%%%%%%%%%%
% The GEM Technical Documentation LaTeX Template
% Version 1.0 (10/10/2015)
%
% Original author:
% Mathias Legrand (legrand.mathias@gmail.com)
%
% Adapted for use by the GEM Foundation by:
% James Brown (james.brown@globalquakemodel.org)
% Anirudh Rao (anirudh.rao@globalquakemodel.org)
%
% License:
% CC BY-NC-SA 3.0 (http://creativecommons.org/licenses/by-nc-sa/3.0/)
%
% Compiling this template:
% This template uses bibtex for its bibliography and makeindex for its index.
% When you first open the template, compile it from the command line with the
% commands below:
%
% 1) pdflatex -interaction=nonstopmode oq-manual.tex
% 2) bibtex oq-manual
% 3) pdflatex -interaction=nonstopmode oq-manual.tex
% 4) pdflatex -interaction=nonstopmode oq-manual.tex
% 5) makeindex oq-manual.idx -s configuration/StyleInd.ist
% 6) makeglossaries oq-manual
% 7) pdflatex -interaction=nonstopmode oq-manual.tex
%
% After this, when you wish to update the bibliography/index use the appropriate
% command above and make sure to compile with pdflatex several times
% afterwards to propagate your changes to the document.
%
% This template also uses a number of packages which may need to be updated
% to the newest versions for the template to compile. It is strongly
% recommended you update your LaTeX distribution if you have any
% compilation errors.
%
% Important note:
% Chapter heading images should have a 2:1 width:height ratio,
% e.g. 920px width and 460px height.
%
%%%%%%%%%%%%%%%%%%%%%%%%%%%%%%%%%%%%%%%%%%%%%%%%%%%%%%%%%%%%%%%%%%%%%%%%%%%%%%%%

%-------------------------------------------------------------------------------
%  PACKAGES AND OTHER DOCUMENT CONFIGURATION
%-------------------------------------------------------------------------------

% Page layout and margins
\usepackage[top=3cm,bottom=3cm,left=3.2cm,right=3.2cm,headsep=10pt,a4paper]{geometry}
\linespread{1.25}

% Font settings
\usepackage[condensed]{roboto} % Use the Roboto font for headings
\usepackage[bitstream-charter]{mathdesign} % Use the Bitstream Charter font for text
% \usepackage{amsmath,amsfonts,amssymb,amsthm} % For math equations, theorems, symbols, etc
\usepackage{microtype} % Slightly tweak font spacing for aesthetics
\usepackage[utf8]{inputenc} % Required for including letters with accents
\usepackage[T1]{fontenc} % Use 8-bit encoding that has 256 glyphs

% Color settings
\usepackage{color, colortbl}
\usepackage{xcolor} % Required for specifying colors by name
\definecolor{darkgray}{gray}{.25}
% Colors from the GEM Brand Guidelines
\definecolor{oqblue}{RGB}{27,117,165}
\definecolor{gembrown}{cmyk}{.53,.54,.55,.54}

% Bibliography settings
\usepackage{csquotes}
\usepackage[style=alphabetic,
            sorting=nyt,
            sortcites=true,
            natbib=true,
            style=authoryear,
            maxcitenames=2,
            maxbibnames=100,
            autopunct=true,
            autolang=hyphen,
            hyperref=true,
            doi=true,
            abbreviate=false,
            backref=true,
            backend=bibtex,
            bibencoding=ascii,
            firstinits=true,
	    	uniquename=false,
	    	uniquelist=false]{biblatex}
\addbibresource{bibliography/hazard.bib} % Hazard BibTeX bibliography file
\addbibresource{bibliography/risk.bib} % Risk BibTeX bibliography file
\defbibheading{bibempty}{}

% Figure caption settings
\usepackage[textfont=it,margin=10pt,font=small,labelfont=bf,labelsep=endash]{caption}
\usepackage{subcaption}

% Rotate any object
\usepackage{rotating}

% Verbatim environments
\usepackage{verbatim}
\usepackage{fancyvrb}

% Index settings
\usepackage{calc} % Infix notation for \setcounter, \addtocounter, \setlength, \addtolength
\usepackage{makeidx} % Required to make an index
\setcounter{secnumdepth}{3}
\setcounter{tocdepth}{3} % Entries down to \subsubsections in the TOC
\makeindex % Tells LaTeX to create the files required for indexing

\usepackage{todonotes}
\usepackage{marginnote}

% Bold symbols in maths mode
\usepackage{bm}

% Flexible typesetting of tables and figures
\usepackage{ctable}
\usepackage{booktabs}

% Customization of section titles and table of contents
\usepackage{titlesec}
\usepackage{titletoc}

% Header and footer customization
\usepackage{fancyhdr}
\usepackage{etoolbox}

\usepackage{graphicx} % Required for including pictures
\usepackage{tikz} % Required for drawing custom shapes
\usepackage{eso-pic} % Required for specifying an image background in the title page
\usepackage{pdfpages} % Needed to load .pdf pages, used for the cover page

% English language hyphenation
\usepackage[english]{babel}

\usepackage{enumitem} % Customize lists
\setlist{nolistsep} % Reduce spacing between bullet points and numbered lists
\usepackage{listings} % Required for embedding code snippets

\usepackage{hyperref}
\hypersetup{hidelinks,colorlinks=true,breaklinks=true,citecolor=oqblue,linkcolor=gembrown,urlcolor=oqblue,bookmarksopen=false,pdftitle={Title},pdfauthor={Author}}

% Package to create a glossary - must be loaded after hyperref
\usepackage[acronym,nonumberlist,style=altlist]{glossaries}
\glstoctrue
\makeglossaries


%----------------------------------------------------------------------------------------
%     MAIN TABLE OF CONTENTS
%----------------------------------------------------------------------------------------

% Remove the default margin
\contentsmargin{0cm}

% \titlecontents{section}
%                       [left]
%                       {above}
%                       {before with label}
%                       {before without label}
%                       {filler and page}
%                       [after]

% Part text styling
\titlecontents{part}
                    [0cm] % Indentation
                    {\addvspace{24pt}\Large\sffamily\bfseries} % Spacing and font options for parts
                    {\color{black!60}\contentslabel[\Large\thecontentslabel]{1.25cm}\color{black}} % Part number
                    {}
                    {\color{black!60}\Large\sffamily\bfseries\hfill\thecontentspage} % Page number
                    []

% Chapter text styling
\titlecontents{chapter}
                    [1.25cm] % Indentation
                    {\addvspace{18pt}\large\sffamily\bfseries} % Spacing and font options for chapters
                    {\color{darkgray!80}\contentslabel[\Large\thecontentslabel]{1.25cm}\color{darkgray}} % Chapter number
                    {\color{darkgray}}
                    {\color{darkgray!40}\large\sffamily\bfseries\;\titlerule*[.5pc]{.}\;\color{darkgray!80}\thecontentspage} % Page number
                    []

% Section text styling
\titlecontents{section}
                    [1.25cm] % Indentation
                    {\addvspace{12pt}\sffamily\bfseries} % Spacing and font options for sections
                    {\color{black!60}\contentslabel[\thecontentslabel]{1.25cm}\color{black}} % Section number
                    {}
                    {\color{black!20}\normalsize\sffamily\bfseries\;\titlerule*[.5pc]{.}\;\color{black!60}\thecontentspage} % Page number
                    []

% Subsection text styling
\titlecontents{subsection}
                    [1.25cm] % Indentation
                    {\addvspace{6pt}\sffamily\small} % Spacing and font options for subsections
                    {\color{black!60}\contentslabel[\thecontentslabel]{1.25cm}\color{black}} % Subsection number
                    {}
                    {\color{black!20}\normalsize\sffamily\;\titlerule*[.5pc]{.}\;\color{black!60}\thecontentspage} % Page number
                    []

% Subsubsection text styling
\titlecontents{subsubsection}
                    [1.25cm] % Indentation
                    {\addvspace{3pt}\sffamily\footnotesize} % Spacing and font options for subsubsections
                    {\color{black!60}\contentslabel[\thecontentslabel]{1.25cm}\color{black}\em} % Subsubsection number
                    {}
                    {\color{black!20}\normalsize\sffamily\;\titlerule*[.5pc]{.}\;\color{black!60}\thecontentspage} % Page number
                    []

%----------------------------------------------------------------------------------------
%     MINI TABLE OF CONTENTS IN CHAPTER HEADS
%----------------------------------------------------------------------------------------

% Section text styling
\titlecontents{lsection}[0em] % Indendation
{\footnotesize\sffamily} % Font settings
{}
{}
{}

% Subsection text styling
\titlecontents{lsubsection}[.5em] % Indentation
{\normalfont\footnotesize\sffamily} % Font settings
{}
{}
{}

%----------------------------------------------------------------------------------------
%     PAGE HEADERS
%----------------------------------------------------------------------------------------

% Patch fancyhdr to set the font and rule colors for headers and footers
\makeatletter
\patchcmd{\@fancyhead}{\rlap}{\color{oqblue}\rlap}{}{}
\patchcmd{\headrule}{\hrule}{\color{black}\hrule}{}{}
\patchcmd{\@fancyfoot}{\rlap}{\color{oqblue}\rlap}{}{}
\patchcmd{\footrule}{\hrule}{\color{black}\hrule}{}{}
\makeatother

\pagestyle{fancy}
\renewcommand{\chaptermark}[1]{\markboth{\sffamily\normalsize\bfseries\chaptername\ \thechapter.\ #1}{}} % Chapter text font settings
\renewcommand{\sectionmark}[1]{\markright{\sffamily\normalsize\thesection\hspace{5pt}#1}{}} % Section text font settings
\fancyhf{} \fancyhead[LE,RO]{\sffamily\normalsize\thepage} % Font setting for the page number in the header
\fancyhead[LO]{\rightmark} % Print the nearest section name on the left side of odd pages
\fancyhead[RE]{\leftmark} % Print the current chapter name on the right side of even pages
\renewcommand{\headrulewidth}{0.5pt} % Width of the rule under the header
\addtolength{\headheight}{7.5pt} % Increase the spacing around the header
\renewcommand{\footrulewidth}{0pt} % Removes the rule in the footer
\fancypagestyle{plain}{\fancyhead{}\renewcommand{\headrulewidth}{0pt}} % Style for when a plain pagestyle is specified

% Remove the header from odd empty pages at the end of chapters
\makeatletter
\renewcommand{\cleardoublepage}{
\clearpage\ifodd\c@page\else
\hbox{}
\vspace*{\fill}
\thispagestyle{empty}
\newpage
\fi}
\makeatother

%----------------------------------------------------------------------------------------
%     SECTION NUMBERING IN THE MARGIN
%----------------------------------------------------------------------------------------

\makeatletter
\renewcommand{\@seccntformat}[1]{\llap{\textcolor{oqblue}{\csname the#1\endcsname}\hspace{1em}}}
\renewcommand{\section}{\@startsection{section}{1}{\z@}
{-4ex \@plus -1ex \@minus -.4ex}
{1ex \@plus.2ex }
{\normalfont\large\sffamily\bfseries}}
\renewcommand{\subsection}{\@startsection {subsection}{2}{\z@}
{-3ex \@plus -0.1ex \@minus -.4ex}
{0.5ex \@plus.2ex }
{\normalfont\sffamily\bfseries}}
\renewcommand{\subsubsection}{\@startsection {subsubsection}{3}{\z@}
{-2ex \@plus -0.1ex \@minus -.2ex}
{.2ex \@plus.2ex }
{\normalfont\small\sffamily\bfseries}}
\renewcommand{\paragraph}{\@startsection {paragraph}{4}{\z@}
{-2ex \@plus-.2ex \@minus .2ex}
{.1ex}
{\normalfont\small\sffamily\bfseries\em}}
\makeatother

%----------------------------------------------------------------------------------------
%     PART HEADINGS
%----------------------------------------------------------------------------------------

% \titleformat{command}[shape]{format}{label}{sep}{before}[after]
\titleformat{\part}[display]{\bfseries\filcenter\Huge\sffamily}{\textcolor{gembrown}{\partname~\thepart}}{20pt}{\textcolor{gembrown}}

%----------------------------------------------------------------------------------------
%     CHAPTER HEADINGS
%----------------------------------------------------------------------------------------

% The set-up below should be manually adapted to the overall page
% layout and margin setup controlled by the geometry package.

\newcommand{\thechapterimage}{}
\newcommand{\chapterimage}[1]{\renewcommand{\thechapterimage}{#1}}

% Numbered chapters with mini tableofcontents
\makeatletter
\def\thechapter{\arabic{chapter}}
\def\@makechapterhead#1{
\thispagestyle{empty}
{\centering \normalfont\sffamily
\ifnum \c@secnumdepth >\m@ne
\if@mainmatter
\startcontents
\begin{tikzpicture}[remember picture,overlay]
\node at (current page.north west)
{\begin{tikzpicture}[remember picture,overlay]
\node[anchor=north west,inner sep=0pt] at (0,0) {\includegraphics[width=\paperwidth]{\thechapterimage}};

% Commenting the 3 lines below removes the small contents box in the chapter heading
\fill[color=oqblue!10!white,opacity=.6] (1cm,0) rectangle (12cm,-7cm);
\node[anchor=north west] at (1.5cm,.05cm) {\parbox[t][6.9cm][t]{10.5cm}{
    \huge\bfseries\flushleft \printcontents{l}{1}{\setcounter{tocdepth}{2}}}};
% The 3 lines below control the box environment for chapter titles
\draw[anchor=west] (1.8cm,-10cm) node [fill=white,text opacity=1,draw=white,draw opacity=1,line width=1.5pt,fill opacity=.6,inner sep=12pt]{\huge\sffamily\bfseries\textcolor{oqblue}{\thechapter.\hspace{0.35cm}#1\strut\makebox[22cm]{}}};

\end{tikzpicture}};
\end{tikzpicture}}
\par\vspace*{230\p@}
\fi
\fi}

% Unnumbered chapters without mini tableofcontents
\def\@makeschapterhead#1{
\thispagestyle{empty}
{\centering \normalfont\sffamily
\ifnum \c@secnumdepth >\m@ne
\if@mainmatter
\begin{tikzpicture}[remember picture,overlay]
\node at (current page.north west)
{\begin{tikzpicture}[remember picture,overlay]
\node[anchor=north west,inner sep=0pt] at (0,0) {\includegraphics[width=\paperwidth]{\thechapterimage}};
\draw[anchor=west] (2.6cm,-11cm) node [fill=white,text opacity=1,draw=white,draw opacity=1,line width=1.5pt,fill opacity=.6,inner sep=12pt]{\huge\sffamily\bfseries\textcolor{oqblue}{#1\strut\makebox[22cm]{}}};
\end{tikzpicture}};
\end{tikzpicture}}
\par\vspace*{230\p@}
\fi
\fi
}
\makeatother

%----------------------------------------------------------------------------------------
%     PYTHON ENVIRONMENT
%----------------------------------------------------------------------------------------
\lstset{language=Python}
\definecolor{Code}{rgb}{0,0,0}
\definecolor{Decorators}{rgb}{0.5,0.5,0.5}
\definecolor{Numbers}{rgb}{0.5,0,0}
\definecolor{MatchingBrackets}{rgb}{0.25,0.5,0.5}
\definecolor{Keywords}{rgb}{0,0,1}
\definecolor{self}{rgb}{0,0,0}
\definecolor{Strings}{rgb}{0,0.63,0}
\definecolor{Comments}{rgb}{0,0.63,1}
\definecolor{Backquotes}{rgb}{0,0,0}
\definecolor{Classname}{rgb}{0,0,0}
\definecolor{FunctionName}{rgb}{0,0,0}
\definecolor{Operators}{rgb}{0,0,0}
\definecolor{Background}{rgb}{0.98,0.98,0.98}

\lstnewenvironment{python}[1][]{
\lstset{
numbers=left,
numberstyle=\footnotesize,
numbersep=1em,
xleftmargin=1em,
framextopmargin=2em,
framexbottommargin=2em,
showspaces=false,
showtabs=false,
showstringspaces=false,
frame=l,
tabsize=4,
basicstyle=\ttfamily\footnotesize,
backgroundcolor=\color{Background},
language=Python,
commentstyle=\color{Comments}\slshape,
stringstyle=\color{Strings},
morecomment=[s][\color{Strings}]{"""}{"""},
morecomment=[s][\color{Strings}]{'''}{'''},
morekeywords={import,from,class,def,for,while,if,is,in,elif,else,not,and,or,print,break,continue,return,True,False,None,access,as,,del,except,exec,finally,global,import,lambda,pass,print,raise,try,assert},
keywordstyle={\color{Keywords}\bfseries},
morekeywords={[2]@invariant},
keywordstyle={[2]\color{Decorators}\slshape},
emph={self},
emphstyle={\color{self}\slshape},
}}{}